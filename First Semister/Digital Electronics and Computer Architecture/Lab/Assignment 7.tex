\documentclass[11pt]{article}

% Preamble

\usepackage[margin=1in]{geometry}
\usepackage{amsfonts, amsmath, amssymb}
\usepackage{fancyhdr, float, graphicx}
\usepackage[utf8]{inputenc} % Required for inputting international characters
\usepackage[T1]{fontenc} % Output font encoding for international characters
\usepackage{fouriernc} % Use the New Century Schoolbook font
\usepackage[nottoc, notlot, notlof]{tocbibind}
\usepackage{listings}
\usepackage{xcolor}
\usepackage{karnaugh-map}
% \usepackage[table,xcdraw]{xcolor}

\definecolor{codegreen}{rgb}{0,0.6,0}
\definecolor{codegray}{rgb}{0.5,0.5,0.5}
\definecolor{codepurple}{rgb}{0.58,0,0.82}
\definecolor{backcolour}{rgb}{0.95,0.95,0.92}

\lstdefinestyle{mystyle}{
    backgroundcolor=\color{backcolour},   
    commentstyle=\color{codegreen},
    keywordstyle=\color{magenta},
    numberstyle=\tiny\color{codegray},
    stringstyle=\color{codepurple},
    basicstyle=\ttfamily\footnotesize,
    breakatwhitespace=false,         
    breaklines=true,                 
    captionpos=b,                    
    keepspaces=true,                 
    numbers=left,                    
    numbersep=5pt,                  
    showspaces=false,                
    showstringspaces=false,
    showtabs=false,                  
    tabsize=2
}

\lstset{style=mystyle}

% Header and Footer
\pagestyle{fancy}
\fancyhead{}
\fancyfoot{}
\fancyhead[L]{\textit{\Large{DECA Assignment 6}}}
%\fancyhead[R]{\textit{something}}
\fancyfoot[C]{\thepage}
\renewcommand{\footrulewidth}{1pt}



% Other Doc Editing
% \parindent 0ex
%\renewcommand{\baselinestretch}{1.5}

\begin{document}

\begin{titlepage}
	\centering

	%---------------------------NAMES-------------------------------

	\huge\textsc{
		MIT World Peace University
	}\\

	\vspace{0.75\baselineskip} % space after Uni Name

	\LARGE{
		Digital Electronics and Computer Architecture\\
		Second Year B. Tech, Semester 3
	}

	\vfill % space after Sub Name

	%--------------------------TITLE-------------------------------

	\rule{\textwidth}{1.6pt}\vspace*{-\baselineskip}\vspace*{2pt}
	\rule{\textwidth}{0.6pt}
	\vspace{0.75\baselineskip} % Whitespace above the title



	\huge{\textsc{
			Write an Assembly Language Program to Display 2 Digit and 4 Digit Hexadecimal Numbers using 64 Bit Assembly.
		}} \\



	\vspace{0.5\baselineskip} % Whitespace below the title
	\rule{\textwidth}{0.6pt}\vspace*{-\baselineskip}\vspace*{2.8pt}
	\rule{\textwidth}{1.6pt}

	\vspace{1\baselineskip} % Whitespace after the title block

	%--------------------------SUBTITLE --------------------------	

	\LARGE\textsc{
		Practical Report\\
		Assignment 7
	} % Subtitle or further description
	\vfill

	%--------------------------AUTHOR-------------------------------

	\vspace{0.5\baselineskip} % Whitespace before the editors

	\Large{
		Krishnaraj Thadesar \\
		Cyber Security and Forensics\\
		Batch A1, PA 20
	}


	\vspace{0.5\baselineskip} % Whitespace below the editor list
	\today

\end{titlepage}


\tableofcontents
\thispagestyle{empty}
\clearpage


\section{\textbf{Problem Statement}}
Write an assembly language program (ALP) to display 2-digit and 4-digit hex numbers using 64-bit assembly language programming.

\section{\textbf{Objective}}
\begin{itemize}
    \item To understand the structure of the assembly language program.
    \item To understand system call function for write and exit.
\end{itemize}

\section{\textbf{Platform}}
CPU - Core i7 Duo, 64 bit with 4 GHz clock frequency.\\
OS - Arch Linux, 64 bit\\
Editor - VS code \\
Assembler - NASM (Netwide Assembler)\\
Linker - LD, GNU linker.

\section{\textbf{Theory}}
\subsection{\textbf{Assembly Language Program Basic Structure}}
\begin{verbatim}
    section.Data (declare data segment)
    ; initialized data declaration 
    section.bss (declare block started by segment sort)
    ; uninitialized data declaration.
    section.text (declare code segment)
    global-start (entry point for program)

    -start:
    ;code.

    (;semicolon is used to give the comment)

\end{verbatim}

\subsection{\textbf{System calls to write and exit.}}
\subsubsection{\textbf{System Call to write/output call}}
display variable -name contents of specified variable length on monitor.
\begin{verbatim}
    mov rax,1   ; function number for writing/outputting the data.
    mov rdi,0   ; file descriptor ID for standard input device (keyboard)
    mov rsi,arr ; starting addresses of the variable used to store the data.
    mov rdx,8   ; maximum bytes to be read.
    syscall     ; system call (in built function)
\end{verbatim}
\subsubsection{\textbf{System exit call}}

function to exit or terminate program.
\begin{verbatim}
    mov rax,60 ; function number for sys-exit
    mov rdi,0  ; return code for zero error.
    syscall    ; system call.
\end{verbatim}
\subsection{\textbf{Instruction used in the program for implementation}}
\begin{verbatim}
    ADD
    add two numbers together

    COMPARE
    compare numbers

    JUMP
    jump to designated RAM address.

    LOAD 
    Load information from RAM to the CPU.
\end{verbatim}

\subsection{\textbf{Commands to execute the program}}
\begin{verbatim}
    to assemble:
    nasm -f elf64 hello.asm

    to link:
    ld -o hello hello.o

    to execute:
    ./hello

    where, hello is the filename.
\end{verbatim}

\section{\textbf{Algorithm}}
\begin{enumerate}
    \item start
    \item Display message "Two-digit HEX Number"
    \item Initialize hard coded two digit and four-digit number.
    \item Write a procedure for unpacking BCD number (Display Ouputs)
    \item Display two digit and four digit numbers.
    \item end.
\end{enumerate}
\section{\textbf{Input}}
Two digit and four digit numbers.

\section{\textbf{Output}}
Two digit and four digit numbers.
\begin{verbatim}
The Two digit Hex number is: 
2A
\end{verbatim}
\section{Code}
\lstinputlisting[]{../Programs/assignment_1.asm}

\section{\textbf{Conclusion}}
Thus, implemented the program in assembly language to display two digit and four-digit hex numbers

\section{\textbf{FAQs}}
\begin{enumerate}
	\item {\textbf{Explain assembler directives. List the assembler directives in your program.}}
	
	Assembler Directives supply data to the program and control the assembly process. It enables to do the following:
\begin{itemize}
    \item Assemble code and data into specified sections. 
    \item Reserve space in memory for unitized variables.
    \item Control the appearance of listings.
    \item Initialize memory.
    \item Assemble conditional blocks.
    \item Define global variables.
    \item Specify libraries from which the assembler can obtain macros.
    \item Examine symbolic debugging information.
\end{itemize}

Assembler Directives in the program:
\begin{itemize}
    \item .text      switch to text segment.
    \item .data      switch to initialized part of data segment.
    \item .bss       switch to uninitialized port of data segment.
\end{itemize}

\item {\textbf{Illustrate the significance of the sections: .data, .bss, .text}}
.bss segment stands for 'block starting symbol' is the memory space for uninitialized variable of your code . IT is the method of optimization to reduce the code size.\\
syntax: section.bss\\
\indent var-name RES memory-Type memory size.\\
\begin{verbatim}
    Eg. section.bss
        A resb 5D
        (Declare variable A allocate 50 bytes memory)
\end{verbatim}

.data section holds the initialized value. It holds the data of the initialized variable (global or local)
\begin{verbatim}
syntax:
    section.data
    var_name data_type variable_value.
Eg:
    A DB 50
    (declared variable A of type byte with value 50)

\end{verbatim}

.text segment is the code, vector table and constants. It is the section that holds the executable instructions. 
\begin{verbatim}
syntax:
       section.text
       global_start
       _start:
       ;code
\end{verbatim}

\item {\textbf{What is the difference between RESB and DB?}}

DB stands for Declare/Define Byte. It is a directive that is used to allocate space for initialized data in the data section.\\
\\
RESB Stands for Reserve Byte. It is a directive that is used to allocate space for uninitialized data in .bss section.

\end{enumerate}


\end{document}