
\documentclass[11pt]{article}
 
% Preamble

\usepackage[margin=1in]{geometry}
\usepackage{amsfonts, amsmath, amssymb}
\usepackage{fancyhdr, float, graphicx}
\usepackage[utf8]{inputenc} % Required for inputting international characters
\usepackage[T1]{fontenc} % Output font encoding for international characters
\usepackage{fouriernc} % Use the New Century Schoolbook font
\usepackage[nottoc, notlot, notlof]{tocbibind}
\usepackage{url}

% Header and Footer
\pagestyle{fancy}
\fancyhead{}
\fancyfoot{}
\fancyhead[L]{\textit{\Large{Operating Systems}}}
%\fancyhead[R]{\textit{something}}
\fancyfoot[C]{\thepage}
\renewcommand{\footrulewidth}{1pt}


\usepackage{listings}
\usepackage{xcolor}

\definecolor{codegreen}{rgb}{0,0.6,0}
\definecolor{codegray}{rgb}{0.5,0.5,0.5}
\definecolor{codepurple}{rgb}{0.58,0,0.82}
\definecolor{backcolour}{rgb}{0.95,0.95,0.92}

\lstdefinestyle{mystyle}{
    backgroundcolor=\color{backcolour},   
    commentstyle=\color{codegreen},
    keywordstyle=\color{magenta},
    numberstyle=\tiny\color{codegray},
    stringstyle=\color{codepurple},
    basicstyle=\ttfamily\footnotesize,
    breakatwhitespace=false,         
    breaklines=true,                 
    captionpos=b,                    
    keepspaces=true,                 
    numbers=left,                    
    numbersep=5pt,                  
    showspaces=false,                
    showstringspaces=false,
    showtabs=false,                  
    tabsize=2
}

\lstset{style=mystyle}




% Other Doc Editing
% \parindent 0ex
%\renewcommand{\baselinestretch}{1.5}

\begin{document}

\begin{titlepage}
	\centering

	%---------------------------NAMES-------------------------------

	\huge\textsc{
		MIT World Peace University
	}\\

	\vspace{0.75\baselineskip} % space after Uni Name

	\LARGE{
		Computer Networks\\
		Second Year B.Tech Semister 3\\
		Academic Year 2022-23
	}

	\vfill % space after Sub Name

	%--------------------------TITLE-------------------------------

	\rule{\textwidth}{1.6pt}\vspace*{-\baselineskip}\vspace*{2pt}
	\rule{\textwidth}{0.6pt}
	\vspace{0.75\baselineskip} % Whitespace above the title



	\huge{\textsc{
			Operating Systems
		}} \\



	\vspace{0.5\baselineskip} % Whitespace below the title
	\rule{\textwidth}{0.6pt}\vspace*{-\baselineskip}\vspace*{2.8pt}
	\rule{\textwidth}{1.6pt}

	\vspace{1\baselineskip} % Whitespace after the title block

	%--------------------------SUBTITLE --------------------------	

	\LARGE\textsc{
		Notes from Tananbaum and Classes
	} % Subtitle or further description
	\vfill

	%--------------------------AUTHOR-------------------------------

	Prepared By
	\vspace{0.5\baselineskip} % Whitespace before the editors

	\Large{
		P34. Krishnaraj Thadesar\\
		\vspace{1cm}
		Batch A2
	}


	\vspace{0.5\baselineskip} % Whitespace below the editor list
	\today

\end{titlepage}

\clearpage
\tableofcontents
\clearpage

\section{Concurrency Control - Waht to learn }

\paragraph Process Synchronization: Principles of Concurrency, Requirements for Mutual Exclusion: Hardware Support, OS Support

\paragraph Classical Synchronization Problems: Readers Writers Problem, Producer and Consumer Problem

\paragraph Principles of Deadlock, Deadlock modelling, prevention, avoiance and stuff.

\textbf{When 2 processes are running at the same time, when they are interdependent on each other through inputs and outputs, then you would call that concurrency. We arent doing multiprocessing by choice, here we just gotta do it at the same time somehow. }

\section{Design Issues in concurrency}
\begin{enumerate}
	\item Communication among processes
	\item Sharing and Competition for resources
	\item Synchronization of activities of multiple processes
	\item Allocation of processor time to processes.
\end{enumerate}

\section{Contexts of Concurrency}
\begin{enumerate}
	\item Multiple Aplications like Multiprogramming(diff programs on a single processor) and Multiprocessing(on diff processors)
	\item Strucctured Applications: Some applictions can be effectively programmed as a set of concurrent proesses. (Principles of modular design and structured programming)
	\item OS Structure: OS often implemented as a set of processes or threads.
\end{enumerate}

\section{Key terms related to concurrency}

\begin{enumerate}
	\item Atomic Operation: A sequence of one or more statements that appear to be inivisible that is no other process can see an intermediate state or interrupt the operations
	\item Critical Section: A section of code within a process that requires access to shared resources and that must not be executed while another process is in corresponding section of code.
	\item Deadlock: A situation in which two or more processes are unable to proceed because each is waiting for one of the others to do something.
	\item Mutual Exclusion: The requirement that when one process is in critical section that access shared resources, no other processes may be in critical section that accesses any of those shared resources.
	\item Race Condition: A situation in which multiple threads/Processes read and write a shared data item and final result depends on relative timing of their execution.
	\item Starvation: A situation in which a runnable processes is overlooked indefinitely by the scheduler, although it is able to proceed, it is never chosen.
\end{enumerate}

\subsection{Race Condition}
A race Condition occurs when multiple competing processes or threads read and write data items so that final result depends on the order of eecution of instructions in multiple processes. so 2 processes p1 and p2, say they share global variable 'a'. P1 updates a to 1, and p updates it to 2. Thus two tasks are in a race to write variable 'a'. Loser of race is the one that determines the value of 'a'.

\section{Difficulties due to concurrency}
\begin{enumerate}
	\item Sharing of global resources: Eg. Two processes both make use of global variable and both perfrom read and write on that variable, in which read and write are done is critical.
	\item Management of resources optimally. Eg. Process has gained the ownership of IO devices but is suspended before using it, thus locking IO device and preventing its use by other processes.
	\item Error locating in Program: Pesults are not deterministic are reproducible.
\end{enumerate}

\subsection{Example}

\begin{verbatim}
    void echo()
    {
        chin = getchar();
        chout = chin;
        putchar(chout);
    }
\end{verbatim}

\begin{enumerate}
	\item Uniprocessor multiprogramming, single user environment
	\item Many applications can call this procedure repeatedly to accept user ip and display on screen.
	\item User can jump from one applciation to other.
	\item Each application needs or uses procedure echo.
	\item Echo is made shared procedure and loaded into a portion of memory global to all applications
	\item Single copy of echo procedure is used, saving space.
	\item When echo procedure is invoked by some process, an dif fthe process gets suspended for any reason beffore completing it, then no other process can invoke echo till prcess that was suspended gets resumed, and completes echo. Thus other processes are not allowed to access it.
	\item It would make more sense if you visualize this in terms of a multi user operating system. So if many people are using a server kinda thing at the same time.
\end{enumerate}

\subsection{Example}
Say you have global variables that hvae values b = 1, c = 3
If P1 executes b = b + c \\
P2 does c = c + b\\

Now if P1 executes first, you would get b = 3, c = 5;\\
If P2 was to Execute first, then b = 4, c = 3\\

This is a problem.


\subsection{Process Interaction}

\begin{itemize}
	\item Processes that are unaware of each Other (Competition)
	      \begin{enumerate}
		      \item Here Multiprogramming of Multiple indepent Processes
		      \item OS Needs to know about competition for resources such as printer, disk, file etc.
		      \item Potential Problems: Mutual Exclusion, Deadlock, starvation
	      \end{enumerate}
	\item Processes that are indirectly aware of each Other (Cooperation by sharing)
	      \begin{enumerate}
		      \item Shared Access to some object such as shared variable.
		      \item Cooperation by sharing.
		      \item Potential Probelm: Mutual Exclusion, Deadlock, starvation, data coherence etc.
	      \end{enumerate}
	\item Processes that are directly aware of each other (Cooperation by communication)
	      \begin{enumerate}
		      \item Cooperation by communicatio, communication primitives are available.
		      \item Potential control problems: Deadlock and starvation.
		      \item Mutual exclusion not a probelm, as both are aware of each other.
		      \item DeadLock is possible, and so is starvation.
	      \end{enumerate}
\end{itemize}


\subsection{Three Control problems}
\begin{itemize}
	\item The need for mutual Exclusion: Two or more processes require access to single non sharable resources such as printer, say. Such a resource is called a critical resource, and the portion of code using it is called as critical section of the program. Mutual Exclusion is when you allow only process to work when its in its critical section.
	\item A common problem would be deadlock, where say P1 is using the printer, and P2 is using a file, Now P1 has to wait for P2 for using the file, and P2 has to wait for P1 to use the printer.
	\item Starvation would be like how we schedule the processes, and so periodic access would be given. While one process is using some resource, others have to patiently wait, and so they are starving.
\end{itemize}

\textbf{Remember to Examples a whole lot while answering questions of OS}

\subsubsection{Cooperation among processes by sharing}
\begin{itemize}
	\item Processes that interact with other processes without begin explicitly aware of them, access to shared variables, or files or databases is what is being talked about here.
	\item Processes may use and update shared data without reference to other processes but know that other processes may have access to same data.
	\item Processes must cooperate to ensure that the data that they share are properly managed.
	\item control problems of mutual exclusion, deadlock and starvation are again present.
	\item Data Items are accessed in 2 Modes: Reading and Writing.
	\item Writing operations must be mutually exclusive.
\end{itemize}


\subsubsection{Requirements for Mutual Exclusion}
\begin{itemize}
	\item Any facility or capability providing support for mutual exclusion should meet following requirements:
	\item Mutual Exclusion must be \textbf{enforced}. Only one process at a time in the CS, among all processes that have CS for same resource or shared object.
	\item A process that \textbf{halts in its non CS} must do without interfering with other processes.
	\item A Process requiring access to CS must not be \textbf{denied or delayed } indefinitely. So no deadlock or starvation.
	\item When no process is in its CS, any process that requests an entry to its CS, ust be permitted to enter without delay.
	\item No Asusmptions are made about the relative speed of the process.
\end{itemize}

\subsubsection{Approaches to satisfy the requirements of Mutual Exclusion}
\begin{enumerate}
	\item Hardware Approaches
	\item Support from the OS or programming language itself.
	\item Software Approach (No support from OS or programming language) You have to write it yourself.
\end{enumerate}

\subsection{OS or programming language support}
\begin{enumerate}
	\item Semaphore : It is an integer variable that sovles critical section problem, after initialization. Its value can only be accessed by 2 atomic instructions.
	      \begin{enumerate}
		      \item wait(s)
		            \begin{lstlisting}[language = C]
					wait(s) {
						while(s<= 0) {
							s --;
						}
					}
					\end{lstlisting}

		      \item signal()
		      \item \textit{Internal Working}: Consider if P1, P2, P3, P4 etc are processes that want to go to the critical section.
		            \begin{lstlisting}[language = C++]
				do{
					wait(s);
					// goes in critical section. 
					signal(s);
					// remainder section;
				} while(true);
			  \end{lstlisting}
	      \end{enumerate}
	      Explicitly you cant change the value of semaphore.
	\item There are 2 types, Binary and Counting Semaphore.
	      \begin{itemize}
		      \item Mutex: Mutual exclusion
		      \item Progress
		      \item Bounded Wait
	      \end{itemize}
	\item Mutex
	      \begin{itemize}
		      \item
	      \end{itemize}
	\item Monitors
\end{enumerate}

\subsubsection{Problem}
Suppose we want to synchronize two concurrent processes P1 and P2 using binary semaphores s and t. The code for the processes P2 and P2 is shown below.

\begin{verbatim}
	Process P1: 
	while(1)
	{
		w: ____
		print '0'
		print '0'
		x: ____
	}
\end{verbatim}

\begin{verbatim}
	while(1)
	{
		y = ___
		print(1)
		print(1)
		z = ___
	}
\end{verbatim}

The required output is "001100110010011" Processes are sunchronized using P and V operations on semaphores s and t. Choose the correct options from the following at points w x y and z. While of the following option is correct?
\begin{enumerate}
	\item P(s) at w, V(s) at x, P(t) at z, s and t initially 0
	\item P(s) at w, V(t) at x, P(t) at y, V(s) at z, s initially 1 and t 0
	\item P(s) at w, V(s) at x, P(t) at z, s and t initially 0
	\item P(s) at w, V(s) at x, P(t) at z, s 1 and t 0

\end{enumerate}

\section{Producer Consumer Problem}
\begin{itemize}
	\item it is one fo the classic problems of Synchronization
	\item Producer produces an item and adds to a buffer of limited size ( bounded buffer)
	\item Consuder takes out an item from buffer and consumes it. 
	\item Buffer is a shared resource and must be used in a mutual exclusion manner by processes. 
	\item Producers to be prevented from adding into a full buffer. 
	\item Consumers to be stopped from taking out an item from an empty buffer. 
	\item To solve this you have to use the semaphores. And wait and signal will be used. 
	\item A Cooperative Process: The Processes which are sharing the resources like code, memory, data item, and stuff like that. 
\end{itemize}

\begin{lstlisting}[language=C]
	// Here count is used to count number of items present in the buffer. 
	// Assume that the buffer has like 8 items. 
	// initially obviously the value of count would be 0;

	int count = 0;
	int in = 0; // it is the position where the producer is going to write the newly generated data item. 
	void Producer(void)
	{
		int itemp;
		while(true)
		{
			Produce_item(itemp);
			while(count == n)
			{
				buffer[in] = itemp;
				in = in + (1 mod n); // to 
				count = count + 1; // this is then again split into like 3 other commands in assembly. 
				// The micro instructions for this line are (in assembly)
				// load Rp, m[count]
				// incr Rp
				// load m[count] Rp
			}
		}
	}
	void consumer()
	{
		int itemc = 0;
		int out = 0;
		while(true)
		{
			while(count == 0)
			{
				itemc = buffer[out]
				out = out + (1 mod n)
				count = count - 1;
				Consume_item(itemc)
			}
		}
	}
\end{lstlisting}

\subsection{Solution to producer cnsumer problem using semaphore}
\begin{itemize}
	\item With a Bounded Buffer, The bounded buffer producer problem assumes that there is a fixed buffer size, in this case the consumer must wait if the buffer is empty and the producer must wait if the buffer is full.
	\item \begin{lstlisting}[language = C]
	\\ initialization

	char item; // could be any data type. 
	char buffer[n];
	semaphore full = 0; // counting semaphore for full slots
	semaphore empty = 1; // counting for empty slot
	semaphore mutex = 1; // binary sema for mutual exclusion of buffer. 
	char nextp, nextc;
	
	\end{lstlisting}
	
	\item \begin{lstlisting}
		// Producer
		do
		{
			produce and item in nextp
			wait(empty); // wait until empty > 0 and then decrement 'empty'
			wait(mutex); // acquire lock
			add nextp to bufer // add data to buffer
			signal(mutex); // release lock
			signal(full); // increment full
		} while(true)


		// Consumer
		do{
			wait(full);
			wait(mutex);
			remove an item from buffer to nextc // remove data from buffer
			signal(mutex);
			signal(empty);
			consume the item in nextc
		} while(true);
	\end{lstlisting}
\end{itemize}


\subsection{Reader and Writer Problem}

\begin{itemize}
	\item If you are running concurrently, Reader and Writer together would be a problem
	\item Writer and Reader, this would also be a problem
	\item Writer and Writer, this would be a problem
	\item Reader and Reader, this wont be a problem
\end{itemize}

\begin{lstlisting}
	semaphore db = 1;
	semaphore mutex = 1

	void reader()
	{
		semaphore rc = 0; // reader count
		while(true)
		{
			down(mutex); // this is like wait, coz you are decrementing anyway
			rc = rc + 1
			if(rc == 1)
			{
				down(db);
			}
			up(mutex);
			
			// read the data 
			
			// Exit section
			rc = rc - 1;
			if(rc == 0)
			{
				up(db)
			}
			up(mutex);
		}
	}

	void writer()
	{
		while(true)
		{
			down(mutex);
			// do something or write
			up(db);
		}
	}


\end{lstlisting}

\section{Inter Process Communication}
There are several mechanisms for inter process communication (IPC)
\begin{itemize}
	\item Signals
	\item FIFOS that are called pipes. 
	\item pipes
	\item Sockets
	\item Message passing
	\item Shared memory
	\item Semaphores. 
\end{itemize}

\subsection{Pipe}
A Pipe is a communication medium between two or more related ro interrelated processes. It can be either within one process ro a communciation betwen the child and the paretn processes.
UNIX deals with pipes the same way it deals with files. A process can send data down a pipe using a write system call and another process that can receive the data by using read at the other end.  

\begin{itemize}
	\item Within programs a pipe is created using a system call named pipie.
	\item This system call would create a pipe for one way communication. 
	\item This call would return zero on success and -1 on case of failure. 
	\item If successful this cal returns two file descriptors. 
	\begin{lstlisting}[language = C]
		#include<unistd.h>
		int pipe(int filedes[2]);
		// filedes is the file handler. 
	\end{lstlisting}
	\item filesdes is a two integer array that that will hold the file descriptors that will indentify the pipe if successful. 
	\item filedes[0] will be open for reading from the pipe. 
	\item filedes[1] will be open for writing down it. 
	\item piep can fail returns -1 if it cannot obtain the file descriptors (exceeds user limit or kernel limit)
	\item console > parent > write > read by child > console > display > parent.
\end{itemize}

\section{Monitor}
\begin{itemize}
	\item Monitors are a synchronizatiion construct. 
	\item Monitors contain data variables and procedures. 
	\item Data variables cannot be directly accessed by a process. 
	\item Monitors allow only a single process to access the variabels at a time. 
\end{itemize}

\section{Deadlock}
As mentioned before, its whne a finite number of resources are to be distributed among a number of competing processes, then this happens. 
A set of blocked processes each holding a resource and waiting to acquire a resource held by another process in the set. 

\subsection{Deadlock Characterization}
\textbf{Deadlock would happen if some of these things happen at the same time. }
\begin{itemize}
	\item Mutual Exclusion: Resources must be held in a nonsharable mode. 
	\item Hold and Wait: A process holding at least one resource is waiting to acquire additional resources held by other processes. 
	\item No Preemption: A resource can be released only voluntarily by the process holding it. after that the process has completed its task. 
	\item Circular wait: There exists a set of waiting processes such that P0 is waiting for a resoruce that is held by P1. Who is then again waiting for a resource that is hed by P2, and so on, until the last process being dependent on P0 completing its task, which means no one is going to do anything. 
\end{itemize}

\subsection{Resource Allocation graph}
\begin{itemize}
	\item If there isnt a cycle, then there there is no deadlock. 
	\item If graph has a cycle, then if only one instance per resource type, then deadlockk. 
	\item If several instances of the resource, then chance of deadlock. 
\end{itemize}

\subsection{Method of Handling a Deadlock. }
Ensure that the system will never enter into deadlock. 
\begin{itemize}
	\item You can either prevent it, naturally 
	\item or you can avoid it, which i dont even get like arent they the same thing?
	\item By ensuring that atleast one of the conditions cannot hold, we can prevent deadlock. 
	\item Like say \textbf{Mutual Exclusion}, Things like printers arent sharable by nature. so we better not allow a 2 processes to use the printer at the same time. 
	\item Holding and waiting is another way to avoid deadlock. Require the process to request what resoruce it needs, and allocate it all the resources before it begins execution. If there is, Low Resource Unitlization, so starvation is also possible.
	\item Another problem would be \textbf{no preemption}: so if a process that is holding some resoruce and requests another resrouce, that cannot be immediately allocated to it, then all resoruce currently being held are released. 
	\item Preempted resourcees are added to the list of resrouces for which the process is waiting. 
	\item Process will be something 
	\item \textbf{Circular Weight}:  Impose a total ordering of all resource tyes and require that each process requests resources in an increasing order of enumeration. 
\end{itemize}

Allow the system to enter a deadlock state and then recover. Ignore the problem and pretend that deadlock never occured in the system. Used by most Operating systems. 

\end{document}