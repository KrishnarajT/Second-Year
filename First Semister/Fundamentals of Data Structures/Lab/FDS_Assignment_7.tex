\documentclass[11pt]{article}

% Preamble

\usepackage[margin=1in]{geometry}
\usepackage{amsfonts, amsmath, amssymb}
\usepackage{fancyhdr, float, graphicx}
\usepackage[utf8]{inputenc} % Required for inputting international characters
\usepackage[T1]{fontenc} % Output font encoding for international characters
\usepackage{fouriernc} % Use the New Century Schoolbook font
\usepackage[nottoc, notlot, notlof]{tocbibind}
\usepackage{listings}
\usepackage{xcolor}

\definecolor{codegreen}{rgb}{0,0.6,0}
\definecolor{codegray}{rgb}{0.5,0.5,0.5}
\definecolor{codepurple}{rgb}{0.58,0,0.82}
\definecolor{backcolour}{rgb}{0.95,0.95,0.92}

\lstdefinestyle{mystyle}{
    backgroundcolor=\color{backcolour},   
    commentstyle=\color{codegreen},
    keywordstyle=\color{magenta},
    numberstyle=\tiny\color{codegray},
    stringstyle=\color{codepurple},
    basicstyle=\ttfamily\footnotesize,
    breakatwhitespace=false,         
    breaklines=true,                 
    captionpos=b,                    
    keepspaces=true,                 
    numbers=left,                    
    numbersep=5pt,                  
    showspaces=false,                
    showstringspaces=false,
    showtabs=false,                  
    tabsize=2
}

\lstset{style=mystyle}

% Header and Footer
\pagestyle{fancy}
\fancyhead{}
\fancyfoot{}
\fancyhead[L]{\textit{\Large{FDS Assignment 7}}}

\fancyfoot[C]{\thepage}
\renewcommand{\footrulewidth}{1pt}

\begin{document}

\begin{titlepage}
	\centering

	%---------------------------NAMES-------------------------------

	\huge\textsc{
		MIT World Peace University
	}\\

	\vspace{0.75\baselineskip} % space after Uni Name

	\LARGE{
		Fundamental Data Structures\\
		Second Year B. Tech, Semester 1
	}

	\vfill % space after Sub Name

	%--------------------------TITLE-------------------------------

	\rule{\textwidth}{1.6pt}\vspace*{-\baselineskip}\vspace*{2pt}
	\rule{\textwidth}{0.6pt}
	\vspace{0.75\baselineskip} % Whitespace above the title

	\huge{\textsc{
			Expression conversion using stack
		}} \\



	\vspace{0.5\baselineskip} % Whitespace below the title
	\rule{\textwidth}{0.6pt}\vspace*{-\baselineskip}\vspace*{2.8pt}
	\rule{\textwidth}{1.6pt}

	\vspace{1\baselineskip} % Whitespace after the title block

	%--------------------------SUBTITLE --------------------------	

	\LARGE\textsc{
		Practical Report\\
		Assignment 7
	} % Subtitle or further description
	\vfill

	%--------------------------AUTHOR-------------------------------

	Prepared By
	\vspace{0.5\baselineskip} % Whitespace before the editors

	\Large{
		Krishnaraj Thadesar \\
		Cyber Security and Forensics\\
		Batch A1, PA 20
	}


	\vspace{0.5\baselineskip} % Whitespace below the editor list
	\today

\end{titlepage}


\tableofcontents
\thispagestyle{empty}
\clearpage
\setcounter{page}{1}

\section{Aim}
Implement stack as an ADT and apply it for different expression conversions (infix to postfix or
infix to prefix (Any one), prefix to postfix or prefix to infix, postfix to infix or postfix to prefix
(Any one)).

\section{Objectives}
\begin{enumerate}
	\item To study Stack and its operations
	\item To study the importance of expression conversions
\end{enumerate}

\section{Problem Statements}
\textit{Department of Computer Engineering has student's node named 'Pinnacle Node'. Students of
	second, third and final year of department can be granted membership on request. Similarly, one
	may cancel the membership of node. First node is reserved for president of node and last node is
	reserved for the secretary of the node. Write C program to maintain node members information
	using singly linked list. Store student PRN and Name. Write functions to:}
\begin{enumerate}
	\item \textit{ Add members as well as president or even secretary.}
	\item \textit{ Compute total number of members of node }
	\item \textit{ Display members}
	\item \textit{ sorting of two linked list}
	\item \textit{ merging of two linked list}
	\item \textit{ Reversing using three pointers}
	\item \textit{ Add and delete the}
\end{enumerate}
\section{Theory}
\subsection{Stack}
A stack is a linear data structure which follows a particular order in which the operations are
performed. The order may be LIFO (Last In First Out) or FILO (First In Last Out). Mainly the
following three basic operations are performed in the stack:
\begin{enumerate}
	\item \textit{Push:} Adds an item in the stack. If the stack is full, then it is said to be an
	      Overflow condition.
	\item \textit{Pop:} Removes an item from the stack. The items are popped in the reversed order
	      in which they are pushed. If the stack is empty, then it is said to be an Underflow
	      condition.
	\item \textit{Peek or Top:} Returns top element of stack.
	\item \textit{isEmpty:} Returns true if stack is empty, else false.
\end{enumerate}


\subsection{Expression Conversion}
Expression conversion is the process of converting an expression from one form to another. The
following are the different forms of expression conversion:
\begin{enumerate}
	\item \textit{Infix to Postfix:} Infix expression is converted to postfix expression.
	\item \textit{Infix to Prefix:} Infix expression is converted to prefix expression.
	\item \textit{Postfix to Infix:} Postfix expression is converted to infix expression.
	\item \textit{Postfix to Prefix:} Postfix expression is converted to prefix expression.
	\item \textit{Prefix to Infix:} Prefix expression is converted to infix expression.
	\item \textit{Prefix to Postfix:} Prefix expression is converted to postfix expression.
\end{enumerate}

\subsection{Infix to Postfix}
The algorithm for conversion of infix expression to postfix expression is as follows:
\begin{enumerate}
	\item Scan the infix expression from left to right.
	\item If the scanned character is an operand, output it.
	\item Else,
	      \begin{enumerate}
		      \item If the precedence of the scanned operator is greater than the precedence of the
		            operator in the stack(or the stack is empty or the stack contains a '(' ), push it.
		      \item Else, Pop all the operators from the stack which are greater than or equal to in
		            precedence than that of the scanned operator. After doing that Push the scanned
		            operator to the stack. (If you encounter parenthesis while popping then stop there
		            and push the scanned operator in the stack.)
	      \end{enumerate}
	\item If the scanned character is an '(', push it to the stack.
	\item If the scanned character is an ')', pop the stack and and output it until a '(' is
	      encountered, and discard both the parenthesis.
	\item Repeat steps 2-6 until infix expression is scanned.
	\item Print the output
	\item Pop and output from the stack until it is not empty.
\end{enumerate}

\subsection{Postfix to Prefix}
The algorithm for conversion of postfix expression to prefix expression is as follows:
\begin{enumerate}
	\item Read the postfix expression from right to left.
	\item If the symbol is an operand, then push it onto the Stack.
	\item If the symbol is an operator, then pop two operands from the Stack
	\item Create a string by concatenating the two operands and the operator before them.
	\item string = operator + operand1 + operand2
	\item And push the resultant string back to Stack
	\item Repeat the above steps until end of Prefix expression.
\end{enumerate}

\section{Platform}
\textbf{Operating System}: Arch Linux x86-64 \\
\textbf{IDEs or Text Editors Used}: Visual Studio Code\\
\textbf{Compilers} : gcc on linux for C\\

\section{Input}

\begin{itemize}
	\item Elements of the Stack
\end{itemize}

\section{Output}
\begin{itemize}
	\item Menu to display all the operations you can perform on the Stack.
	\item Display of All the elements of the Stack. 
\end{itemize}

\section{Test Conditions}
\begin{enumerate}
	\item Input at least 5 records.
	\item Inserting an Element at All Positions
	\item Delete an Element from All positions
\end{enumerate}

\section{Code}
\subsection{Pseudo Code}
\subsubsection{Pseudo Code for checking if the stack is full}
\begin{lstlisting}[language=C]
	isFull()
		if (top == MAX_SIZE - 1)
			return 1;
		else
			return 0;
\end{lstlisting}\subsubsection{Pseudo Code for checking if the stack is empty}

\begin{lstlisting}[language=C]
	isEmpty()
		if (top == -1)
			return 1;
		else
			return 0;
\end{lstlisting}\subsubsection{Pseudo Code for Pushing onto stack}

\begin{lstlisting}[language=C]
	push(char item)
		if (!isFull())
			top++;
			stack[top] = item;
		else
			printf("\nSTACK OVERFLOW!\n");
\end{lstlisting}\subsubsection{Pseudo Code for popping from the stack}

\begin{lstlisting}[language=C]
	pop()
		if (isEmpty())
			printf("Stack is Empty \n\n STACK UNDERFLOW!!");
			return 0;
		else
			printf("Removed this thing %c\n", stack[top]);
			top--;
			return stack[top + 1];
	
\end{lstlisting}\subsubsection{Pseudo Code for checking incoming character precedence and incoming sign precedence}

\begin{lstlisting}[language=C]

	icp(char ch) // incoming char precedence
		if (ch == '+' || ch == '-')
			return 1;
		if (ch == '*' || ch == '/')
			return 2;
		if (ch == '^')
			return 4;
		if (ch == '(')
			return 5;
		else
			return 0;
	isp(char ch) // incoming sign precedence
		if (ch == '+' || ch == '-')
			return 1;
		if (ch == '*' || ch == '/')
			return 2;
		if (ch == '^')
			return 3;
		else
			return 0;
\end{lstlisting}\subsubsection{Pseudo Code infix to postfix}

\begin{lstlisting}[language=C]
	
	infix_to_postfix(char inexp[10])
		int postexp[10];
		int k = 0, i = 0;
		char tkn = inexp[i];
		while (tkn != '\0')
			if (tkn >= 97 && tkn <= 122)
				postexp[k] = inexp[i];
				k++;
			else
				if (tkn == '(')
					push('(');
				else
					if (tkn == ')')
						while ((tkn = pop()) != '(')
							postexp[k] = tkn;
							k++;
					else
						while (!isEmpty() && isp(stack[top] >= icp(tkn)))
							postexp[k] = pop();
							k++;
						push(tkn);
			i++;
			tkn = inexp[i];
		while (!isEmpty())
			postexp[k] = pop();
			k++;
		postexp[k] = '\0';
		for (int i = 0; i < k; i++)
			printf("%c", postexp[i]);
\end{lstlisting}\subsubsection{Pseudo Code for postfix to infix}

\begin{lstlisting}[language=C]
	
	postfix_to_infix(char post[MAX_SIZE])
		for (int i = 0; post[i] != '\0'; i++)
			if (post[i] >= 97 && post[i] <= 127)
				temp[0] = post[i];
				temp[1] = '\0';
				push_str(temp);
				// temp[0] = '\0';
			else
				temp = pop_str();
				temp1 = pop_str();
				temp2[0] = post[i];
				temp2[1] = '\0';
				strcpy(inf, "(");
				strcat(inf, temp1);
				strcat(inf, temp2);
				strcat(inf, temp);
				strcat(inf, ")");
				push_str(inf);
		inf = pop_str();
		printf("\nThe infix expression is: ");
		printf("\n%s", inf);
\end{lstlisting}
\subsection{C Implementation of Problem Statement}

\lstinputlisting[language=C, caption=Main.Cpp]{../Programs/[Krishnaraj]_[FDS]_Assignment_7.c}

\subsection{Input and Output}
\lstinputlisting[caption=Output]{../Programs/[Krishnaraj]_[FDS]_Assignment_7_output.txt}

\section{Time Complexity}
\begin{itemize}
	\item \textbf{Insertion}: O(1)
	\item \textbf{Deletion}: O(1)
	\item \textbf{Searching}: O(n)
\end{itemize}

\section{Conclusion}
\begin{itemize}
	\item \textbf{Pros}: Easy to implement, Easy to understand, Easy to use.
	\item \textbf{Cons}: No random access, No reverse access, No search.
\end{itemize}


\section{FAQs}
\begin{enumerate}
	\item \textbf{What is the advantage of prefix and postfix over infix expression?}\\

	      The advantage of prefix and postfix over infix expression is that there is no need of brackets in prefix and postfix expression. So there is no problem of precedence.
	\item \textbf{Explain how postfix/prefix expression is evaluated.}\\

	      Postfix expression is evaluated by scanning the expression from left to right. If the scanned character is an operand, push it onto the stack. If the scanned character is an operator, pop two operands from the stack and apply the operator on them. Push the result back onto the stack. Repeat the steps until the end of the expression. The result obtained at the end of the expression is the final result.

	\item \textbf{What is ISP and ICP?}\\
	      In the process of creating machine code from source code, compilers translate infix expressions to postfix expressions.
	      Uses the 2 notions of precedence:
	      \begin{itemize}
		      \item incoming sign precedence isp()
		      \item incoming character Presedence icp().
	      \end{itemize}
	      ISP is the precedence of the operator in the stack and ICP is the precedence of the operator in the scanned expression.

	\item \textbf{Give various applications of stack}\\

	      \begin{enumerate}
		      \item \textbf{Balancing of symbols:} In this application, we check whether the given expression has balanced symbols or not. For example, in the expression (a+b)*(c-d), symbols are balanced. But in the expression (a+b*(c-d), symbols are not balanced.
		      \item \textbf{Infix to Postfix Conversion:} In this application, we convert the given infix expression to postfix expression. For example, the infix expression a+b*c-d/e is converted to abc*+de/-. The postfix expression is evaluated easily as compared to infix expression.
		      \item \textbf{Evaluation of Postfix Expression:} In this application, we evaluate the given postfix expression. For example, the postfix expression abc*+de/- is evaluated as (a+b*c)-(d/e).
		      \item \textbf{Redo-undo:} In this application, we can undo the last performed operation and redo it again.
		      \item \textbf{Forward and backward feature in web browsers:} In this application, we can go to the previous web page and next web page.
		      \item \textbf{Infix to Prefix Conversion:} In this application, we convert the given infix expression to prefix expression. For example, the infix expression a+b*c-d/e is converted to -+a*bc/de. The prefix expression is evaluated easily as compared to infix expression.
		      \item \textbf{Evaluation of Prefix Expression:} In this application, we evaluate the given prefix expression. For example, the prefix expression -+a*bc/de is evaluated as (a+b*c)-(d/e).
		      \item \textbf{Parenthesis Matching:} In this application, we check whether the given expression has balanced parenthesis or not. For example, in the expression (a+b)*(c-d), parenthesis are balanced. But in the expression (a+b*(c-d), parenthesis are not balanced.
		      \item \textbf{Tower of Hanoi:} In this application, we solve the Tower of Hanoi problem using stack.
		      \item \textbf{Expression Conversion:} In this application, we convert the given expression from one form to another form. For example, we can convert the given infix expression to postfix expression or prefix expression.
	      \end{enumerate}
	\item \textbf{Why is stack used in expression conversion?}\\

	      One of the applications of Stack is in the conversion of arithmetic expressions in high-level programming languages into machine readable form. As our computer system can only understand and work on a binary language, it assumes that an arithmetic operation can take place in two operands only e.g., A+B, C*D,D/A etc. But in our usual form an arithmetic expression may consist of more than one operator and two operands e.g. (A+B)*C(D/(J+D)).\\

	      These complex arithmetic operations can be converted into polish notation using stacks which then can be executed in two operands and an operator form.

	\item \textbf{Give stack full and stack empty conditions for stack.}\\

	      if Top == size - 1, then stack is full. If top == -1 then stack is empty.


\end{enumerate}
\end{document}

