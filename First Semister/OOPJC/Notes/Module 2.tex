% This is a basic Math Paper

\documentclass[11pt]{article}

% Preamble

\usepackage[margin=1in]{geometry}
\usepackage{amsfonts, amsmath, amssymb}
\usepackage{fancyhdr, float, graphicx}
\usepackage[utf8]{inputenc} % Required for inputting international characters
\usepackage[T1]{fontenc} % Output font encoding for international characters
\usepackage{fouriernc} % Use the New Century Schoolbook font
\usepackage[nottoc, notlot, notlof]{tocbibind}
\usepackage{url}

% Header and Footer
\pagestyle{fancy}
\fancyhead{}
\fancyfoot{}
\fancyhead[L]{\textit{\Large{Object Oriented Programming}}}
%\fancyhead[R]{\textit{something}}
\fancyfoot[C]{\thepage}
\renewcommand{\footrulewidth}{1pt}

\usepackage{listings}
\usepackage{xcolor}

\definecolor{codegreen}{rgb}{0,0.6,0}
\definecolor{codegray}{rgb}{0.5,0.5,0.5}
\definecolor{codepurple}{rgb}{0.58,0,0.82}
\definecolor{backcolour}{rgb}{0.95,0.95,0.92}

\lstdefinestyle{mystyle}{
    backgroundcolor=\color{backcolour},   
    commentstyle=\color{codegreen},
    keywordstyle=\color{magenta},
    numberstyle=\tiny\color{codegray},
    stringstyle=\color{codepurple},
    basicstyle=\ttfamily\footnotesize,
    breakatwhitespace=false,         
    breaklines=true,                 
    captionpos=b,                    
    keepspaces=true,                 
    numbers=left,                    
    numbersep=5pt,                  
    showspaces=false,                
    showstringspaces=false,
    showtabs=false,                  
    tabsize=2
}

\lstset{style=mystyle}



% Other Doc Editing
% \parindent 0ex
%\renewcommand{\baselinestretch}{1.5}

\begin{document}

\begin{titlepage}
	\centering

	%---------------------------NAMES-------------------------------

	\huge\textsc{
		MIT World Peace University
	}\\

	\vspace{0.75\baselineskip} % space after Uni Name

	\LARGE{
		Computer Networks\\
		Second Year B.Tech Semister 3\\
		Academic Year 2022-23
	}

	\vfill % space after Sub Name

	%--------------------------TITLE-------------------------------

	\rule{\textwidth}{1.6pt}\vspace*{-\baselineskip}\vspace*{2pt}
	\rule{\textwidth}{0.6pt}
	\vspace{0.75\baselineskip} % Whitespace above the title



	\huge{\textsc{
			Module 1 - Class Notes
		}} \\



	\vspace{0.5\baselineskip} % Whitespace below the title
	\rule{\textwidth}{0.6pt}\vspace*{-\baselineskip}\vspace*{2.8pt}
	\rule{\textwidth}{1.6pt}

	\vspace{1\baselineskip} % Whitespace after the title block

	%--------------------------SUBTITLE --------------------------	

	\LARGE\textsc{
		Notes
	} % Subtitle or further description
	\vfill

	%--------------------------AUTHOR-------------------------------

	Prepared By
	\vspace{0.5\baselineskip} % Whitespace before the editors

	\Large{
		P34. Krishnaraj Thadesar\\
		\vspace{1cm}
		Batch A2
	}


	\vspace{0.5\baselineskip} % Whitespace below the editor list
	\today

\end{titlepage}

\clearpage
\tableofcontents
\clearpage


\section{Things to do}
\begin{enumerate}
	\item Types of Inheritance
	\item Virtual base Classs
	\item Polymorphism
	\item Vitual functions
\end{enumerate}

\section{Inheritance}

\begin{itemize}
	\item It is the mechanism by which one class acquires the properties of another class
	\item Provides a way to create a new class from an existing class
	\item The new class is a specialized fersion of the existing class
	\item Inheritance establishes an "is a" relationship or a parent child relationship between classes. 
	\item Allows sharing off the behavior of the parent class into its child classes
	\item child class can add new behavior or override existing behaviour from parent
	\item It allows a hierarchy of classes to be built moving from the most general to the most specific class. 
\end{itemize}

\subsection{Differnece between overloading and overiding}
\begin{itemize}
	\item Overloading is when you write the same function many times within the same class
	\item Overriding is when you do that same thing, but in sub classes. 
\end{itemize}

\subsection{Benefits of using Inheritance}
\begin{itemize}
	\item Reusablity : Reuse the methods and data of the existing class
	\item Extendability: Extend the existing class by adding new data and new methods. 
	\item Modifyability: Modify the existang class by overloading its methods with newer implementations, saves memory space, increases reliability, saves the developing process. 
\end{itemize}

\section{Class Derivation in C++}
syntax: class DerivedClassName : specification BaseClassName\\
like class child : public parent() // private by default
{};

\section{Types of Inheritance}
{
	\begin{enumerate}
		\item Single level Inheritance : You have 1 base class --> 1 Child class.
		\item Multiple Inheritance : 2 or more Base Classes --> 1 Child Class
		\item Multi-Level inheritance : 1 Base Class --> 1 Child Class --> Another Child Class and so on
		\item Heirarchical Inheritance : 1 Base Class --> 2 or more Child Classes. 
		\item Hybrid Inhertiance : Any legal combination of any of these things. 
	\end{enumerate}
}

\subsection{What Access modifiers mean when inheriting}
\begin{enumerate}
	\item If you do class child : private parent; then every private data member becomes inaccessible, coz anyway thats what should happen, then the protected data members become private, and public data members also become private. 
	\item If you do class child : protected parent; then its the same thing, except you still cant access private variables, but protected and public data members become protected
	\item Same with class child : public parent; everything remains unchanged. The objects will behave in accordance with the usual laws of objects.  
\end{enumerate}


\subsection{Constructors and Destructors in Base and Derived classes}
\begin{enumerate}
	\item Derived classes can have their own constructors and destructors
	\item When an object of a derived class is created, the base class's constructor is executed frist followed by the derived class's constructor is executed first, followed bt the derived class's constructor
	\item In case of multiple inheritances, the base classes are constructed in the order in which they appear in the declaration of the derived class. 
	\item For destructors, the order is reversed. 
\end{enumerate}

\section{Overriding Member Functions}
\begin{itemize}
\item If a base and derived class have member functions with same name, and arguements then method is said to be overridden and it is caled as "function overriding" or "method overriding". 
\item The Child class provides alternative implementation for parent class method  specific to a particular subclass type. 
\item You might need to do this if your child class has something to add to the previous definiton. You could still call it from that function. 
\item If you have multiple functions tho, you could have some ambiguity in your code, and to fix that you could the scope resolution operator.
\end{itemize}


\lstinputlisting[language=C++]{../Programs/inheritance_ambiguity.cpp}

\section{Virtual Base Class}
\begin{itemize}
\item In hybrid inheritance child class has two direct parents which themselves have a common base class. 
\item So you can prevent mumtiple copies of the base class coming into the child class by declaring the base class as virtual when its being inherited. 
\item So like imagine you have 2 base classes each inheriting the same class. Now imagine a third class that inherits from both of them. So the base, or the grandparent classes methods are copied twice. You can prevent this by declaring them as virtual base classes. 
\end{itemize}

\lstinputlisting[language=C++]{../Programs/virtual_base_classes.cpp}

\section{Inheritance in Java}
\begin{itemize}
	\item It is pretty Much similar to cpp
\end{itemize}

\begin{verbatim}
	Syntax: 
	class derived_class extends base_class Name
	{
		// methods and stuff. 
	}
\end{verbatim}


\end{document}

