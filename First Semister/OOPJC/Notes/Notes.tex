% This is a basic Math Paper

\documentclass[11pt]{article}

% Preamble

\usepackage[margin=1in]{geometry}
\usepackage{amsfonts, amsmath, amssymb}
\usepackage{fancyhdr, float, graphicx}
\usepackage[utf8]{inputenc} % Required for inputting international characters
\usepackage[T1]{fontenc} % Output font encoding for international characters
\usepackage{fouriernc} % Use the New Century Schoolbook font
\usepackage[nottoc, notlot, notlof]{tocbibind}
\usepackage{url}

% Header and Footer
\pagestyle{fancy}
\fancyhead{}
\fancyfoot{}
\fancyhead[L]{\textit{\Large{Object Oriented Programming}}}
%\fancyhead[R]{\textit{something}}
\fancyfoot[C]{\thepage}
\renewcommand{\footrulewidth}{1pt}



% Other Doc Editing
% \parindent 0ex
%\renewcommand{\baselinestretch}{1.5}

\begin{document}

\begin{titlepage}
	\centering

	%---------------------------NAMES-------------------------------

	\huge\textsc{
		MIT World Peace University
	}\\

	\vspace{0.75\baselineskip} % space after Uni Name

	\LARGE{
		Computer Networks\\
		Second Year B.Tech Semister 3\\
		Academic Year 2022-23
	}

	\vfill % space after Sub Name

	%--------------------------TITLE-------------------------------

	\rule{\textwidth}{1.6pt}\vspace*{-\baselineskip}\vspace*{2pt}
	\rule{\textwidth}{0.6pt}
	\vspace{0.75\baselineskip} % Whitespace above the title



	\huge{\textsc{
			Module 1 - Class Notes
		}} \\



	\vspace{0.5\baselineskip} % Whitespace below the title
	\rule{\textwidth}{0.6pt}\vspace*{-\baselineskip}\vspace*{2.8pt}
	\rule{\textwidth}{1.6pt}

	\vspace{1\baselineskip} % Whitespace after the title block

	%--------------------------SUBTITLE --------------------------	

	\LARGE\textsc{
		Notes
	} % Subtitle or further description
	\vfill

	%--------------------------AUTHOR-------------------------------

	Prepared By
	\vspace{0.5\baselineskip} % Whitespace before the editors

	\Large{
		P34. Krishnaraj Thadesar\\
		\vspace{1cm}
		Batch A2
	}


	\vspace{0.5\baselineskip} % Whitespace below the editor list
	\today

\end{titlepage}

\clearpage
\tableofcontents
\clearpage

\section{Procedural Oriented Progamming}
\begin{enumerate}
    \item Emphasis is on doing things. or Algorthms, and procedures, rather than the actual data. C++ however is the opposite.
    \item Large Programs are divided into smaller programs known as functions. 
    \item Most of the functions share global data. 
    \item Data moves openly around the system from function to funnction
    \item Functions transform data from one form to another. This is a problem with Security, as while transferring data, it can be poached. 
    \item Employs top down approach in program design. 
\end{enumerate}

\section{Object Oriented Programming }

\begin{enumerate}
    \item Design methodology for creating a non rigid application
    \item Works on entity alled as objects
    \item Decompose problem in to small unit of work which are accessed via objects. 
    \item Emphasis is on data rather than procedure. 
    \item Data is hidden and cannot be accessed by external function. 
    \item Objects may communicate with each other through functin. 
    \item follows bottom up approach in program design. 
\end{enumerate}


\subsection{Features of OOP}
\begin{enumerate}
    \item Object
    \item Class
    \item Data Abstraction
    \item Data Encapsulation
    \item Inheritence
    \item Polymorphism
    \item Modularity
\end{enumerate}

\subsection{Objects}

\begin{enumerate}
    \item Any real world entity which can have some characteristics or which can perform some work is called as Object
    \item The object is called as an instance that is a copy of an entity in the programming language. 
    \item They are also called instances. 
\end{enumerate}

\subsection{Class}
\begin{enumerate}
    \item A class is a plan whcih describes the object
    \item It is a blue print of how the object should be represented.
\end{enumerate}

\section{Concepts in OOP}
\subsection{Data Abstraction}
\begin{itemize}
	\item Data is hidden from parts of the code and the user
	\item Abstraction is a rather wide topic, it doesnt just end here. 
	\item Abstraction is sort of like the most vital thing for people working with user level software. 
	\item You basically need to hide the useless cumbersome stuff from the programmer, and provide nice and clean methods for them to perform something, that would have been much harder if we gave them the details. 
	\item For an example, writing data on a hard disk requires the maintainence of the speed of that disk, and where to move the head, etc. But the programmer is abstracted from this data, and only cares about creating a file. 
	\item This concept can be extended into a single code as well, which is what we are talking about here.
\end{itemize}

\subsection{Encapsulation}
\begin{itemize}
	\item It is defined as the process of enclosing one or more details from outside world through access right. 
	\item It says how much access should be given to particular details. 
	\item oth abstraction and Encapsulation works hand in hand because Abstraction says what details to be made visible and Encapsulation provides the level of access rights to that visible details. 
	\item It implements the desired level of abstraction.
	\item Things like access modifiers fall in this category.
\end{itemize}


\subsection{Inheritance}
\begin{itemize}
	\item Ability to extend the functionality from base entity to new entity belonging to same group. 
	\item this will help us to reuse the functionality.
	\item We know the whole child parent example here. Stuff like Animal Class and its subclasses, or a phone class and sub classes like Samsung, and MI. 
	\item There are 5 types of Inheritance. No one really cares about this, we just do this for segregating. 
	\begin{enumerate}
		\item Single level Inheritance : You have 1 base class --> 1 Child class.
		\item Multiple Inheritance : 2 or more Base Classes --> 1 Child Class
		\item Multi-Level inheritance : 1 Base Class --> 1 Child Class --> Another Child Class and so on
		\item Heirarchical Inheritance : 1 Base Class --> 2 or more Child Classes. 
		\item Hybrid Inhertiance : Any legal combination of any of these things. 
	\end{enumerate}
	
\end{itemize}

\subsection{Polymorphism}
\begin{itemize}
	\item It is the ability of doing the same thing but with different type of input. 
	\item \textit{Many forms of single entity}
	\item Function overloading basically, where you are calling the same part of code, in a method, but with different names of those methods, each of which might do something unique to them, in addition to their base function. 
	\item Compile time Polymorphism would be function overloading and operator Overloading. The Compiler figures this out. 
	\item Runtime polymorphism : you can point to any derived class from the object of the base class at runtime that shows the ability of runtime binding. Things like virtual functions fall into this. 
\end{itemize}

\subsection{Object Base vs. Object Oriented Languages. }
Theres not really a lot of difference here. Its just that languages like Visual basic have objects, and the concept of objects, but they dont have inhertance and dynamic Binding. C++ and Java have those features and are therefore called OOLanguages. Visual basic also has data Encapsulation, abstraction, auto initialization and clean up of objects, overloading, and is called just an Object based Language.


\subsection{Applications of OOP}
\begin{itemize}
	\item Real time systems
	\item Simulation and modeling
	\item Object oriented data bases. 
	\item Hypertext, Hyper media
	\item AI expect systems
	\item Neural networks and parallel programming. 
	\item Decision support and office automation systems
	\item CIM, CAM, CAD systems. 
\end{itemize}
\end{document}