% This is a Basic Assignment Paper but with like Code and stuff allowed in it. 

\documentclass[11pt]{article}

% Preamble

\usepackage[margin=1in]{geometry}
\usepackage{amsfonts, amsmath, amssymb}
\usepackage{fancyhdr, float, graphicx}
\usepackage[utf8]{inputenc} % Required for inputting international characters
\usepackage[T1]{fontenc} % Output font encoding for international characters
\usepackage{fouriernc} % Use the New Century Schoolbook font
\usepackage[nottoc, notlot, notlof]{tocbibind}
\usepackage{listings}
\usepackage{xcolor}

\definecolor{codegreen}{rgb}{0,0.6,0}
\definecolor{codegray}{rgb}{0.5,0.5,0.5}
\definecolor{codepurple}{rgb}{0.58,0,0.82}
\definecolor{backcolour}{rgb}{0.95,0.95,0.92}

\lstdefinestyle{mystyle}{
    backgroundcolor=\color{backcolour},   
    commentstyle=\color{codegreen},
    keywordstyle=\color{magenta},
    numberstyle=\tiny\color{codegray},
    stringstyle=\color{codepurple},
    basicstyle=\ttfamily\footnotesize,
    breakatwhitespace=false,         
    breaklines=true,                 
    captionpos=b,                    
    keepspaces=true,                 
    numbers=left,                    
    numbersep=5pt,                  
    showspaces=false,                
    showstringspaces=false,
    showtabs=false,                  
    tabsize=2
}

\lstset{style=mystyle}

% Header and Footer
\pagestyle{fancy}
\fancyhead{}
\fancyfoot{}
\fancyhead[L]{\textit{\Large{OOPJC Assignment 3}}}
%\fancyhead[R]{\textit{something}}
\fancyfoot[C]{\thepage}
\renewcommand{\footrulewidth}{1pt}



% Other Doc Editing
% \parindent 0ex
%\renewcommand{\baselinestretch}{1.5}

\begin{document}

\begin{titlepage}
	\centering

	%---------------------------NAMES-------------------------------

	\huge\textsc{
		MIT World Peace University
	}\\

	\vspace{0.75\baselineskip} % space after Uni Name

	\LARGE{
		Object Oriented Programming with Java and C++\\
		Second Year B. Tech, Semester 1
	}

	\vfill % space after Sub Name

	%--------------------------TITLE-------------------------------

	\rule{\textwidth}{1.6pt}\vspace*{-\baselineskip}\vspace*{2pt}
	\rule{\textwidth}{0.6pt}
	\vspace{0.75\baselineskip} % Whitespace above the title



	\huge{\textsc{
			Implementation of Polymorphism using C++ and JAVA
		}} \\



	\vspace{0.5\baselineskip} % Whitespace below the title
	\rule{\textwidth}{0.6pt}\vspace*{-\baselineskip}\vspace*{2.8pt}
	\rule{\textwidth}{1.6pt}

	\vspace{1\baselineskip} % Whitespace after the title block

	%--------------------------SUBTITLE --------------------------	

	\LARGE\textsc{
		Practical Report
	} % Subtitle or further description
	\vfill

	%--------------------------AUTHOR-------------------------------

	Prepared By
	\vspace{0.5\baselineskip} % Whitespace before the editors

	\Large{
		Krishnaraj Thadesar \\
		Cyber Security and Forensics\\
		Batch A2, PA 20
	}


	\vspace{0.5\baselineskip} % Whitespace below the editor list
	\today

\end{titlepage}


\tableofcontents
\thispagestyle{empty}
\clearpage


\setcounter{page}{1}

\section{Aim and Objectives}
\subsection*{Aim}
Implementation of Polymorphism using C++ and Java.

\subsection*{Objectives}
\begin{enumerate}
	\item To understand the use of pure virtual funcitons. 
	\item To understnad implantation of compile time and run time polymorphism.
	\item To learn implementation ofmethod overriding in java. 
\end{enumerate}

\section{Problem Statements}
\subsection{Problem 1 in C++}

Write a C++ program with base class Employee and three derived classes namely
\begin{itemize}
	\item SalariedEmployees
	\item CommissionEmployees
	\item HourlyEmployees
\end{itemize}

Declare calculateSalary() as a pure virtual function in base class and define it in respective
derived classes to calculate salary of an employee.
The company wants to implement an Object Oriented Application that performs its payroll
calculations polymorphically.


\subsection{Problem 2 in Java}

Define a Class \textbf{Shapes} as the Base Class that can find the area of the following : 
\begin{itemize}
	\item Circle
	\item Square
	\item Rectangle
\end{itemize}

Find the area of these shapes using construtor overloading and method overloading. 

\subsection{Probelm 3 in Java}
Create a Parent Class Hillstations with the methods location() and famousfor().
Create three subclasses by Hill Station names. 
These subclasses must extend the superclass and override its methods location() and famousfor().
It should refer to the base class object and the base class method overrides the superclass method, and the base class method is invoked at runtime. 

\section{Theory}

\subsection{Concept of Compile time Polymorphism}

\subsection{Concept of Run Time Polymorphism}

\subsection{Use of Pure Virtual Functions}

\section{Platform}
\textbf{Operating System}: Arch Linux x86-64 \\
\textbf{IDEs or Text Editors Used}: Visual Studio Code\\
\textbf{Compilers} : g++ and gcc on linux for C++, and javac, with JDK 18.0.2 for Java\\

\section{Input}

\subsection*{For C++}
\begin{enumerate}
	\item Number of Each Type of Employee
	\item Name, Age, Address City, and Salary of Each Employee
\end{enumerate}

\subsection*{For Java}
\begin{enumerate}
	\item The Side of the Square
	\item The Radius of the Circle
	\item The Length and Breadth of the Rectangle. 
\end{enumerate}

\section{Output}
\subsection*{For C++}
\begin{enumerate}
	\item General Information about Each Employee
	\item The Weekly, hourly and commisioned Salary for Respective Employees. 
\end{enumerate}

\subsection*{For Java}
\begin{enumerate}
	\item The Area of the Shapes
	\item The Location of the Hill Stations
	\item The Reason the Hill stations are Famous for. 
\end{enumerate}


\section{Code}
\subsection{C++ Implementation}

\lstinputlisting[language=c++, caption=Main.Cpp]{../Programs/cpp_implementations/Assignment_3.cpp}

\subsubsection{C++ Input and Output}
\lstinputlisting[caption=Output for Problem 1]{../Programs/cpp_implementations/Assignment_3_output.txt}

\subsection{Java Implementation of Problem 2}

\lstinputlisting[language=java, caption=Full Time Employee.java]{../Programs/java_implementations/assignment_3/Shapes.java}
\lstinputlisting[language=java, caption=Main.java]{../Programs/java_implementations/assignment_3/Problem_A.java}

\subsubsection{Java Output for Problem 2}
\lstinputlisting[caption=Output for Problem 2]{../Programs/java_implementations/assignment_3/Problem_A_output.txt}

\subsection{Java Implementation of Problem 3 using Interfaces} 

\lstinputlisting[language=java, caption=HillStation]{../Programs/java_implementations/assignment_3/HillStation.java}
\lstinputlisting[language=java, caption=Main.java]{../Programs/java_implementations/assignment_3/Problem_B.java}

\subsubsection{Java Output}
\lstinputlisting[caption=Output for Problem 3]{../Programs/java_implementations/assignment_3/Problem_B_output.txt}

\pagebreak

\section{Conclusion}
Thus, learned to use polymorphism and implemented solution of the given problem statement using C++ and Java. 

\section{FAQs}

\begin{enumerate}
	\item \textbf{Discuss the use of Virtual Functions. }
	\item \textbf{What is the difference ebtween early binding and late binding. }
	\item \textbf{Explain the use of abstract keyword in java with examples.}
	\item \textbf{State Features of abstract base classes.} 
\end{enumerate}

\end{document}