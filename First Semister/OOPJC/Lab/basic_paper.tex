% This is a Basic Assignment Paper but with like Code and stuff allowed in it. 

\documentclass[11pt]{article}

% Preamble

\usepackage[margin=1in]{geometry}
\usepackage{amsfonts, amsmath, amssymb}
\usepackage{fancyhdr, float, graphicx}
\usepackage[utf8]{inputenc} % Required for inputting international characters
\usepackage[T1]{fontenc} % Output font encoding for international characters
\usepackage{fouriernc} % Use the New Century Schoolbook font
\usepackage[nottoc, notlot, notlof]{tocbibind}
\usepackage{listings}
\usepackage{xcolor}

\definecolor{codegreen}{rgb}{0,0.6,0}
\definecolor{codegray}{rgb}{0.5,0.5,0.5}
\definecolor{codepurple}{rgb}{0.58,0,0.82}
\definecolor{backcolour}{rgb}{0.95,0.95,0.92}

\lstdefinestyle{mystyle}{
    backgroundcolor=\color{backcolour},   
    commentstyle=\color{codegreen},
    keywordstyle=\color{magenta},
    numberstyle=\tiny\color{codegray},
    stringstyle=\color{codepurple},
    basicstyle=\ttfamily\footnotesize,
    breakatwhitespace=false,         
    breaklines=true,                 
    captionpos=b,                    
    keepspaces=true,                 
    numbers=left,                    
    numbersep=5pt,                  
    showspaces=false,                
    showstringspaces=false,
    showtabs=false,                  
    tabsize=2
}

\lstset{style=mystyle}

% Header and Footer
\pagestyle{fancy}
\fancyhead{}
\fancyfoot{}
\fancyhead[L]{\textit{\Large{OOPJC Assignment 1}}}
%\fancyhead[R]{\textit{something}}
\fancyfoot[C]{\thepage}
\renewcommand{\footrulewidth}{1pt}



% Other Doc Editing
% \parindent 0ex
%\renewcommand{\baselinestretch}{1.5}

\begin{document}
	
	\begin{titlepage} 
		\centering 
		
		%---------------------------NAMES-------------------------------
		
		\huge\textsc{
			MIT World Peace University
		}\\
	
		\vspace{0.75\baselineskip} % space after Uni Name
		
		\LARGE{
			Object Oriented Programming with Java and C++\\
			Second Year B. Tech, Semester 1
		}
		
		\vfill % space after Sub Name
		
		%--------------------------TITLE-------------------------------
		
		\rule{\textwidth}{1.6pt}\vspace*{-\baselineskip}\vspace*{2pt}
		\rule{\textwidth}{0.6pt}
		\vspace{0.75\baselineskip} % Whitespace above the title
		
		
		
		\huge{\textsc{
			To demonstrate the use of objects, classes, constructors and destructors using C++ and JAVA.
			}} \\
		
		
		
		\vspace{0.5\baselineskip} % Whitespace below the title
		\rule{\textwidth}{0.6pt}\vspace*{-\baselineskip}\vspace*{2.8pt}
		\rule{\textwidth}{1.6pt}
		
		\vspace{1\baselineskip} % Whitespace after the title block

		%--------------------------SUBTITLE --------------------------	
			
		\LARGE\textsc{
			Practical Report
		} % Subtitle or further description
		\vfill
		
		%--------------------------AUTHOR-------------------------------
		
		Prepared By
		\vspace{0.5\baselineskip} % Whitespace before the editors
		
		\Large{
			Krishnaraj Thadesar \\
			Cyber Security and Forensics\\
			Batch A2, PA 34
		}
		
		
		\vspace{0.5\baselineskip} % Whitespace below the editor list
		\today

	\end{titlepage}
	
	
\tableofcontents
\thispagestyle{empty}
\clearpage


\setcounter{page}{1}

\section{Aim and Objectives}
To demonstrate the use of objects, classes, constructors and destructors using C++
and JAVA.
\begin{enumerate}
	\item To study various OOP concepts
	\item To acquaint with the use of objects and classes in C++ and Java.
	\item To learn implementation of constructor, destructors and dynamic memory
	allocation
\end{enumerate}

\section{Problem Statement}

Develop an object-oriented program to create a database of employee information system
containing the following information: Employee Name, Employee number, qualification,
address, contact number, salary details (basic, DA, TA, Net salary), etc. Construct the
database with suitable member functions for initializing and destroying the data viz.
constructor, default constructor, Copy constructor, destructor. Use dynamic memory
allocation concept while creating and destroying the object of a class. Use static data
member concept wherever required. Accept and display the information of Employees.

\section{Theory}

% Explain following concepts with their syntax and appropriate example in C++ and Java
% Algorithm:
% Class
% Object
% Default, Parameterized and copy Constructor
% Destructor
% Use dynamic allocation and deallocation

\section{Algorithm}

\begin{enumerate}
	\item Start
	\item Create Employee Class
	\item Delcare appropirate Data Memebers and define the member funtions
	\item Accept multiple employee's Data using an Array of Objects
	\item Assign some Basic employees using default, copy and parameterized constructors
	\item Show usage of Default Constructor and display the Data
	\item Show usage of Parameterized constructor and display that data
	\item Show usage of Copy Constructor and display that Data. 
	\item Distory the objects if possible
	\item End
\end{enumerate}

\section{Platform}
	\textbf{Operating System}: Arch Linux x86-64\\
	\textbf{IDEs or Text Editors Used}: Visual Studio Code\\
	\textbf{Compilers} : g++ and gcc on linux for C++, and javac, with JDK 18.0.2 for Java\\

\section{Input}

\begin{enumerate}
	\item Number of Employees for enrollment
	\item Employee ID
	\item Employee Name
	\item Employee Position
	\item Employee Address
	\item Employee Salary
\end{enumerate}

\section{Output}

Employee Data should be displayed by use of member functions. 

\section{Conclusion}
Thus, learned to use objects, classes, constructor and destructor and implemented solution
of the given problem statement using C++ and Java.

\section{Code}

\subsection{Java Implementation}

\lstinputlisting[language=java, caption=Employee.java]{../Programs/java_imp/src/Employee.java}

\lstinputlisting[language=java, caption=Source.java]{../Programs/java_imp/src/Source.java}

\subsubsection{Java Input}
\begin{lstlisting}[language=bash, caption=Python example]
Enter the number of employees :
2
Default Constructor was called
Enter the age :
35
Employee ID is:
006
Employee Name:
Peter
Employee Age:
17
Employee Position:
Avenger
Employee basic Salary:
500000
Employee DA:
3440
Employee TA:
3550
Employee Address City:
Brooklyn


Default Constructor was called
Enter the age :
4
Employee ID is:
007
Employee Name:
Thor
Employee Age:
1500
Employee Position:
Avenger
Employee basic Salary:
6000000
Employee DA:
3000
Employee TA:
5000
Employee Address City:
Asgard

\end{lstlisting}

\subsubsection{Java Output}
\begin{lstlisting}[language=bash, caption=Python example]
This is the first Assignment
Default Constructor was called
Copy constructor was called
Parameterized constructor was called
Information about the CEO
Employee ssn is: 1
Employee ID is : 1000
Employee Name: Kom Pany Seeio
Employee Age: 45
Employee Position: CEO
Employee basic Salary: 1000000
Employee DA: 1000
Employee TA: 2000
Employee Gross Salary: 853000.0
Employee Address City: Seoul


Information about the President
Employee ssn is: 2
Employee ID is : 1000
Employee Name: Precy Dent
Employee Age: 45
Employee Position: President
Employee basic Salary: 2000000
Employee DA: 1000
Employee TA: 2000
Employee Gross Salary: 1703000.0
Employee Address City: Delhi


Information about the President
Employee ssn is: 3
Employee ID is : 1003
Employee Name: Visey Presed Ent
Employee Age: 50
Employee Position: Vice President
Employee basic Salary: 200000
Employee DA: 3000
Employee TA: 1000
Employee Gross Salary: 174000.0
Employee Address City: Mumbai


Employee ssn is: 4
Employee ID is : 6
Employee Name: Peter
Employee Age: 17
Employee Position: Avenger
Employee basic Salary: 500000
Employee DA: 3440
Employee TA: 3550
Employee Gross Salary: 431990.0
Employee Address City: Brooklyn


Employee ssn is: 5
Employee ID is : 7
Employee Name: Thor
Employee Age: 1500
Employee Position: Avenger
Employee basic Salary: 6000000
Employee DA: 3000
Employee TA: 5000
Employee Gross Salary: 5108000.0
Employee Address City: Asgard

\end{lstlisting}

\subsection{C++ Implementation}

\lstinputlisting[language=c++, caption=Main.Cpp]{../Programs/cpp_implementation/Assignment_1.cpp}

\subsubsection{C++ Input}
\begin{lstlisting}[language=bash, caption=C++ Input]
The Default Constructor was called
Copy Constructor was called
Parameterized constructor was called
How many values do you wanna input ? 2
The Default Constructor was called
The Default Constructor was called

Enter the Details

Enter information about the Employee Number: 1
Enter the Employee ID:
005
Enter the Employee Name:
Tony
Enter the Employee Age:
40
Enter the Employee Position:
Philanthropist
Enter the Employee basic Salary:
5000000
Enter the Employee DA:
3000
Enter the Employee TA:
3000
Enter the Employee Address City:
NewYork

Enter information about the Employee Number: 2
Enter the Employee ID:
006
Enter the Employee Name:
Steve
Enter the Employee Age:
105
Enter the Employee Position:
Captain
Enter the Employee basic Salary:
600000
Enter the Employee DA:
2999
Enter the Employee TA:
2000
Enter the Employee Address City:
Brooklyn

\end{lstlisting}

\subsubsection{C++ Output}
\begin{lstlisting}[language=bash, caption=C++ Output]
Information about the CEO
Employee ssn is:1003
Employee ID is : 1000
Employee Name: Kom Pany Seeio
Employee Age: 45
Employee Position: CEO
Employee basic Salary: 1000000
Employee DA: 1000
Employee TA: 2000
Employee Gross Salary: 853000
Employee Address City: Seoul

Information about the President
Employee ssn is:1003
Employee ID is : 1000
Employee Name: Precy Dent
Employee Age: 45
Employee Position: President
Employee basic Salary: 2000000
Employee DA: 1000
Employee TA: 2000
Employee Gross Salary: 1.703e+06
Employee Address City: Delhi

Information about the Vice President
Employee ssn is:1003
Employee ID is : 1003
Employee Name: Visey Presed Ent
Employee Age: 50
Employee Position: Vice President
Employee basic Salary: 200000
Employee DA: 3000
Employee TA: 1000
Employee Gross Salary: 174000
Employee Address City: Mumbai

Information about the Employee Number: 1
Employee ssn is:1003
Employee ID is : 5
Employee Name: Tony
Employee Age: 40
Employee Position: Philanthropist
Employee basic Salary: 5000000
Employee DA: 3000
Employee TA: 3000
Employee Gross Salary: 4.256e+06
Employee Address City: NewYork

Information about the Employee Number: 2
Employee ssn is:1003
Employee ID is : 6
Employee Name: Steve
Employee Age: 105
Employee Position: Captain
Employee basic Salary: 600000
Employee DA: 2999
Employee TA: 2000
Employee Gross Salary: 514999
Employee Address City: Brooklyn

The Destructor was called
The Destructor was called
The Destructor was called
The Destructor was called
The Destructor was called
\end{lstlisting}

\pagebreak

\section{FAQs}

\begin{enumerate}
	\item What are classes?\\

	In object-oriented programming, a class is a blueprint for creating objects (a particular data structure), providing initial values for state (member variables or attributes), and implementations of behavior (member functions or methods).
	\\ It is a basic concept of Object-Oriented Programming which revolve around the real-life entities. 
	
	\begin{verbatim}
		class <class_name>{  
			field;  
			method;  
  	}
	\end{verbatim}

	\item Explain : Array of Objects: \\
 
	An array of objects is like any other array in C++ and Java. An Array usually is just a collection of variables that have the same data type, and are placed in contiguous memory locations. An Array of Objects is similar in that instead of variables there are objects which are placed contigiously in memory. 

	\begin{verbatim}
	Syntax: 
	Employee obj[5];
	\end{verbatim}
	\item Explain when to use different types of constructors?
	There are 3 Types of constructors: 
	\begin{enumerate}
		\item Default Constructor: This type is called when the Object is just created in its most simple declaration. 
		It does not take any parameters, or arguements. So if you have a simple class, that does not have many user dependent variables and fields, that does a rather general task, then it is better to use default constructors, where you do not have to assign any user variables to class variables, and have to just call some basic intantiating functions depending on class requirements and functions. 
		\item Parameterized Constructor: Say there are variables that the user has entered that need to be assigned to the class object, or there are certain properties of each object different from other objects of the same class, like in enemies in a game, or employees in a Company, each object can be initialized with a set of variables. In this situation it is better to just use a parameterized constructor. 
		\item Copy Constructor: If you have many constructors that are often similar in defintion and declaration, but have very few dissimilar properties, it is better to use copy constructors. For example Trees in a RPG game, where each tree has the same basic structure, but you might have small variation in just the height or the position of the tree. 
	\end{enumerate}
	\item Explain use of static member functions.\\
	
	Static member functions in java are those that can be accessed by other classes without declaring an Object of that class. This is often why the main function needs to be public and static. Every other member function needs to be accessed by an obejct of that class, as opposed to static ones. 

	In terms of memory, the static keyword in C++ is used when you have a variable that needs to be accessed by several objects of the same class, and this variable doesnt need to be different for each object. An Example would be the Security number of an Employee, which just needs to be incremented as an object is created. It is accessed by each object, and therefore it makes sense for it to be declared in a way where it does not get copied for each object, thereby saving space. 
	\item How java program is executed?\\
	\item What is the use of JVM?\\
	\item What are the different control statements used in C++ and Java?\\
	\item Write couple of examples/applications suitable to use OOP concepts specially use
	of classes, objects and constructors.\\
	\begin{enumerate}
		\item Game Development: Enemies, walls, obstacles, trees, NPCs, are often structured as classes. This is because they have a set template that each member follows, and there are often many of them in a game. This makes OOP the perfect choice. 
		\item Machine Learning: Machine learning often requires extensive and complex algorithms that need to be written and applied on a set of data. If such algorithms are put together as classes, then their objects can be fed that data and that algorithm can be run on it efficiently and easily, as opposed to writing it every time for each data set. 
		\item Software Development: GUI components like buttons, sliders, panels, frames, bars, etc often have a singular functionality associated with them. As there are many such components in a GUI, it makes sense to make them into classes, and spawn their objects in various meaningful positions in the UI. 
	\end{enumerate}
\end{enumerate}
	
\end{document}