\documentclass{article}
\usepackage{listings}
\usepackage{xcolor}

\definecolor{codegreen}{rgb}{0,0.6,0}
\definecolor{codegray}{rgb}{0.5,0.5,0.5}
\definecolor{codepurple}{rgb}{0.58,0,0.82}
\definecolor{backcolour}{rgb}{0.95,0.95,0.92}

\lstdefinestyle{mystyle}{
    backgroundcolor=\color{backcolour},   
    commentstyle=\color{codegreen},
    keywordstyle=\color{magenta},
    numberstyle=\tiny\color{codegray},
    stringstyle=\color{codepurple},
    basicstyle=\ttfamily\footnotesize,
    breakatwhitespace=false,         
    breaklines=true,                 
    captionpos=b,                    
    keepspaces=true,                 
    numbers=left,                    
    numbersep=5pt,                  
    showspaces=false,                
    showstringspaces=false,
    showtabs=false,                  
    tabsize=2
}

\lstset{style=mystyle}

\begin{document}
\begin{lstlisting}[language=Python, caption=Python example]
    import numpy as np
        
    def incmatrix(genl1,genl2):
        m = len(genl1)
        n = len(genl2)
        M = None #to become the incidence matrix
        VT = np.zeros((n*m,1), int)  #dummy variable
        
        #compute the bitwise xor matrix
        M1 = bitxormatrix(genl1)
        M2 = np.triu(bitxormatrix(genl2),1) 
    
        for i in range(m-1):
            for j in range(i+1, m):
                [r,c] = np.where(M2 == M1[i,j])
                for k in range(len(r)):
                    VT[(i)*n + r[k]] = 1;
                    VT[(i)*n + c[k]] = 1;
                    VT[(j)*n + r[k]] = 1;
                    VT[(j)*n + c[k]] = 1;
                    
                    if M is None:
                        M = np.copy(VT)
                    else:
                        M = np.concatenate((M, VT), 1)
                    
                    VT = np.zeros((n*m,1), int)
        
        return M
    \end{lstlisting}
\lstinputlisting[language=java]{../Programs/java_imp/src/Employee.java}
% \begin{lstlisting}[language=Python]
%     import numpy as np
        
%     def incmatrix(genl1,genl2):
%         m = len(genl1)
%         n = len(genl2)
%         M = None #to become the incidence matrix
%         VT = np.zeros((n*m,1), int)  #dummy variable
        
%         #compute the bitwise xor matrix
%         M1 = bitxormatrix(genl1)
%         M2 = np.triu(bitxormatrix(genl2),1) 
    
%         for i in range(m-1):
%             for j in range(i+1, m):
%                 [r,c] = np.where(M2 == M1[i,j])
%                 for k in range(len(r)):
%                     VT[(i)*n + r[k]] = 1;
%                     VT[(i)*n + c[k]] = 1;
%                     VT[(j)*n + r[k]] = 1;
%                     VT[(j)*n + c[k]] = 1;
                    
%                     if M is None:
%                         M = np.copy(VT)
%                     else:
%                         M = np.concatenate((M, VT), 1)
                    
%                     VT = np.zeros((n*m,1), int)
        
%         return M
%     \end{lstlisting}

\end{document}