% This is a Basic Assignment Paper but with like Code and stuff allowed in it. 

\documentclass[11pt]{article}

% Preamble

\usepackage[margin=1in]{geometry}
\usepackage{amsfonts, amsmath, amssymb}
\usepackage{fancyhdr, float, graphicx}
\usepackage[utf8]{inputenc} % Required for inputting international characters
\usepackage[T1]{fontenc} % Output font encoding for international characters
\usepackage{fouriernc} % Use the New Century Schoolbook font
\usepackage[nottoc, notlot, notlof]{tocbibind}
\usepackage{listings}
\usepackage{xcolor}

\definecolor{codegreen}{rgb}{0,0.6,0}
\definecolor{codegray}{rgb}{0.5,0.5,0.5}
\definecolor{codepurple}{rgb}{0.58,0,0.82}
\definecolor{backcolour}{rgb}{0.95,0.95,0.92}

\lstdefinestyle{mystyle}{
    backgroundcolor=\color{backcolour},   
    commentstyle=\color{codegreen},
    keywordstyle=\color{magenta},
    numberstyle=\tiny\color{codegray},
    stringstyle=\color{codepurple},
    basicstyle=\ttfamily\footnotesize,
    breakatwhitespace=false,         
    breaklines=true,                 
    captionpos=b,                    
    keepspaces=true,                 
    numbers=left,                    
    numbersep=5pt,                  
    showspaces=false,                
    showstringspaces=false,
    showtabs=false,                  
    tabsize=2
}

\lstset{style=mystyle}

% Header and Footer
\pagestyle{fancy}
\fancyhead{}
\fancyfoot{}
\fancyhead[L]{\textit{\Large{OOPJC Assignment 4}}}
%\fancyhead[R]{\textit{something}}
\fancyfoot[C]{\thepage}
\renewcommand{\footrulewidth}{1pt}



% Other Doc Editing
% \parindent 0ex
%\renewcommand{\baselinestretch}{1.5}

\begin{document}

\begin{titlepage}
	\centering

	%---------------------------NAMES-------------------------------

	\huge\textsc{
		MIT World Peace University
	}\\

	\vspace{0.75\baselineskip} % space after Uni Name

	\LARGE{
		Object Oriented Programming with Java and C++\\
		Second Year B. Tech, Semester 1
	}

	\vfill % space after Sub Name

	%--------------------------TITLE-------------------------------

	\rule{\textwidth}{1.6pt}\vspace*{-\baselineskip}\vspace*{2pt}
	\rule{\textwidth}{0.6pt}
	\vspace{0.75\baselineskip} % Whitespace above the title



	\huge{\textsc{
		Collection Frameworks in Java
		}} \\



	\vspace{0.5\baselineskip} % Whitespace below the title
	\rule{\textwidth}{0.6pt}\vspace*{-\baselineskip}\vspace*{2.8pt}
	\rule{\textwidth}{1.6pt}

	\vspace{1\baselineskip} % Whitespace after the title block

	%--------------------------SUBTITLE --------------------------	

	\LARGE\textsc{
		Practical Report\\
		Assignment 6
	} % Subtitle or further description
	\vfill

	%--------------------------AUTHOR-------------------------------

	Prepared By
	\vspace{0.5\baselineskip} % Whitespace before the editors

	\Large{
		Krishnaraj Thadesar \\
		Cyber Security and Forensics\\
		Batch A1, PA 20
	}


	\vspace{0.5\baselineskip} % Whitespace below the editor list
	\today

\end{titlepage}


\tableofcontents
\thispagestyle{empty}
\clearpage


\setcounter{page}{1}

\section{Aim and Objectives}
\subsection*{Aim}
Demonstrating the use of Collection Frameworks in Java.
\subsection*{Objective}
\begin{itemize}
	\item To study the concept of collection framework
	\item To get familiar with features of collection framework in Java
\end{itemize}
\section{Problem Statement}
Write a Java Program to: 
\begin{itemize}
	\item create a new array list and print out the collection by position
	\item add some elements (string)
	\item search an element in an array list
	\item reverse elements in an array list
	\item test an array list is empty or not
	\item join two array lists
\end{itemize}
	
\section{Theory}
\subsection{Concept of collection framework in Java}

\subsection{Benefit of Generics in Collections Framework}

\subsection{Basic interfaces of Java Collections Framework}

\subsection{Types of collections allowed by the Java collections framework, Explain with suitable syntax and examples}

\section{Platform}
\textbf{Operating System}: Arch Linux x86-64 \\
\textbf{IDEs or Text Editors Used}: Visual Studio Code\\
\textbf{Compilers} : g++ and gcc on linux for C++, and javac, with JDK 18.0.2 for Java\\

\section{Input}

\subsection*{For C++}
\begin{enumerate}
	\item Number of Each Type of Employee
	\item Name, Age, Address City, and Salary of Each Employee
\end{enumerate}

\subsection*{For Java}
\begin{enumerate}
	\item The Side of the Square
	\item The Radius of the Circle
	\item The Length and Breadth of the Rectangle.
\end{enumerate}

\section{Output}
\subsection*{For C++}
\begin{enumerate}
	\item General Information about Each Employee
	\item The Weekly, hourly and commisioned Salary for Respective Employees.
\end{enumerate}

\subsection*{For Java}
\begin{enumerate}
	\item The Area of the Shapes
	\item The Location of the Hill Stations
	\item The Reason the Hill stations are Famous for.
\end{enumerate}


\section{Code}
\lstinputlisting[language=Java, caption=Assignment_6.java]{../Programs/cpp_implementations/Assignment_4.cpp}
\subsubsection{Output}
\lstinputlisting[caption=Output]{../Programs/cpp_implementations/Assignment_4_output.txt}


\section{Conclusion}
Thus, learnt the use of collection framework in java and performed array list operations
\pagebreak

\section{FAQs}

\begin{enumerate}
	\item Why do we use collection framework?

	\item Which is best collection framework in Java?

	\item What is difference between array and collection?

	\item What is HashMap in Java?

\end{enumerate}

\end{document}