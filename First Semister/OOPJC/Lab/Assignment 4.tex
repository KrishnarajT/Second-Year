% This is a Basic Assignment Paper but with like Code and stuff allowed in it. 

\documentclass[11pt]{article}

% Preamble

\usepackage[margin=1in]{geometry}
\usepackage{amsfonts, amsmath, amssymb}
\usepackage{fancyhdr, float, graphicx}
\usepackage[utf8]{inputenc} % Required for inputting international characters
\usepackage[T1]{fontenc} % Output font encoding for international characters
\usepackage{fouriernc} % Use the New Century Schoolbook font
\usepackage[nottoc, notlot, notlof]{tocbibind}
\usepackage{listings}
\usepackage{xcolor}

\definecolor{codegreen}{rgb}{0,0.6,0}
\definecolor{codegray}{rgb}{0.5,0.5,0.5}
\definecolor{codepurple}{rgb}{0.58,0,0.82}
\definecolor{backcolour}{rgb}{0.95,0.95,0.92}

\lstdefinestyle{mystyle}{
    backgroundcolor=\color{backcolour},   
    commentstyle=\color{codegreen},
    keywordstyle=\color{magenta},
    numberstyle=\tiny\color{codegray},
    stringstyle=\color{codepurple},
    basicstyle=\ttfamily\footnotesize,
    breakatwhitespace=false,         
    breaklines=true,                 
    captionpos=b,                    
    keepspaces=true,                 
    numbers=left,                    
    numbersep=5pt,                  
    showspaces=false,                
    showstringspaces=false,
    showtabs=false,                  
    tabsize=2
}

\lstset{style=mystyle}

% Header and Footer
\pagestyle{fancy}
\fancyhead{}
\fancyfoot{}
\fancyhead[L]{\textit{\Large{OOPJC Assignment 43}}}
%\fancyhead[R]{\textit{something}}
\fancyfoot[C]{\thepage}
\renewcommand{\footrulewidth}{1pt}



% Other Doc Editing
% \parindent 0ex
%\renewcommand{\baselinestretch}{1.5}

\begin{document}

\begin{titlepage}
	\centering

	%---------------------------NAMES-------------------------------

	\huge\textsc{
		MIT World Peace University
	}\\

	\vspace{0.75\baselineskip} % space after Uni Name

	\LARGE{
		Object Oriented Programming with Java and C++\\
		Second Year B. Tech, Semester 1
	}

	\vfill % space after Sub Name

	%--------------------------TITLE-------------------------------

	\rule{\textwidth}{1.6pt}\vspace*{-\baselineskip}\vspace*{2pt}
	\rule{\textwidth}{0.6pt}
	\vspace{0.75\baselineskip} % Whitespace above the title



	\huge{\textsc{
			Understanding and Implementation of Exception Handling Concepts in C++ and Java.
		}} \\



	\vspace{0.5\baselineskip} % Whitespace below the title
	\rule{\textwidth}{0.6pt}\vspace*{-\baselineskip}\vspace*{2.8pt}
	\rule{\textwidth}{1.6pt}

	\vspace{1\baselineskip} % Whitespace after the title block

	%--------------------------SUBTITLE --------------------------	

	\LARGE\textsc{
		Practical Report
	} % Subtitle or further description
	\vfill

	%--------------------------AUTHOR-------------------------------

	Prepared By
	\vspace{0.5\baselineskip} % Whitespace before the editors

	\Large{
		Krishnaraj Thadesar \\
		Cyber Security and Forensics\\
		Batch A2, PA 20
	}


	\vspace{0.5\baselineskip} % Whitespace below the editor list
	\today

\end{titlepage}


\tableofcontents
\thispagestyle{empty}
\clearpage


\setcounter{page}{1}

\section{Aim and Objectives}
Implementation and Understanding of Exception handling in Java and C++, and to learn and use the exception handling mechanisms with try and catch blocks. 

\section{Problem Statements}


\subsection{Problem 1 in C++}
Define a class Employee consisting following:\\
\textbf{Data Members}\\
\begin{enumerate}
	\item Employee ID
	\item Name of Employee
	\item Age
	\item Income
	\item City
	\item Vehicle
\end{enumerate}
\textbf{Member Functions}\\
\begin{enumerate}
	\item To assign initial values.
	\item To display.
\end{enumerate}
Accept Employee ID, Name, Age, Income, City and Vehicle from the user. Create
an exception to check the following conditions and throw an exception if the
condition does not meet.

\begin{itemize}
	\item Employee age between 18 and 55
	\item Employee income between Rs. 50,000 - Rs. 1,00,000 per month
	\item Employee staying in Pune/ Mumbai/ Bangalore / Chennai
	\item Employee having 4-wheeler
\end{itemize}


\subsection{Problem 2 in Java}
Implement the program to handle the arithmetic exception, ArrayIndexOutofBounds .
The user enters the two numbers: n1, n2. The division of n1 and n2 is displayed. If n1, n2
are not integers then program will throw number format exception. If n2 is zero the
program will throw Arithmetic exception.

\subsection{Problem 3 in Java}
Validate the employee record with custom exception
Create a class employee with attributes eid, name, age and department.
Initialize values through parameterized constructor. If age of employee is not in between
25 and 60 then generate user-defined exception "AgeNotWithinRangeException". If
name contains numbers or special symbols raise exception "NameNotValidException".
Define the two exception classes.

\subsection{Problem 4 in Java}
Write a menu-driven program for banking system which accept the personal data for
Customer(cid, cname, amount).
Implement the user-defined/standard exceptions, wherever required to handle the
following situations:

\begin{enumerate}
	\item Account should be created with minimum amount of 1000 Rs. 
	\item For withdrawal of amount, if withdrawal Amount is greater than the Amount in the Account.
	\item Customer Id should be between 1 and 20 only. 
	\item Entered amount should be positive. 
\end{enumerate}
\section{Theory}

\subsection{Concept of Compile time Polymorphism}

\subsection{Concept of Run Time Polymorphism}

\subsection{Use of Pure Virtual Functions}

\section{Platform}
\textbf{Operating System}: Arch Linux x86-64 \\
\textbf{IDEs or Text Editors Used}: Visual Studio Code\\
\textbf{Compilers} : g++ and gcc on linux for C++, and javac, with JDK 18.0.2 for Java\\

\section{Input}

\subsection*{For C++}
\begin{enumerate}
	\item Number of Each Type of Employee
	\item Name, Age, Address City, and Salary of Each Employee
\end{enumerate}

\subsection*{For Java}
\begin{enumerate}
	\item The Side of the Square
	\item The Radius of the Circle
	\item The Length and Breadth of the Rectangle. 
\end{enumerate}

\section{Output}
\subsection*{For C++}
\begin{enumerate}
	\item General Information about Each Employee
	\item The Weekly, hourly and commisioned Salary for Respective Employees. 
\end{enumerate}

\subsection*{For Java}
\begin{enumerate}
	\item The Area of the Shapes
	\item The Location of the Hill Stations
	\item The Reason the Hill stations are Famous for. 
\end{enumerate}


\section{Code}
\subsection{C++ Implementation of Problem A}

\lstinputlisting[language=c++, caption=Main.Cpp]{../Programs/cpp_implementations/Assignment_4.cpp}

\subsubsection{C++ Output}
\lstinputlisting[caption=Output for Problem 1]{../Programs/cpp_implementations/Assignment_4_output.txt}

\subsection{Java Implementation of Problem B}

\lstinputlisting[language=java, caption=Full Time Employee.java]{../Programs/java_implementations/assignment_4/Division.java}

\subsubsection{Java Output}
\lstinputlisting[caption=Output for Problem 2]{../Programs/java_implementations/assignment_4/problem_b_output.txt}

\subsection{Java Implementation of Problem C} 

\lstinputlisting[language=java, caption=HillStation]{../Programs/java_implementations/assignment_4/Employee.java}

\subsubsection{Java Output}
\lstinputlisting[caption=Output for Problem 3]{../Programs/java_implementations/assignment_4/Problem_B_output.txt}

\subsection{Java Implementation of Problem D} 

\lstinputlisting[language=java, caption=HillStation]{../Programs/java_implementations/assignment_4/Bank.java}

\lstinputlisting[language=java, caption=HillStation]{../Programs/java_implementations/assignment_4/Main.java}
\subsubsection{Java Output}
\lstinputlisting[language=java, caption=Main.java]{../Programs/java_implementations/assignment_4/problem_d_output.txt}


\section{Conclusion}
Thus, learned to use polymorphism and implemented solution of the given problem statement using C++ and Java. 
\pagebreak

\section{FAQs}

\begin{enumerate}
	\item \textbf{Why do we use Exception Handling mechanism?}
	\item \textbf{Is it possible to use multiple catch for single throw? Explain?}
	\item \textbf{What is Exception Specification?}
	\item \textbf{What is Re-throwing Exception?}
	\item \textbf{Explain use of finally keyword in java.}
\end{enumerate}

\end{document}