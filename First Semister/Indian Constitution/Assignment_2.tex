% This is a Basic Assignment Paper but with like Code and stuff allowed in it, there is also url, hyperlinks from contents included. 

\documentclass[11pt]{article}

% Preamble

\usepackage[margin=1in]{geometry}
\usepackage{amsfonts, amsmath, amssymb}
\usepackage{fancyhdr, float, graphicx}
\usepackage[utf8]{inputenc} % Required for inputting international characters
\usepackage[T1]{fontenc} % Output font encoding for international characters
\usepackage{fouriernc} % Use the New Century Schoolbook font
\usepackage[nottoc, notlot, notlof]{tocbibind}
\usepackage{listings}
\usepackage{xcolor}
\usepackage{blindtext}
\usepackage{hyperref}
\hypersetup{
    colorlinks=true,
    linkcolor=black,
    filecolor=magenta,      
    urlcolor=cyan,
    pdfpagemode=FullScreen,
    }

\definecolor{codegreen}{rgb}{0,0.6,0}
\definecolor{codegray}{rgb}{0.5,0.5,0.5}
\definecolor{codepurple}{rgb}{0.58,0,0.82}
\definecolor{backcolour}{rgb}{0.95,0.95,0.92}

\lstdefinestyle{mystyle}{
    backgroundcolor=\color{backcolour},   
    commentstyle=\color{codegreen},
    keywordstyle=\color{magenta},
    numberstyle=\tiny\color{codegray},
    stringstyle=\color{codepurple},
    basicstyle=\ttfamily\footnotesize,
    breakatwhitespace=false,         
    breaklines=true,                 
    captionpos=b,                    
    keepspaces=true,                 
    numbers=left,                    
    numbersep=5pt,                  
    showspaces=false,                
    showstringspaces=false,
    showtabs=false,                  
    tabsize=2
}

\lstset{style=mystyle}

% Header and Footer
\pagestyle{fancy}
\fancyhead{}
\fancyfoot{}
\fancyhead[L]{\textit{\Large{OOPJC Mini Project Report}}}
%\fancyhead[R]{\textit{something}}
\fancyfoot[C]{\thepage}
\renewcommand{\footrulewidth}{1pt}



% Other Doc Editing
% \parindent 0ex
%\renewcommand{\baselinestretch}{1.5}

\begin{document}

\begin{titlepage}
	\centering

	%---------------------------NAMES-------------------------------

	\huge\textsc{
		MIT World Peace University
	}\\

	\vspace{0.75\baselineskip} % space after Uni Name

	\LARGE{
		Indian Constitution\\
		Second Year B. Tech, Semester 1
	}

	\vfill % space after Sub Name

	%--------------------------TITLE-------------------------------

	\rule{\textwidth}{1.6pt}\vspace*{-\baselineskip}\vspace*{2pt}
	\rule{\textwidth}{0.6pt}
	\vspace{0.75\baselineskip} % Whitespace above the title



	\huge{\textsc{
			Influence of Other Countries on Indian Constitution
		}} \\



	\vspace{0.5\baselineskip} % Whitespace below the title
	\rule{\textwidth}{0.6pt}\vspace*{-\baselineskip}\vspace*{2.8pt}
	\rule{\textwidth}{1.6pt}

	\vspace{1\baselineskip} % Whitespace after the title block

	%--------------------------SUBTITLE --------------------------	

	\LARGE\textsc{
		Theory Assignment 2
	} % Subtitle or further description
	\vfill

	%--------------------------AUTHOR-------------------------------

	Prepared By
	\vspace{0.5\baselineskip} % Whitespace before the editors

	\Large{
		Krishnaraj Thadesar \\
		Cyber Security and Forensics\\
		Batch A1, PA 20
	}


	\vspace{0.5\baselineskip} % Whitespace below the editor list
	\today

\end{titlepage}


\tableofcontents
\thispagestyle{empty}
\clearpage

\setcounter{page}{1}

\section{Basic Definitons}
\subsection{Constitution}

\begin{quote}
	\textbf{Constitution} is a set of rules and principles that define the structure and function of a government. It is the supreme law of the land and is the foundation of a nation's legal system. The Constitution of India is the supreme law of India. It lays down the framework defining fundamental political principles, establishes the structure, procedures, powers and duties of government institutions and sets out fundamental rights, directive principles and the duties of citizens. It is the longest written constitution of any sovereign country in the world.
\end{quote}

\section{Influence of Other Countries on Indian Constitution}

The Indian Constitution, which was adopted on January 26, 1950, is one of the longest and most comprehensive constitutions in the world. It contains over 450 articles, 12 schedules, and 98 amendments, and covers a wide range of topics, including the fundamental rights and duties of citizens, the organization and powers of the government, and the procedures for amending the constitution.\\

Despite its length and complexity, the Indian Constitution was not created in a vacuum. It was heavily influenced by a number of other constitutions, both from within India and from other countries around the world. In this essay, we will explore some of the key influences on the Indian Constitution and how they helped shape the document into what it is today.\\

One of the most important influences on the Indian Constitution was the Government of India Act of 1935. This act, which was passed by the British Parliament, was intended to provide a framework for the governance of India, and it contained many provisions that were later incorporated into the Indian Constitution. For example, the 1935 act introduced the concept of a federal system of government, with a division of powers between the central government and the provinces. It also provided for a bicameral legislature, consisting of a lower house (the Legislative Assembly) and an upper house (the Council of States).\\

Another significant influence on the Indian Constitution was the Constitution of the United States. The framers of the Indian Constitution were heavily influenced by the principles of federalism, separation of powers, and checks and balances that are found in the US Constitution. In fact, many of the provisions of the Indian Constitution, such as the establishment of a supreme court and the powers of the president, are similar to those found in the US Constitution.\\

Additionally, the Indian Constitution was influenced by the constitutions of other countries, such as Canada and Japan. For example, the Indian Constitution includes a provision for the establishment of a public service commission, which is similar to the civil service commission found in the Canadian Constitution. Similarly, the Indian Constitution includes a provision for the protection of linguistic and cultural minorities, which is similar to the provisions found in the Japanese Constitution.\\

Overall, the Indian Constitution is a unique document that reflects the diverse influences of different constitutions from around the world. It combines the principles of federalism, separation of powers, and checks and balances with the specific needs and concerns of the people of India. As a result, it is a complex and dynamic document that has played a crucial role in the development and governance of India since its adoption in 1950.\\


\section{What makes it Unique?}

The Indian Constitution is a unique document that sets forth the fundamental laws and principles governing the Republic of India. Adopted on January 26, 1950, the constitution is one of the longest and most comprehensive in the world, containing over 450 articles, 12 schedules, and 98 amendments.\\

One of the key features that makes the Indian Constitution unique is its length and complexity. While many other constitutions are relatively short and concise, the Indian Constitution is a sprawling document that covers a wide range of topics, including the fundamental rights and duties of citizens, the organization and powers of the government, and the procedures for amending the constitution. This comprehensive approach reflects the diversity and complexity of Indian society, and the need to address a wide range of issues and concerns in a single document.\\

Another unique aspect of the Indian Constitution is its emphasis on the protection of individual rights. The constitution guarantees a wide range of fundamental rights to all Indian citizens, including the right to equality, freedom of speech and expression, and the protection of life and liberty. It also includes a number of provisions that aim to protect the rights of disadvantaged and marginalized groups, such as the Scheduled Castes and Tribes, and the provisions for the protection of linguistic and cultural minorities.\\

In addition to its emphasis on individual rights, the Indian Constitution also contains a number of provisions that reflect the country's commitment to social and economic justice. For example, the constitution includes a directive principle of state policy, which sets forth a number of goals that the government is expected to pursue in order to promote the welfare of the people. These goals include the promotion of social and economic justice, the provision of opportunities for education and employment, and the protection of the environment.\\

Finally, the Indian Constitution is unique in its flexibility and adaptability. The constitution includes a number of provisions that allow it to be amended in response to changing circumstances and the needs of the people. For example, the constitution can be amended by a simple majority vote in both houses of parliament, provided that the proposed amendment is approved by a majority of the state legislatures. This flexibility has allowed the constitution to evolve and adapt over time, ensuring that it remains relevant and effective in the face of changing social, economic, and political conditions.\\

In conclusion, the Indian Constitution is a unique document that reflects the country's history, culture, and values. Its length and complexity, its emphasis on the protection of individual rights, its commitment to social and economic justice, and its flexibility and adaptability make it a unique and powerful instrument of governance. Despite the challenges and difficulties that India has faced since its independence, the constitution has remained a cornerstone of the country.\\

\section{Conclusion}

The Indian Constitution is a unique document that sets forth the fundamental laws and principles governing the Republic of India. Adopted on January 26, 1950, the constitution is one of the longest and most comprehensive in the world, containing over 450 articles, 12 schedules, and 98 amendments.\\

One of the key features that makes the Indian Constitution unique is its length and complexity. While many other constitutions are relatively short and concise, the Indian Constitution is a sprawling document that covers a wide range of topics, including the fundamental rights and duties of citizens, the organization and powers of the government, and the procedures for amending the constitution. This comprehensive approach reflects the diversity and complexity of Indian society, and the need to address a wide range of issues and concerns in a single document.\\

\end{document}