% This is a Basic Assignment Paper but with like Code and stuff allowed in it, there is also url, hyperlinks from contents included. 

\documentclass[11pt]{article}

% Preamble

\usepackage[margin=1in]{geometry}
\usepackage{amsfonts, amsmath, amssymb}
\usepackage{fancyhdr, float, graphicx}
\usepackage[utf8]{inputenc} % Required for inputting international characters
\usepackage[T1]{fontenc} % Output font encoding for international characters
\usepackage{fouriernc} % Use the New Century Schoolbook font
\usepackage[nottoc, notlot, notlof]{tocbibind}
\usepackage{listings}
\usepackage{xcolor}
\usepackage{blindtext}
\usepackage{hyperref}
\hypersetup{
    colorlinks=true,
    linkcolor=black,
    filecolor=magenta,      
    urlcolor=cyan,
    pdfpagemode=FullScreen,
    }

\definecolor{codegreen}{rgb}{0,0.6,0}
\definecolor{codegray}{rgb}{0.5,0.5,0.5}
\definecolor{codepurple}{rgb}{0.58,0,0.82}
\definecolor{backcolour}{rgb}{0.95,0.95,0.92}

\lstdefinestyle{mystyle}{
    backgroundcolor=\color{backcolour},   
    commentstyle=\color{codegreen},
    keywordstyle=\color{magenta},
    numberstyle=\tiny\color{codegray},
    stringstyle=\color{codepurple},
    basicstyle=\ttfamily\footnotesize,
    breakatwhitespace=false,         
    breaklines=true,                 
    captionpos=b,                    
    keepspaces=true,                 
    numbers=left,                    
    numbersep=5pt,                  
    showspaces=false,                
    showstringspaces=false,
    showtabs=false,                  
    tabsize=2
}

\lstset{style=mystyle}

% Header and Footer
\pagestyle{fancy}
\fancyhead{}
\fancyfoot{}
\fancyhead[L]{\textit{\Large{Information and Cycbersecurity - 2nd Year B. Tech}}}
%\fancyhead[R]{\textit{something}}
\fancyfoot[C]{\thepage}
\renewcommand{\footrulewidth}{1pt}



% Other Doc Editing
% \parindent 0ex
%\renewcommand{\baselinestretch}{1.5}

\begin{document}

\begin{titlepage}
	\centering

	%---------------------------NAMES-------------------------------

	\huge\textsc{
		MIT World Peace University
	}\\

	\vspace{0.75\baselineskip} % space after Uni Name

	\LARGE{
		Information and Cybersecurity\\
		Second Year B. Tech, Semester 1
	}

	\vfill % space after Sub Name

	%--------------------------TITLE-------------------------------

	\rule{\textwidth}{1.6pt}\vspace*{-\baselineskip}\vspace*{2pt}
	\rule{\textwidth}{0.6pt}
	\vspace{0.75\baselineskip} % Whitespace above the title



	\huge{\textsc{
			Public Key Cryptographic Techniques\\
			\textit{"Rivest, Shamir, Adleman's Algorithm (RSA)"}
		}} \\



	\vspace{0.5\baselineskip} % Whitespace below the title
	\rule{\textwidth}{0.6pt}\vspace*{-\baselineskip}\vspace*{2.8pt}
	\rule{\textwidth}{1.6pt}

	\vspace{1\baselineskip} % Whitespace after the title block

	%--------------------------SUBTITLE --------------------------	

	\LARGE\textsc{
		Lab Assignment 4
	} % Subtitle or further description
	\vfill

	%--------------------------AUTHOR-------------------------------

	Prepared By
	\vspace{0.5\baselineskip} % Whitespace before the editors

	\Large{
		Krishnaraj Thadesar \\
		Cyber Security and Forensics\\
		Batch A1, PA 20
	}


	\vspace{0.5\baselineskip} % Whitespace below the editor list
	\today

\end{titlepage}


\tableofcontents
\thispagestyle{empty}
\clearpage

\setcounter{page}{1}

\section{Aim}
Write a program using JAVA or Python or C++ to implement RSA asymmetric key
algorithm.

\section{Objectives}
To understand the concepts of public key and private key

\section{Theory}

\subsection{Euler Totient Function}
Euler's Totient function, denoted as $\varphi(n)$, is a number theoretic function that counts the number of positive integers less than or equal to $n$ that are relatively prime to $n$. In other words, $\varphi(n)$ gives the number of integers in the range $1 \leq k \leq n$ such that $\gcd(k, n) = 1$.

Here is an example of how to calculate Euler's Totient function for a specific integer $n$:

Let's take $n = 12$. The prime factorization of $n$ is $n = 2^2 \cdot 3^1$, so we can use the formula for calculating $\varphi(n)$ in terms of the prime factorization of $n$:

$$\varphi(n) = n \cdot \prod_{p \mid n} \left(1 - \frac{1}{p}\right)$$

where the product is taken over all distinct prime factors of $n$. Plugging in the prime factorization of $n = 12$, we get:

\begin{align*}
\varphi(12) &= 12 \cdot \left(1 - \frac{1}{2}\right) \cdot \left(1 - \frac{1}{3}\right) \\
&= 12 \cdot \frac{1}{2} \cdot \frac{2}{3} = 4
\end{align*}

Therefore, the value of Euler's Totient function for $n = 12$ is $\varphi(12) = 4$. This means that there are 4 positive integers less than or equal to 12 that are relatively prime to 12: 1, 5, 7, and 11.

\subsection{Euclidean Algorithm}
The Euclidean Algorithm is a method for finding the greatest common divisor (GCD) of two integers. It works by repeatedly finding the remainder when one integer is divided by the other, and then replacing the larger integer with the remainder. This process is repeated until the remainder is zero, at which point the GCD is the last non-zero remainder.

Here is an example of the Euclidean Algorithm applied to finding the GCD of 54 and 24:

\begin{align*}
	54 & = 2 \cdot 24 + 6 \\
	24 & = 4 \cdot 6 + 0 \
\end{align*}

We start by dividing 54 by 24 and finding the remainder, which is 6. We then replace 54 with 24 and 24 with 6, and repeat the process by dividing 24 by 6 and finding the remainder, which is 0. Since the remainder is now zero, the GCD is the last non-zero remainder, which in this case is 6. Therefore, the GCD of 54 and 24 is 6.

\subsection{Extended Euclidean Algorithm}
The Extended Euclidean Algorithm is an extension of the Euclidean Algorithm that not only finds the GCD of two integers, but also finds two coefficients that can be used to express the GCD as a linear combination of the two integers. Specifically, given two integers $a$ and $b$, the Extended Euclidean Algorithm finds integers $x$ and $y$ such that:

$$ax + by = \gcd(a, b)$$

Here is an example of the Extended Euclidean Algorithm applied to finding the GCD of 54 and 24, along with the coefficients $x$ and $y$:

\begin{align*}
	54 & = 2 \cdot 24 + 6 \\
	24 & = 4 \cdot 6 + 0 \
\end{align*}

To start, we apply the Euclidean Algorithm as we did before to find the GCD of 54 and 24, which is 6. We then work backwards through the steps of the algorithm to find the coefficients $x$ and $y$.

Starting from the second-to-last step:

\begin{align*}
	6 & = 54 - 2 \cdot 24 \
\end{align*}

We can rearrange this equation to isolate 6:

\begin{align*}
	6 & = 54 - 2 \cdot 24 \\
	  & = 54 - 2(54 - 2 \cdot 24) \\
	  & = 5 \cdot 54 - 2 \cdot 24
\end{align*}

So, we have found that $x = 5$ and $y = -2$. Substituting these values into the original equation, we get:

\begin{align*}
	54(5) + 24(-2) & = 6 \
\end{align*}

Therefore, the GCD of 54 and 24 is 6, and it can be expressed as a linear combination of 54 and 24 with coefficients 5 and -2, respectively.

\subsection{RSA Algorithm}
\subsubsection{Key Generation}
\begin{enumerate}
	\item Selecting two large primes at random: $p$ and $q$.
	\item Computing their system modulus: $n = (p*q)$.
	\item Compute: $\phi(n) = (p-1)*(q-1)$.
	\item selecting at random the encryption key (public key): $e$ such that $1 < e < \phi(n)$ and $gcd(e, \phi(n)) = 1$.
	\item To find decryption key d such that $d*e \equiv 1 \pmod{\phi(n)}$.
	\item Public Encryption key: $PU = {e, n}$
	\item Private Decryption key: $PR = {d, n}$
\end{enumerate}
\subsubsection{RSA Encryption}
Encrypt the plain text $M$ using the public key $PU$.
\begin{enumerate}
	\item Computes Cipher text: $C = M^e \pmod{n}$, where 0 <= $M$ < $n$.
\end{enumerate}

\subsubsection{RSA Decryption}
Decrypt the cipher text $C$ using the private key $PR$.
\begin{enumerate}
	\item Computes Plaintext: $M = C^d \pmod{n}$
\end{enumerate}

\subsubsection{Example of RSA Encryption}
\begin{enumerate}
	\item Select two large primes at random: $p = 3$ and $q = 11$.
	\item Compute their system modulus: $n = (p*q) = 33$.
	\item Compute: $\phi(n) = (p-1)*(q-1) = 20$.
	\item Select at random the encryption key (public key): $e = 7$ such that $1 < e < \phi(n)$ and $gcd(e, \phi(n)) = 1$.
	\item To find decryption key d such that $d*e \equiv 1 \pmod{\phi(n)}$.
	\item Public Encryption key: $PU = {e, n} = {7, 33}$
	\item Private Decryption key: $PR = {d, n} = {3, 33}$
	\item Encrypt the plain text $M = 30$ using the public key $PU$.
	\item Computes Cipher text: $C = M^e \pmod{n} = 5^7 \pmod{33} = 24$.
	\item Decrypt the cipher text $C = 24$ using the private key $PR$.
	\item Computes Plaintext: $M = C^d \pmod{n} = 24^3 \pmod{33} = 30$.
\end{enumerate}

\section{Platform}
\textbf{Operating System}: Arch Linux x86-64 \\
\textbf{IDEs or Text Editors Used}: Visual Studio Code\\
\textbf{Compilers or Interpreters} : Python 3.10.1\\

\section{Input and Output}
\begin{verbatim}
Enter the Message to be encrypted: 
This Assignment's Due date is very near
private key is:  (14633, 31373)
public key is:  (89, 31373)
<encrypted text>
This Assignment's Due date is very near
\end{verbatim}
\section{Code}
\lstinputlisting[language=Python, caption="RSA Algorithm"]{../Programs/Assignment_4/rsa.py}

\section{Conclusion}
Thus, learnt about the different kinds of public key cryptography works. Also, learnt about the RSA algorithm and its implementation in Python in depth. We also tried to implement RSA on a dummy client server model using sockets in python.
\clearpage

\section{FAQ}

\begin{enumerate}
	\item \textit{Compare symmetric key cryptography and asymmetric key cryptography}\\
	
	      \begin{enumerate}
		      \item \textit{Symmetric Key Cryptography}
		            \begin{itemize}
			            \item \textit{Advantages}
			                  \begin{itemize}
				                  \item \textit{Fast}
				                  \item \textit{Easy to implement}
				                  \item \textit{Easy to share the key}
			                  \end{itemize}
			            \item \textit{Disadvantages}
			                  \begin{itemize}
				                  \item \textit{Key Distribution is a problem}
				                  \item \textit{Key Management is a problem}
			                  \end{itemize}
		            \end{itemize}
		      \item \textit{Asymmetric Key Cryptography}
		            \begin{itemize}
			            \item \textit{Advantages}
			                  \begin{itemize}
				                  \item \textit{Easy to share the key}
				                  \item \textit{Easy to manage the key}
			                  \end{itemize}
			            \item \textit{Disadvantages}
			                  \begin{itemize}
				                  \item \textit{Slow}
				                  \item \textit{Difficult to implement}
			                  \end{itemize}
		            \end{itemize}
	      \end{enumerate}

	\item \textit{Write advantages and disadvantages of RSA algorithm.}\\
	
	      \textbf{Advantages}
	      \begin{itemize}
		      \item \textit{RSA is a public key algorithm} : The public key is available to everyone. The private key is kept secret. The public key is used for encryption and the private key is used for decryption.
		      \item \textit{RSA is a secure algorithm} : The security of RSA is based on the difficulty of factoring large integers. The security of RSA depends on the fact that the factoring of the product of two large prime numbers is difficult. This is to say that it is a very difficult problem to find the two prime factors of a large composite number, and therefore it makes RSA very secure.
		      \item \textit{RSA is a widely used algorithm} : RSA is used in many applications such as secure email, digital signatures, file encryption, etc.
	      \end{itemize}
	      \textbf{Disadvantages}
	      \begin{itemize}
		      \item \textit{RSA is a slow algorithm} : The RSA algorithm is slow because of the large number of multiplications and modular exponentiations that are required.
		      \item \textit{RSA is not suitable for bulk data encryption} : RSA is not suitable for bulk data encryption because of its slowness. It is suitable for encrypting small amounts of data.
	      \end{itemize}
\end{enumerate}

\end{document}