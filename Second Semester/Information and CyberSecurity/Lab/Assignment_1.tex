% This is a Basic Assignment Paper but with like Code and stuff allowed in it, there is also url, hyperlinks from contents included. 

\documentclass[11pt]{article}

% Preamble

\usepackage[margin=1in]{geometry}
\usepackage{amsfonts, amsmath, amssymb}
\usepackage{fancyhdr, float, graphicx}
\usepackage[utf8]{inputenc} % Required for inputting international characters
\usepackage[T1]{fontenc} % Output font encoding for international characters
\usepackage{fouriernc} % Use the New Century Schoolbook font
\usepackage[nottoc, notlot, notlof]{tocbibind}
\usepackage{listings}
\usepackage{xcolor}
\usepackage{blindtext}
\usepackage{hyperref}
\hypersetup{
    colorlinks=true,
    linkcolor=black,
    filecolor=magenta,      
    urlcolor=cyan,
    pdfpagemode=FullScreen,
    }

\definecolor{codegreen}{rgb}{0,0.6,0}
\definecolor{codegray}{rgb}{0.5,0.5,0.5}
\definecolor{codepurple}{rgb}{0.58,0,0.82}
\definecolor{backcolour}{rgb}{0.95,0.95,0.92}

\lstdefinestyle{mystyle}{
    backgroundcolor=\color{backcolour},   
    commentstyle=\color{codegreen},
    keywordstyle=\color{magenta},
    numberstyle=\tiny\color{codegray},
    stringstyle=\color{codepurple},
    basicstyle=\ttfamily\footnotesize,
    breakatwhitespace=false,         
    breaklines=true,                 
    captionpos=b,                    
    keepspaces=true,                 
    numbers=left,                    
    numbersep=5pt,                  
    showspaces=false,                
    showstringspaces=false,
    showtabs=false,                  
    tabsize=2
}

\lstset{style=mystyle}

% Header and Footer
\pagestyle{fancy}
\fancyhead{}
\fancyfoot{}
\fancyhead[L]{\textit{\Large{Information and Cycbersecurity - 2nd Year B. Tech}}}
%\fancyhead[R]{\textit{something}}
\fancyfoot[C]{\thepage}
\renewcommand{\footrulewidth}{1pt}



% Other Doc Editing
% \parindent 0ex
%\renewcommand{\baselinestretch}{1.5}

\begin{document}

\begin{titlepage}
	\centering

	%---------------------------NAMES-------------------------------

	\huge\textsc{
		MIT World Peace University
	}\\

	\vspace{0.75\baselineskip} % space after Uni Name

	\LARGE{
		Information and Cybersecurity\\
		Second Year B. Tech, Semester 1
	}

	\vfill % space after Sub Name

	%--------------------------TITLE-------------------------------

	\rule{\textwidth}{1.6pt}\vspace*{-\baselineskip}\vspace*{2pt}
	\rule{\textwidth}{0.6pt}
	\vspace{0.75\baselineskip} % Whitespace above the title



	\huge{\textsc{
			Classical Cryptographic Techniques\\
			\textit{"Ceasar Cipher"}
		}} \\



	\vspace{0.5\baselineskip} % Whitespace below the title
	\rule{\textwidth}{0.6pt}\vspace*{-\baselineskip}\vspace*{2.8pt}
	\rule{\textwidth}{1.6pt}

	\vspace{1\baselineskip} % Whitespace after the title block

	%--------------------------SUBTITLE --------------------------	

	\LARGE\textsc{
		Lab Assignment 1
	} % Subtitle or further description
	\vfill

	%--------------------------AUTHOR-------------------------------

	Prepared By
	\vspace{0.5\baselineskip} % Whitespace before the editors

	\Large{
		Krishnaraj Thadesar \\
		Cyber Security and Forensics\\
		Batch A1, PA 20
	}


	\vspace{0.5\baselineskip} % Whitespace below the editor list
	\today

\end{titlepage}


\tableofcontents
\thispagestyle{empty}
\clearpage

\setcounter{page}{1}

\section{Aim}
Write a program using JAVA or Python or C++ to implement any classical cryptographic
technique.

\section{Objectives}
To conceal the context of some message from all accept the sender and recipient.

\section{Theory}

\subsection{Cryptography}

Cryptography is the practice and study of techniques for secure communication in the presence of third parties called adversaries. More generally, cryptography is about constructing and analyzing protocols that prevent third parties or the public from reading private messages; various aspects in information security such as data confidentiality, data integrity, authentication, and non-repudiation are central to modern cryptography. \\

\subsection{Types of Cryptography}

\subsubsection{Classical Cryptography}

Classical cryptography is the study of cryptography before the advent of modern computers. The classical ciphers are those ciphers that were in use before the advent of computers.

Some Classical Cryptographic Techniques are:

\begin{enumerate}
	\item \textit{Caesar Cipher} - A Caesar cipher is a type of substitution cipher in which each letter in the plaintext is replaced by a letter some fixed number of positions down the alphabet. For example, with a left shift of 3, D would be replaced by A, E would become B, and so on. The method is named after Julius Caesar, who used it in his private correspondence.

	      \begin{verbatim}
		Example:
		Plain Text: ABCDEFGHIJKLMNOPQRSTUVWXYZ
		Key: 3
		Cipher Text: DEFGHIJKLMNOPQRSTUVWXYZABC
	\end{verbatim}
	\item \textit{Vigenere Cipher} - A Vigenère cipher is a method of encrypting alphabetic text by using a series of interwoven Caesar ciphers, based on the letters of a keyword. It employs a form of polyalphabetic substitution.

	      \begin{verbatim}
		Example:
		Input : Plaintext :   GEEKSFORGEEKS
		Keyword :  AYUSH
		Output : Ciphertext :  GCYCZFMLYLEIM
	\end{verbatim}

	\item \textit{Hill Cipher} - The Hill cipher is a polygraphic substitution cipher based on linear algebra. Invented by Lester S. Hill in 1929, it was the first polygraphic cipher in which it was practical (though barely) to operate on more than three symbols at once. The matrix used for encryption is known as a key matrix.
	      \begin{verbatim}
		Example of Hill Cipher:
		Input  : Plaintext: ACT
		Key: GYBNQKURP
		Output : Ciphertext: POH

	\end{verbatim}
	\item \textit{Playfair Cipher} - The Playfair cipher is a manual symmetric encryption technique and was the first literal digram substitution cipher. The scheme was invented in 1854 by Charles Wheatstone, but was named after Lord Playfair for promoting its use.

	      \begin{verbatim}
		Example:
		Key text: Monarchy
		Plain text: instruments
		Cipher text: gatlmzclrqtx
	\end{verbatim}


	\item \textit{Affine Cipher} - In cryptography, an affine cipher is a type of monoalphabetic substitution cipher, wherein each letter in an alphabet is mapped to its numeric equivalent, encrypted using a simple mathematical function, and converted back to a letter. The formula used means that each letter encrypts to one other letter, and back again, meaning the cipher is essentially a standard substitution cipher with a rule governing which letter goes to which. As such, it has the weaknesses of all substitution ciphers. Each letter is enciphered with the function (ax + b) mod 26, where b is the magnitude of the shift.

	      \begin{verbatim}
		Example:
		Encrypted Message is : UBBAHK CAPJKX
		Decrypted Message is: AFFINE CIPHER
		Key: a = 17, b = 20
	\end{verbatim}

	\item \textit{DES} - The Data Encryption Standard (DES) is a symmetric-key algorithm for the encryption of electronic data. Although its short key length of 56 bits makes it too insecure for most current applications, it has been highly influential in the advancement of modern cryptography.
	\item \textit{AES} - The Advanced Encryption Standard (AES), also known by its original name Rijndael is a specification for the encryption of electronic data established by the U.S. National Institute of Standards and Technology (NIST) in 2001. AES is based on the Rijndael cipher developed by two Belgian cryptographers, Joan Daemen and Vincent Rijmen, who submitted a proposal to NIST during the AES selection process. AES is a subset of Rijndael designed by Daemen and Rijmen to be easy to implement in hardware and software. The AES standard has been adopted by the U.S. government and is now used worldwide to protect classified and sensitive but unclassified information.
	\item \textit{Substitution Ciphers} - In cryptography, a substitution cipher is a method of encoding by which units of plaintext are replaced with ciphertext, according to a fixed system; the "units" may be single letters (the most common), pairs of letters, triplets of letters, mixtures of the above, and so forth. The receiver deciphers the text by performing an inverse substitution.\\

	      A substitution cipher is a method of encoding by which units of plaintext are replaced with ciphertext, according to a fixed system; the "units" may be single letters (the most common), pairs of letters, triplets of letters, mixtures of the above, and so forth. The receiver deciphers the text by performing an inverse substitution.

	\item \textit{Transposition Ciphers} - In cryptography, a transposition cipher is a method of encryption by which the positions held by units of plaintext (which are commonly characters or groups of characters) are shifted according to a regular system, so that the ciphertext constitutes a permutation of the plaintext. That is, the plaintext is written out in a certain order, then the ciphertext is formed by reading down the columns going left to right.\\

\end{enumerate}

\subsubsection{Modern Cryptography}

Modern cryptography exists at the intersection of the disciplines of mathematics, computer science, electrical engineering, communication science, and physics. Applications of cryptography include electronic commerce, chip-based payment cards, digital currencies, computer passwords, and military communications.

Some Modern Cryptographic Techniques are:

\begin{enumerate}
	\item \textit{Diffie-Hellman Key Exchange} - In cryptography, the Diffie Hellman key exchange is a method of securely exchanging cryptographic keys over a public channel and was one of the first public-key protocols as originally conceived by Whitfield Diffie and Martin Hellman. It is one of the earliest practical examples of public key exchange implemented within the field of cryptography. It was the first public key exchange protocol that did not rely on trusted third parties and was the first public key protocol for which polynomial-time algorithms were found. It is also the first public key protocol for which a practical attack was found.
	\item \textit{ElGamal Encryption} - In cryptography, ElGamal encryption is an asymmetric key encryption algorithm based on the Diffie Hellman key exchange. It is named after Taher ElGamal, who published it in 1985. ElGamal encryption is a public-key encryption scheme, meaning that a pair of keys, one public and one private, are used. The public key may be known to everyone, while the private key is kept secret. The scheme is based on the difficulty of the discrete logarithm problem in a finite field.
	\item \textit{RSA} - RSA is an algorithm used by modern computers to encrypt and decrypt messages. It is an asymmetric cryptographic algorithm. Asymmetric means that there are two different keys. This is also called public key cryptography, because one of the keys can be given to anyone. The other key must be kept private. The algorithm is based on the fact that finding the factors of a large composite number is difficult: when the factors are prime numbers, the problem is called prime factorization. It is also a key pair (public and private) generator.
	\item \textit{Digital Signature} - A digital signature is a mathematical scheme for verifying the authenticity of digital messages or documents. A valid digital signature gives a recipient reason to believe that the message was created by a known sender, the message has not been altered in transit, and the sender cannot deny having sent the message. Digital signatures are a common method of authentication and data integrity in computer systems and communications.
	\item \textit{Hashing} - In cryptography, a hash function is any function that can be used to map data of arbitrary size to data of fixed size. The values returned by a hash function are called hash values, hash codes

\end{enumerate}

\section{Platform}
\textbf{Operating System}: Arch Linux x86-64 \\
\textbf{IDEs or Text Editors Used}: Visual Studio Code\\
\textbf{Compilers or Interpreters} : Python 3.10.1\\

\section{Pseudo Code or Algorithm}
\begin{lstlisting}[language=Python]
// Pseudo Code for Ceasar Cipher
// Input : String, Key
// Output : Ciphered String
// Function to Cipher the String

def Cipher(String, Key):
	Ciphered_String = ""
	for i in String:
		if i.isupper():
			Ciphered_String += chr((ord(i) + Key - 65) % 26 + 65)
		else:
			Ciphered_String += chr((ord(i) + Key - 97) % 26 + 97)
	return Ciphered_String

\end{lstlisting}
\section{Input and Output}
\begin{verbatim}
	Welcome to Assignment 1 in Information and CyberSecurity, working with Ceasar Ciphers. 
	Enter the string that you want to Cipher. 
	Assignments of Information and Cybersecurity
	Enter the key : 9
	Applying the Ceasar Cipher to it. 
	jBBRPWVNWCB XO rWOXAVJCRXW JWM lHKNABNLDARCH
\end{verbatim}
\section{Code}
\lstinputlisting[language=Python, caption="Ceasar Cipher"]{../Programs/Assignment_1/Ceasar_cipher.py}

\lstinputlisting[language=Python,
	caption="Rail Transportation Cipher"]
{../Programs/Assignment_1/Rail_transportation.py}
\section{Conclusion}
Thus, learnt about the different kinds of ciphers, classical cryptographic techniques, and how to implement some of them in python.
\clearpage

\section{FAQ}

\begin{enumerate}
	\item \textbf{What are various classical ciphers?}\\
	      Answer:
	      There are many different kinds of ciphers, some of them are:
	      \begin{enumerate}
		      \item \textit{Caesar Cipher}
		      \item  \textit{Vigenere Cipher}
		      \item  \textit{Rail Transportation Cipher}
		      \item  \textit{Hill Cipher}
		      \item  \textit{Playfair Cipher}
		      \item  \textit{Autokey Cipher}
		      \item  \textit{Columnar Transposition Cipher}
		      \item  \textit{Affine Cipher}
		      \item  \textit{Monoalphabetic Cipher}
		      \item  \textit{Polyalphabetic Cipher}
		      \item  \textit{Transposition Cipher}
		      \item  \textit{Substitution Cipher}
	      \end{enumerate}

	\item \textbf{Compare steganography and Cryptography}

	      Answer: \textbf{Steganography} is the practice of concealing a file, message, image, or video within another file, message, image, or video.\\

	      \textbf{Cryptography} is the practice and study of techniques for secure communication in the presence of third parties called adversaries. More generally, cryptography is about constructing and analyzing protocols that prevent third parties or the public from reading private messages; various aspects in information security such as data confidentiality, data integrity, authentication, and non-repudiation are central to modern cryptography. Modern cryptography exists at the intersection of the disciplines of mathematics, computer science, electrical engineering, communication science, and physics. Applications of cryptography include electronic commerce, chip-based payment cards, digital currencies, computer passwords, and military communications.\\

	      Steganogrphy is a type of cryptography.

	      Steganography comes from the words steganos (meaning covered or hidden) and graphein (meaning writing). It is the practice of concealing a file, message, image, or video within another file, message, image, or video.\\

	      Cryptography however, comes from the Greek words kryptos (meaning hidden) and graphein (meaning writing). It is the practice and study of techniques for secure communication in the presence of third parties called adversaries.\\

	\item \textbf{What are the few major applications of cryptography in the modern world?}\\
	      Answer:
	      \begin{enumerate}
		      \item \textit{Electronic Commerce} - Cryptography is used to protect the privacy of credit card numbers, bank account numbers, and other sensitive information exchanged over the Internet.
		      \item \textit{Chip-based Payment Cards} - Cryptography is used to protect the privacy of credit card numbers, bank account numbers, and other sensitive information exchanged over the Internet.
		      \item \textit{Digital Currencies} - Cryptography is used to protect the privacy of credit card numbers, bank account numbers, and other sensitive information exchanged over the Internet.
		      \item \textit{Computer Passwords} - Cryptography is used to protect the privacy of credit card numbers, bank account numbers, and other sensitive information exchanged over the Internet.
		      \item \textit{Military Communications} - Cryptography is used to protect the privacy of credit card numbers, bank account numbers, and other sensitive information exchanged over the Internet.
	      \end{enumerate}

	\item \textbf{How can Caesar cipher be cracked?}\\

	      Answer:
	      \begin{enumerate}
		      \item \textit{Frequency Analysis} - Frequency analysis is a method of analyzing the frequency of each letter in a message. The frequency of each letter is compared to the frequency of letters in the English language. If the frequency of a letter in the message is close to the frequency of that letter in the English language, then it is likely that the letter is an English letter. This method is not very effective because it is possible to create a message that has the same frequency of letters as the English language.
		      \item \textit{Brute Force} - Brute force is a method of trying every possible key until the correct key is found. This method is very effective, but it is very time consuming.
		      \item \textit{Cryptanalysis} - Cryptanalysis is a method of analyzing the cipher to find a weakness in the cipher. This method is very effective, but it is very difficult to find a weakness in a cipher.

	      \end{enumerate}
\end{enumerate}


\end{document}