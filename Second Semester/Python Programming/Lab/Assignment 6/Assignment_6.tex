% This is a Basic Assignment Paper but with like Code and stuff allowed in it, there is also url, hyperlinks from contents included. 

\documentclass[11pt]{article}

% Preamble

\usepackage[margin=1in]{geometry}
\usepackage{amsfonts, amsmath, amssymb}
\usepackage{fancyhdr, float, graphicx}
\usepackage[utf8]{inputenc} % Required for inputting international characters
\usepackage[T1]{fontenc} % Output font encoding for international characters
\usepackage{fouriernc} % Use the New Century Schoolbook font
\usepackage[nottoc, notlot, notlof]{tocbibind}
\usepackage{listings}
\usepackage{xcolor}
\usepackage{blindtext}
\usepackage{hyperref}


\definecolor{codegreen}{rgb}{0,0.6,0}
\definecolor{codegray}{rgb}{0.5,0.5,0.5}
\definecolor{codepurple}{rgb}{0.58,0,0.82}
\definecolor{backcolour}{rgb}{0.95,0.95,0.92}

\lstdefinestyle{mystyle}{
    backgroundcolor=\color{backcolour},   
    commentstyle=\color{codegreen},
    keywordstyle=\color{magenta},
    numberstyle=\tiny\color{codegray},
    stringstyle=\color{codepurple},
    basicstyle=\ttfamily\footnotesize,
    breakatwhitespace=false,         
    breaklines=true,                 
    captionpos=b,                    
    keepspaces=true,                 
    numbers=left,                    
    numbersep=5pt,                  
    showspaces=false,                
    showstringspaces=false,
    showtabs=false,                  
    tabsize=2
}

\lstset{style=mystyle}

% Header and Footer
\pagestyle{fancy}
\fancyhead{}
\fancyfoot{}
\fancyhead[L]{\textit{\Large{Python Programming Assignment 6}}}
%\fancyhead[R]{\textit{something}}
\fancyfoot[C]{\thepage}
\renewcommand{\footrulewidth}{1pt}

\usepackage[breakable]{tcolorbox}
\usepackage{parskip} % Stop auto-indenting (to mimic markdown behaviour)


% Basic figure setup, for now with no caption control since it's done
% automatically by Pandoc (which extracts ![](path) syntax from Markdown).
\usepackage{graphicx}
% Maintain compatibility with old templates. Remove in nbconvert 6.0
\let\Oldincludegraphics\includegraphics
% Ensure that by default, figures have no caption (until we provide a
% proper Figure object with a Caption API and a way to capture that
% in the conversion process - todo).
\usepackage{caption}
\DeclareCaptionFormat{nocaption}{}
\captionsetup{format=nocaption,aboveskip=0pt,belowskip=0pt}

\usepackage{float}
\floatplacement{figure}{H} % forces figures to be placed at the correct location
\usepackage{xcolor} % Allow colors to be defined
\usepackage{enumerate} % Needed for markdown enumerations to work
\usepackage{geometry} % Used to adjust the document margins
\usepackage{amsmath} % Equations
\usepackage{amssymb} % Equations
\usepackage{textcomp} % defines textquotesingle
% Hack from http://tex.stackexchange.com/a/47451/13684:
\AtBeginDocument{%
	\def\PYZsq{\textquotesingle}% Upright quotes in Pygmentized code
}
\usepackage{upquote} % Upright quotes for verbatim code
\usepackage{eurosym} % defines \euro

\usepackage{iftex}
% \ifPDFTeX
% 	\usepackage[T1]{fontenc}
% 	\IfFileExists{alphabeta.sty}{
% 		  \usepackage{alphabeta}
% 	  }{
% 		  \usepackage[mathletters]{ucs}
% 		  \usepackage[utf8x]{inputenc}
% 	  }
% \else
% 	\usepackage{fontspec}
% 	\usepackage{unicode-math}
% \fi

\usepackage{fancyvrb} % verbatim replacement that allows latex
\usepackage{grffile} % extends the file name processing of package graphics
					 % to support a larger range
\makeatletter % fix for old versions of grffile with XeLaTeX
\@ifpackagelater{grffile}{2019/11/01}
{
  % Do nothing on new versions
}
{
  \def\Gread@@xetex#1{%
	\IfFileExists{"\Gin@base".bb}%
	{\Gread@eps{\Gin@base.bb}}%
	{\Gread@@xetex@aux#1}%
  }
}
\makeatother
\usepackage[Export]{adjustbox} % Used to constrain images to a maximum size
\adjustboxset{max size={0.9\linewidth}{0.9\paperheight}}

% The hyperref package gives us a pdf with properly built
% internal navigation ('pdf bookmarks' for the table of contents,
% internal cross-reference links, web links for URLs, etc.)
\usepackage{hyperref}
% The default LaTeX title has an obnoxious amount of whitespace. By default,
% titling removes some of it. It also provides customization options.
\usepackage{titling}
\usepackage{longtable} % longtable support required by pandoc >1.10
\usepackage{booktabs}  % table support for pandoc > 1.12.2
\usepackage{array}     % table support for pandoc >= 2.11.3
\usepackage{calc}      % table minipage width calculation for pandoc >= 2.11.1
\usepackage[inline]{enumitem} % IRkernel/repr support (it uses the enumerate* environment)
\usepackage[normalem]{ulem} % ulem is needed to support strikethroughs (\sout)
							% normalem makes italics be italics, not underlines
\usepackage{mathrsfs}



% Colors for the hyperref package
\definecolor{urlcolor}{rgb}{0,.145,.698}
\definecolor{linkcolor}{rgb}{.71,0.21,0.01}
\definecolor{citecolor}{rgb}{.12,.54,.11}

% ANSI colors
\definecolor{ansi-black}{HTML}{3E424D}
\definecolor{ansi-black-intense}{HTML}{282C36}
\definecolor{ansi-red}{HTML}{E75C58}
\definecolor{ansi-red-intense}{HTML}{B22B31}
\definecolor{ansi-green}{HTML}{00A250}
\definecolor{ansi-green-intense}{HTML}{007427}
\definecolor{ansi-yellow}{HTML}{DDB62B}
\definecolor{ansi-yellow-intense}{HTML}{B27D12}
\definecolor{ansi-blue}{HTML}{208FFB}
\definecolor{ansi-blue-intense}{HTML}{0065CA}
\definecolor{ansi-magenta}{HTML}{D160C4}
\definecolor{ansi-magenta-intense}{HTML}{A03196}
\definecolor{ansi-cyan}{HTML}{60C6C8}
\definecolor{ansi-cyan-intense}{HTML}{258F8F}
\definecolor{ansi-white}{HTML}{C5C1B4}
\definecolor{ansi-white-intense}{HTML}{A1A6B2}
\definecolor{ansi-default-inverse-fg}{HTML}{FFFFFF}
\definecolor{ansi-default-inverse-bg}{HTML}{000000}

% common color for the border for error outputs.
\definecolor{outerrorbackground}{HTML}{FFDFDF}

% commands and environments needed by pandoc snippets
% extracted from the output of `pandoc -s`
\providecommand{\tightlist}{%
  \setlength{\itemsep}{0pt}\setlength{\parskip}{0pt}}
\DefineVerbatimEnvironment{Highlighting}{Verbatim}{commandchars=\\\{\}}
% Add ',fontsize=\small' for more characters per line
\newenvironment{Shaded}{}{}
\newcommand{\KeywordTok}[1]{\textcolor[rgb]{0.00,0.44,0.13}{\textbf{{#1}}}}
\newcommand{\DataTypeTok}[1]{\textcolor[rgb]{0.56,0.13,0.00}{{#1}}}
\newcommand{\DecValTok}[1]{\textcolor[rgb]{0.25,0.63,0.44}{{#1}}}
\newcommand{\BaseNTok}[1]{\textcolor[rgb]{0.25,0.63,0.44}{{#1}}}
\newcommand{\FloatTok}[1]{\textcolor[rgb]{0.25,0.63,0.44}{{#1}}}
\newcommand{\CharTok}[1]{\textcolor[rgb]{0.25,0.44,0.63}{{#1}}}
\newcommand{\StringTok}[1]{\textcolor[rgb]{0.25,0.44,0.63}{{#1}}}
\newcommand{\CommentTok}[1]{\textcolor[rgb]{0.38,0.63,0.69}{\textit{{#1}}}}
\newcommand{\OtherTok}[1]{\textcolor[rgb]{0.00,0.44,0.13}{{#1}}}
\newcommand{\AlertTok}[1]{\textcolor[rgb]{1.00,0.00,0.00}{\textbf{{#1}}}}
\newcommand{\FunctionTok}[1]{\textcolor[rgb]{0.02,0.16,0.49}{{#1}}}
\newcommand{\RegionMarkerTok}[1]{{#1}}
\newcommand{\ErrorTok}[1]{\textcolor[rgb]{1.00,0.00,0.00}{\textbf{{#1}}}}
\newcommand{\NormalTok}[1]{{#1}}

% Additional commands for more recent versions of Pandoc
\newcommand{\ConstantTok}[1]{\textcolor[rgb]{0.53,0.00,0.00}{{#1}}}
\newcommand{\SpecialCharTok}[1]{\textcolor[rgb]{0.25,0.44,0.63}{{#1}}}
\newcommand{\VerbatimStringTok}[1]{\textcolor[rgb]{0.25,0.44,0.63}{{#1}}}
\newcommand{\SpecialStringTok}[1]{\textcolor[rgb]{0.73,0.40,0.53}{{#1}}}
\newcommand{\ImportTok}[1]{{#1}}
\newcommand{\DocumentationTok}[1]{\textcolor[rgb]{0.73,0.13,0.13}{\textit{{#1}}}}
\newcommand{\AnnotationTok}[1]{\textcolor[rgb]{0.38,0.63,0.69}{\textbf{\textit{{#1}}}}}
\newcommand{\CommentVarTok}[1]{\textcolor[rgb]{0.38,0.63,0.69}{\textbf{\textit{{#1}}}}}
\newcommand{\VariableTok}[1]{\textcolor[rgb]{0.10,0.09,0.49}{{#1}}}
\newcommand{\ControlFlowTok}[1]{\textcolor[rgb]{0.00,0.44,0.13}{\textbf{{#1}}}}
\newcommand{\OperatorTok}[1]{\textcolor[rgb]{0.40,0.40,0.40}{{#1}}}
\newcommand{\BuiltInTok}[1]{{#1}}
\newcommand{\ExtensionTok}[1]{{#1}}
\newcommand{\PreprocessorTok}[1]{\textcolor[rgb]{0.74,0.48,0.00}{{#1}}}
\newcommand{\AttributeTok}[1]{\textcolor[rgb]{0.49,0.56,0.16}{{#1}}}
\newcommand{\InformationTok}[1]{\textcolor[rgb]{0.38,0.63,0.69}{\textbf{\textit{{#1}}}}}
\newcommand{\WarningTok}[1]{\textcolor[rgb]{0.38,0.63,0.69}{\textbf{\textit{{#1}}}}}


% Define a nice break command that doesn't care if a line doesn't already
% exist.
\def\br{\hspace*{\fill} \\* }
% Math Jax compatibility definitions
\def\gt{>}
\def\lt{<}
\let\Oldtex\TeX
\let\Oldlatex\LaTeX
\renewcommand{\TeX}{\textrm{\Oldtex}}
\renewcommand{\LaTeX}{\textrm{\Oldlatex}}
% Document parameters
% Document title
\title{Assignment\_1}





% Pygments definitions
\makeatletter
\def\PY@reset{\let\PY@it=\relax \let\PY@bf=\relax%
\let\PY@ul=\relax \let\PY@tc=\relax%
\let\PY@bc=\relax \let\PY@ff=\relax}
\def\PY@tok#1{\csname PY@tok@#1\endcsname}
\def\PY@toks#1+{\ifx\relax#1\empty\else%
\PY@tok{#1}\expandafter\PY@toks\fi}
\def\PY@do#1{\PY@bc{\PY@tc{\PY@ul{%
\PY@it{\PY@bf{\PY@ff{#1}}}}}}}
\def\PY#1#2{\PY@reset\PY@toks#1+\relax+\PY@do{#2}}

\@namedef{PY@tok@w}{\def\PY@tc##1{\textcolor[rgb]{0.73,0.73,0.73}{##1}}}
\@namedef{PY@tok@c}{\let\PY@it=\textit\def\PY@tc##1{\textcolor[rgb]{0.24,0.48,0.48}{##1}}}
\@namedef{PY@tok@cp}{\def\PY@tc##1{\textcolor[rgb]{0.61,0.40,0.00}{##1}}}
\@namedef{PY@tok@k}{\let\PY@bf=\textbf\def\PY@tc##1{\textcolor[rgb]{0.00,0.50,0.00}{##1}}}
\@namedef{PY@tok@kp}{\def\PY@tc##1{\textcolor[rgb]{0.00,0.50,0.00}{##1}}}
\@namedef{PY@tok@kt}{\def\PY@tc##1{\textcolor[rgb]{0.69,0.00,0.25}{##1}}}
\@namedef{PY@tok@o}{\def\PY@tc##1{\textcolor[rgb]{0.40,0.40,0.40}{##1}}}
\@namedef{PY@tok@ow}{\let\PY@bf=\textbf\def\PY@tc##1{\textcolor[rgb]{0.67,0.13,1.00}{##1}}}
\@namedef{PY@tok@nb}{\def\PY@tc##1{\textcolor[rgb]{0.00,0.50,0.00}{##1}}}
\@namedef{PY@tok@nf}{\def\PY@tc##1{\textcolor[rgb]{0.00,0.00,1.00}{##1}}}
\@namedef{PY@tok@nc}{\let\PY@bf=\textbf\def\PY@tc##1{\textcolor[rgb]{0.00,0.00,1.00}{##1}}}
\@namedef{PY@tok@nn}{\let\PY@bf=\textbf\def\PY@tc##1{\textcolor[rgb]{0.00,0.00,1.00}{##1}}}
\@namedef{PY@tok@ne}{\let\PY@bf=\textbf\def\PY@tc##1{\textcolor[rgb]{0.80,0.25,0.22}{##1}}}
\@namedef{PY@tok@nv}{\def\PY@tc##1{\textcolor[rgb]{0.10,0.09,0.49}{##1}}}
\@namedef{PY@tok@no}{\def\PY@tc##1{\textcolor[rgb]{0.53,0.00,0.00}{##1}}}
\@namedef{PY@tok@nl}{\def\PY@tc##1{\textcolor[rgb]{0.46,0.46,0.00}{##1}}}
\@namedef{PY@tok@ni}{\let\PY@bf=\textbf\def\PY@tc##1{\textcolor[rgb]{0.44,0.44,0.44}{##1}}}
\@namedef{PY@tok@na}{\def\PY@tc##1{\textcolor[rgb]{0.41,0.47,0.13}{##1}}}
\@namedef{PY@tok@nt}{\let\PY@bf=\textbf\def\PY@tc##1{\textcolor[rgb]{0.00,0.50,0.00}{##1}}}
\@namedef{PY@tok@nd}{\def\PY@tc##1{\textcolor[rgb]{0.67,0.13,1.00}{##1}}}
\@namedef{PY@tok@s}{\def\PY@tc##1{\textcolor[rgb]{0.73,0.13,0.13}{##1}}}
\@namedef{PY@tok@sd}{\let\PY@it=\textit\def\PY@tc##1{\textcolor[rgb]{0.73,0.13,0.13}{##1}}}
\@namedef{PY@tok@si}{\let\PY@bf=\textbf\def\PY@tc##1{\textcolor[rgb]{0.64,0.35,0.47}{##1}}}
\@namedef{PY@tok@se}{\let\PY@bf=\textbf\def\PY@tc##1{\textcolor[rgb]{0.67,0.36,0.12}{##1}}}
\@namedef{PY@tok@sr}{\def\PY@tc##1{\textcolor[rgb]{0.64,0.35,0.47}{##1}}}
\@namedef{PY@tok@ss}{\def\PY@tc##1{\textcolor[rgb]{0.10,0.09,0.49}{##1}}}
\@namedef{PY@tok@sx}{\def\PY@tc##1{\textcolor[rgb]{0.00,0.50,0.00}{##1}}}
\@namedef{PY@tok@m}{\def\PY@tc##1{\textcolor[rgb]{0.40,0.40,0.40}{##1}}}
\@namedef{PY@tok@gh}{\let\PY@bf=\textbf\def\PY@tc##1{\textcolor[rgb]{0.00,0.00,0.50}{##1}}}
\@namedef{PY@tok@gu}{\let\PY@bf=\textbf\def\PY@tc##1{\textcolor[rgb]{0.50,0.00,0.50}{##1}}}
\@namedef{PY@tok@gd}{\def\PY@tc##1{\textcolor[rgb]{0.63,0.00,0.00}{##1}}}
\@namedef{PY@tok@gi}{\def\PY@tc##1{\textcolor[rgb]{0.00,0.52,0.00}{##1}}}
\@namedef{PY@tok@gr}{\def\PY@tc##1{\textcolor[rgb]{0.89,0.00,0.00}{##1}}}
\@namedef{PY@tok@ge}{\let\PY@it=\textit}
\@namedef{PY@tok@gs}{\let\PY@bf=\textbf}
\@namedef{PY@tok@gp}{\let\PY@bf=\textbf\def\PY@tc##1{\textcolor[rgb]{0.00,0.00,0.50}{##1}}}
\@namedef{PY@tok@go}{\def\PY@tc##1{\textcolor[rgb]{0.44,0.44,0.44}{##1}}}
\@namedef{PY@tok@gt}{\def\PY@tc##1{\textcolor[rgb]{0.00,0.27,0.87}{##1}}}
\@namedef{PY@tok@err}{\def\PY@bc##1{{\setlength{\fboxsep}{\string -\fboxrule}\fcolorbox[rgb]{1.00,0.00,0.00}{1,1,1}{\strut ##1}}}}
\@namedef{PY@tok@kc}{\let\PY@bf=\textbf\def\PY@tc##1{\textcolor[rgb]{0.00,0.50,0.00}{##1}}}
\@namedef{PY@tok@kd}{\let\PY@bf=\textbf\def\PY@tc##1{\textcolor[rgb]{0.00,0.50,0.00}{##1}}}
\@namedef{PY@tok@kn}{\let\PY@bf=\textbf\def\PY@tc##1{\textcolor[rgb]{0.00,0.50,0.00}{##1}}}
\@namedef{PY@tok@kr}{\let\PY@bf=\textbf\def\PY@tc##1{\textcolor[rgb]{0.00,0.50,0.00}{##1}}}
\@namedef{PY@tok@bp}{\def\PY@tc##1{\textcolor[rgb]{0.00,0.50,0.00}{##1}}}
\@namedef{PY@tok@fm}{\def\PY@tc##1{\textcolor[rgb]{0.00,0.00,1.00}{##1}}}
\@namedef{PY@tok@vc}{\def\PY@tc##1{\textcolor[rgb]{0.10,0.09,0.49}{##1}}}
\@namedef{PY@tok@vg}{\def\PY@tc##1{\textcolor[rgb]{0.10,0.09,0.49}{##1}}}
\@namedef{PY@tok@vi}{\def\PY@tc##1{\textcolor[rgb]{0.10,0.09,0.49}{##1}}}
\@namedef{PY@tok@vm}{\def\PY@tc##1{\textcolor[rgb]{0.10,0.09,0.49}{##1}}}
\@namedef{PY@tok@sa}{\def\PY@tc##1{\textcolor[rgb]{0.73,0.13,0.13}{##1}}}
\@namedef{PY@tok@sb}{\def\PY@tc##1{\textcolor[rgb]{0.73,0.13,0.13}{##1}}}
\@namedef{PY@tok@sc}{\def\PY@tc##1{\textcolor[rgb]{0.73,0.13,0.13}{##1}}}
\@namedef{PY@tok@dl}{\def\PY@tc##1{\textcolor[rgb]{0.73,0.13,0.13}{##1}}}
\@namedef{PY@tok@s2}{\def\PY@tc##1{\textcolor[rgb]{0.73,0.13,0.13}{##1}}}
\@namedef{PY@tok@sh}{\def\PY@tc##1{\textcolor[rgb]{0.73,0.13,0.13}{##1}}}
\@namedef{PY@tok@s1}{\def\PY@tc##1{\textcolor[rgb]{0.73,0.13,0.13}{##1}}}
\@namedef{PY@tok@mb}{\def\PY@tc##1{\textcolor[rgb]{0.40,0.40,0.40}{##1}}}
\@namedef{PY@tok@mf}{\def\PY@tc##1{\textcolor[rgb]{0.40,0.40,0.40}{##1}}}
\@namedef{PY@tok@mh}{\def\PY@tc##1{\textcolor[rgb]{0.40,0.40,0.40}{##1}}}
\@namedef{PY@tok@mi}{\def\PY@tc##1{\textcolor[rgb]{0.40,0.40,0.40}{##1}}}
\@namedef{PY@tok@il}{\def\PY@tc##1{\textcolor[rgb]{0.40,0.40,0.40}{##1}}}
\@namedef{PY@tok@mo}{\def\PY@tc##1{\textcolor[rgb]{0.40,0.40,0.40}{##1}}}
\@namedef{PY@tok@ch}{\let\PY@it=\textit\def\PY@tc##1{\textcolor[rgb]{0.24,0.48,0.48}{##1}}}
\@namedef{PY@tok@cm}{\let\PY@it=\textit\def\PY@tc##1{\textcolor[rgb]{0.24,0.48,0.48}{##1}}}
\@namedef{PY@tok@cpf}{\let\PY@it=\textit\def\PY@tc##1{\textcolor[rgb]{0.24,0.48,0.48}{##1}}}
\@namedef{PY@tok@c1}{\let\PY@it=\textit\def\PY@tc##1{\textcolor[rgb]{0.24,0.48,0.48}{##1}}}
\@namedef{PY@tok@cs}{\let\PY@it=\textit\def\PY@tc##1{\textcolor[rgb]{0.24,0.48,0.48}{##1}}}

\def\PYZbs{\char`\\}
\def\PYZus{\char`\_}
\def\PYZob{\char`\{}
\def\PYZcb{\char`\}}
\def\PYZca{\char`\^}
\def\PYZam{\char`\&}
\def\PYZlt{\char`\<}
\def\PYZgt{\char`\>}
\def\PYZsh{\char`\#}
\def\PYZpc{\char`\%}
\def\PYZdl{\char`\$}
\def\PYZhy{\char`\-}
\def\PYZsq{\char`\'}
\def\PYZdq{\char`\"}
\def\PYZti{\char`\~}
% for compatibility with earlier versions
\def\PYZat{@}
\def\PYZlb{[}
\def\PYZrb{]}
\makeatother


% For linebreaks inside Verbatim environment from package fancyvrb.
\makeatletter
	\newbox\Wrappedcontinuationbox
	\newbox\Wrappedvisiblespacebox
	\newcommand*\Wrappedvisiblespace {\textcolor{red}{\textvisiblespace}}
	\newcommand*\Wrappedcontinuationsymbol {\textcolor{red}{\llap{\tiny$\m@th\hookrightarrow$}}}
	\newcommand*\Wrappedcontinuationindent {3ex }
	\newcommand*\Wrappedafterbreak {\kern\Wrappedcontinuationindent\copy\Wrappedcontinuationbox}
	% Take advantage of the already applied Pygments mark-up to insert
	% potential linebreaks for TeX processing.
	%        {, <, #, %, $, ' and ": go to next line.
	%        _, }, ^, &, >, - and ~: stay at end of broken line.
	% Use of \textquotesingle for straight quote.
	\newcommand*\Wrappedbreaksatspecials {%
		\def\PYGZus{\discretionary{\char`\_}{\Wrappedafterbreak}{\char`\_}}%
		\def\PYGZob{\discretionary{}{\Wrappedafterbreak\char`\{}{\char`\{}}%
		\def\PYGZcb{\discretionary{\char`\}}{\Wrappedafterbreak}{\char`\}}}%
		\def\PYGZca{\discretionary{\char`\^}{\Wrappedafterbreak}{\char`\^}}%
		\def\PYGZam{\discretionary{\char`\&}{\Wrappedafterbreak}{\char`\&}}%
		\def\PYGZlt{\discretionary{}{\Wrappedafterbreak\char`\<}{\char`\<}}%
		\def\PYGZgt{\discretionary{\char`\>}{\Wrappedafterbreak}{\char`\>}}%
		\def\PYGZsh{\discretionary{}{\Wrappedafterbreak\char`\#}{\char`\#}}%
		\def\PYGZpc{\discretionary{}{\Wrappedafterbreak\char`\%}{\char`\%}}%
		\def\PYGZdl{\discretionary{}{\Wrappedafterbreak\char`\$}{\char`\$}}%
		\def\PYGZhy{\discretionary{\char`\-}{\Wrappedafterbreak}{\char`\-}}%
		\def\PYGZsq{\discretionary{}{\Wrappedafterbreak\textquotesingle}{\textquotesingle}}%
		\def\PYGZdq{\discretionary{}{\Wrappedafterbreak\char`\"}{\char`\"}}%
		\def\PYGZti{\discretionary{\char`\~}{\Wrappedafterbreak}{\char`\~}}%
	}
	% Some characters . , ; ? ! / are not pygmentized.
	% This macro makes them "active" and they will insert potential linebreaks
	\newcommand*\Wrappedbreaksatpunct {%
		\lccode`\~`\.\lowercase{\def~}{\discretionary{\hbox{\char`\.}}{\Wrappedafterbreak}{\hbox{\char`\.}}}%
		\lccode`\~`\,\lowercase{\def~}{\discretionary{\hbox{\char`\,}}{\Wrappedafterbreak}{\hbox{\char`\,}}}%
		\lccode`\~`\;\lowercase{\def~}{\discretionary{\hbox{\char`\;}}{\Wrappedafterbreak}{\hbox{\char`\;}}}%
		\lccode`\~`\:\lowercase{\def~}{\discretionary{\hbox{\char`\:}}{\Wrappedafterbreak}{\hbox{\char`\:}}}%
		\lccode`\~`\?\lowercase{\def~}{\discretionary{\hbox{\char`\?}}{\Wrappedafterbreak}{\hbox{\char`\?}}}%
		\lccode`\~`\!\lowercase{\def~}{\discretionary{\hbox{\char`\!}}{\Wrappedafterbreak}{\hbox{\char`\!}}}%
		\lccode`\~`\/\lowercase{\def~}{\discretionary{\hbox{\char`\/}}{\Wrappedafterbreak}{\hbox{\char`\/}}}%
		\catcode`\.\active
		\catcode`\,\active
		\catcode`\;\active
		\catcode`\:\active
		\catcode`\?\active
		\catcode`\!\active
		\catcode`\/\active
		\lccode`\~`\~
	}
\makeatother

\let\OriginalVerbatim=\Verbatim
\makeatletter
\renewcommand{\Verbatim}[1][1]{%
	%\parskip\z@skip
	\sbox\Wrappedcontinuationbox {\Wrappedcontinuationsymbol}%
	\sbox\Wrappedvisiblespacebox {\FV@SetupFont\Wrappedvisiblespace}%
	\def\FancyVerbFormatLine ##1{\hsize\linewidth
		\vtop{\raggedright\hyphenpenalty\z@\exhyphenpenalty\z@
			\doublehyphendemerits\z@\finalhyphendemerits\z@
			\strut ##1\strut}%
	}%
	% If the linebreak is at a space, the latter will be displayed as visible
	% space at end of first line, and a continuation symbol starts next line.
	% Stretch/shrink are however usually zero for typewriter font.
	\def\FV@Space {%
		\nobreak\hskip\z@ plus\fontdimen3\font minus\fontdimen4\font
		\discretionary{\copy\Wrappedvisiblespacebox}{\Wrappedafterbreak}
		{\kern\fontdimen2\font}%
	}%

	% Allow breaks at special characters using \PYG... macros.
	\Wrappedbreaksatspecials
	% Breaks at punctuation characters . , ; ? ! and / need catcode=\active
	\OriginalVerbatim[#1,codes*=\Wrappedbreaksatpunct]%
}
\makeatother

% Exact colors from NB
\definecolor{incolor}{HTML}{303F9F}
\definecolor{outcolor}{HTML}{D84315}
\definecolor{cellborder}{HTML}{CFCFCF}
\definecolor{cellbackground}{HTML}{F7F7F7}

% prompt
\makeatletter
\newcommand{\boxspacing}{\kern\kvtcb@left@rule\kern\kvtcb@boxsep}
\makeatother
\newcommand{\prompt}[4]{
	{\ttfamily\llap{{\color{#2}[#3]:\hspace{3pt}#4}}\vspace{-\baselineskip}}
}



% Prevent overflowing lines due to hard-to-break entities
\sloppy
% Setup hyperref package
\hypersetup{
  breaklinks=true,  % so long urls are correctly broken across lines
  colorlinks=true,
  urlcolor=urlcolor,
  linkcolor=linkcolor,
  citecolor=citecolor,
  }
% Slightly bigger margins than the latex defaults

\geometry{verbose,tmargin=1in,bmargin=1in,lmargin=1in,rmargin=1in}
\hypersetup{
    colorlinks=true,
    linkcolor=black,
    filecolor=magenta,      
    urlcolor=cyan,
    pdfpagemode=FullScreen,
    }



% Other Doc Editing
% \parindent 0ex
%\renewcommand{\baselinestretch}{1.5}

\begin{document}

\begin{titlepage}
	\centering

	%---------------------------NAMES-------------------------------

	\huge\textsc{
		MIT World Peace University
	}\\

	\vspace{0.75\baselineskip} % space after Uni Name

	\LARGE{
		Python Programming\\
		Second Year B. Tech, Semester 4
	}

	\vfill % space after Sub Name

	%--------------------------TITLE-------------------------------

	\rule{\textwidth}{1.6pt}\vspace*{-\baselineskip}\vspace*{2pt}
	\rule{\textwidth}{0.6pt}
	\vspace{0.75\baselineskip} % Whitespace above the title



	\huge{\textsc{
		Learning the Basics of 
			\textit{Obeject Oriented Programming with Python}
		}} \\



	\vspace{0.5\baselineskip} % Whitespace below the title
	\rule{\textwidth}{0.6pt}\vspace*{-\baselineskip}\vspace*{2.8pt}
	\rule{\textwidth}{1.6pt}

	\vspace{1\baselineskip} % Whitespace after the title block

	%--------------------------SUBTITLE --------------------------	

	\LARGE\textsc{
		Assignment No. 6
	} % Subtitle or further description
	\vfill

	%--------------------------AUTHOR-------------------------------

	Prepared By
	\vspace{0.5\baselineskip} % Whitespace before the editors

	\Large{
		Krishnaraj Thadesar \\
		Cyber Security and Forensics\\
		Batch A1, PA 20
	}


	\vspace{0.5\baselineskip} % Whitespace below the editor list
	\today

\end{titlepage}

\tableofcontents
\thispagestyle{empty}
\clearpage

\setcounter{page}{1}

\section{Aim}
Write a program to read 3 subject marks and display pass or failed using class and object.

\section{Objectives}
\begin{enumerate}
	\item To learn and implement concepts of Object Oriented Programming in Python.
\end{enumerate}

\section{Problem Statement}
Object oriented programming concepts to display Students grade as Pass or Fail.

\section{Theory}
% 1. Explain Object oriented programming in Python.
% 2. Explain __init__ method.
% 3. Explain the concept of Inheritance and Polymorphism.

\subsection{Object Oriented Programming in Python}

\subsubsection{Definition}

\textit{Object-oriented programming (OOP) is a programming paradigm based on the concept of "objects", which can contain data, in the form of fields, often known as attributes; and code, in the form of procedures, often known as methods. A feature of objects is that an object's procedures can access and often modify the data fields of the object with which they are associated (objects have a notion of "this" or "self"). In OOP, computer programs are designed by making them out of objects that interact with one another. There is significant diversity of OOP languages, but the most popular ones are class-based, meaning that objects are instances of classes, which also determine their types.}

\subsubsection{Example}

\begin{lstlisting}[language=Python]
class Car:
    def __init__(self, make, model, year):
        self.make = make
        self.model = model
        self.year = year
        self.odometer_reading = 0

    def get_descriptive_name(self):
        long_name = f"{self.year} {self.make} {self.model}"
        return long_name.title()

    def read_odometer(self):
        print(f"This car has {self.odometer_reading} miles on it.")

    def update_odometer(self, mileage):
        if mileage >= self.odometer_reading:
            self.odometer_reading = mileage
        else:
            print("You can't roll back an odometer!")

    def increment_odometer(self, miles):
        self.odometer_reading += miles
\end{lstlisting}

\subsubsection{Explanation}

In the above example, we have defined a class Car with attributes \textit{make, model, year, odometer\_reading}. The \textit{\_\_init\_\_} method is used to initialize the attributes of the class. The \textit{get\_descriptive\_name} method is used to return the \textit{long\_name} variable which is a string. The \textit{read\_odometer} method is used to print the \textit{odometer\_reading} attribute. The \textit{update\_odometer} method is used to update the \textit{odometer\_reading} attribute. The \textit{increment\_odometer} method is used to increment the \textit{odometer\_reading} attribute.

\subsection{Inheritance}

\subsubsection{Definition}

\textit{Inheritance is a way to form new classes using classes that have already been defined. The newly formed classes are called derived classes, the classes that we derive from are called base classes. Important benefits of inheritance are code reuse and reduction of complexity of a program. The derived classes (descendants) override or extend the functionality of base classes (ancestors).}

\subsubsection{Example}

\begin{lstlisting}[language=Python]
class ElectricCar(Car):
    def __init__(self, make, model, year):
        super().__init__(make, model, year)
        self.battery = Battery()
\end{lstlisting}

\subsubsection{Explanation}

In the above example, we have defined a class ElectricCar which inherits the Car class. The \textit{\_\_init\_\_} method is used to initialize the attributes of the class. The \textit{super} method is used to inherit the attributes of the parent class. The \textit{self.battery} attribute is used to store the battery instance.

\subsection{Polymorphism}

\subsubsection{Definition}

\textit{Polymorphism is the ability to present the same interface for differing underlying forms (data types).}

\subsubsection{Example}

\begin{lstlisting}[language=Python]
class Car:
    def __init__(self, make, model, year):
        self.make = make
        self.model = model
        self.year = year

    def get_descriptive_name(self):
        long_name = f"{self.year} {self.make} {self.model}"
        return long_name.title()

    def make_sound(self):
        print("Vroom!")

class ElectricCar(Car):
    def __init__(self, make, model, year, battery_size):
        super().__init__(make, model, year)
        self.battery_size = battery_size

    def describe_battery(self):
        print(f"This car has a {self.battery_size}-kWh battery.")

    def make_sound(self):
        print("Silent, but deadly.")

class HybridCar(Car):
    def __init__(self, make, model, year, gas_mileage, battery_size):
        super().__init__(make, model, year)
        self.gas_mileage = gas_mileage
        self.battery_size = battery_size

    def describe_battery(self):
        print(f"This car has a {self.battery_size}-kWh battery.")

    def describe_gas_mileage(self):
        print(f"This car gets {self.gas_mileage} miles per gallon.")

    def make_sound(self):
        print("Vroom! But also, silent sometimes.")

my_car = Car('audi', 'a4', 2020)
my_electric_car = ElectricCar('tesla', 'model s', 2020, 100)
my_hybrid_car = HybridCar('toyota', 'prius', 2020, 50, 1.5)

cars = [my_car, my_electric_car, my_hybrid_car]

for car in cars:
    print(car.get_descriptive_name())
    car.make_sound()
    if isinstance(car, ElectricCar) or isinstance(car, HybridCar):
        car.describe_battery()
    if isinstance(car, HybridCar):
        car.describe_gas_mileage()

\end{lstlisting}

\section{Input and Output}
\subsection{Input}
Reading marks of students from keyboard.

\subsection{Output}
Display Students Grade as Pass/Fail.

% \section{Requirements}
% \begin{enumerate}
%     \item Python 3.7 or above
%     \item Numpy
% \end{enumerate}

\section{Code}

    
\hypertarget{assignment-6---classes-and-objects}{%
\subsection{Assignment 6 - Classes and
Objects}\label{assignment-6---classes-and-objects}}

    \hypertarget{write-a-program-to-read-3-subject-marks-and-display-pass-or-failed-using-class-and-object.}{%
\subsubsection{Write a program to read 3 subject marks and display pass
or failed using class and
object.}\label{write-a-program-to-read-3-subject-marks-and-display-pass-or-failed-using-class-and-object.}}

    \begin{tcolorbox}[breakable, size=fbox, boxrule=1pt, pad at break*=1mm,colback=cellbackground, colframe=cellborder]
\prompt{In}{incolor}{15}{\boxspacing}
\begin{Verbatim}[commandchars=\\\{\}]
\PY{k}{class} \PY{n+nc}{student}\PY{p}{(}\PY{n+nb}{object}\PY{p}{)}\PY{p}{:}
\PY{+w}{    }\PY{l+s+sd}{\PYZdq{}\PYZdq{}\PYZdq{}Class to manage student stuff. \PYZdq{}\PYZdq{}\PYZdq{}}
    \PY{k}{def} \PY{n+nf+fm}{\PYZus{}\PYZus{}init\PYZus{}\PYZus{}}\PY{p}{(}\PY{n+nb+bp}{self}\PY{p}{,} \PY{n}{name}\PY{p}{,} \PY{n}{subjects}\PY{p}{,} \PY{n}{marks}\PY{p}{)}\PY{p}{:}
        \PY{n+nb+bp}{self}\PY{o}{.}\PY{n}{name} \PY{o}{=} \PY{n}{name}
        \PY{n+nb+bp}{self}\PY{o}{.}\PY{n}{subjects} \PY{o}{=} \PY{n}{subjects}
        \PY{n+nb+bp}{self}\PY{o}{.}\PY{n}{marks} \PY{o}{=} \PY{n}{marks}
        \PY{n+nb+bp}{self}\PY{o}{.}\PY{n}{results} \PY{o}{=} \PY{p}{\PYZob{}}\PY{p}{\PYZcb{}}
        \PY{n+nb+bp}{self}\PY{o}{.}\PY{n}{results} \PY{o}{=} \PY{n+nb+bp}{self}\PY{o}{.}\PY{n}{calculate\PYZus{}results}\PY{p}{(}\PY{p}{)}
    
    \PY{k}{def} \PY{n+nf}{calculate\PYZus{}results}\PY{p}{(}\PY{n+nb+bp}{self}\PY{p}{)}\PY{p}{:}
        \PY{k}{for} \PY{n}{sub} \PY{o+ow}{in} \PY{n+nb+bp}{self}\PY{o}{.}\PY{n}{subjects}\PY{p}{:}
            \PY{n+nb+bp}{self}\PY{o}{.}\PY{n}{results}\PY{p}{[}\PY{n}{sub}\PY{p}{]} \PY{o}{=} \PY{l+s+s1}{\PYZsq{}}\PY{l+s+s1}{Fail}\PY{l+s+s1}{\PYZsq{}}
        \PY{k}{for} \PY{n}{sub}\PY{p}{,} \PY{n}{mark} \PY{o+ow}{in} \PY{n+nb}{zip}\PY{p}{(}\PY{n+nb+bp}{self}\PY{o}{.}\PY{n}{subjects}\PY{p}{,} \PY{n+nb+bp}{self}\PY{o}{.}\PY{n}{marks}\PY{p}{)}\PY{p}{:}
            \PY{n+nb+bp}{self}\PY{o}{.}\PY{n}{results}\PY{p}{[}\PY{n}{sub}\PY{p}{]} \PY{o}{=} \PY{l+s+s1}{\PYZsq{}}\PY{l+s+s1}{Pass}\PY{l+s+s1}{\PYZsq{}} \PY{k}{if} \PY{n}{mark} \PY{o}{\PYZgt{}}\PY{o}{=} \PY{l+m+mi}{50} \PY{k}{else} \PY{l+s+s1}{\PYZsq{}}\PY{l+s+s1}{Fail}\PY{l+s+s1}{\PYZsq{}}
        \PY{k}{return} \PY{n+nb+bp}{self}\PY{o}{.}\PY{n}{results}
    
    \PY{k}{def} \PY{n+nf}{get\PYZus{}results}\PY{p}{(}\PY{n+nb+bp}{self}\PY{p}{)}\PY{p}{:}
        \PY{k}{return} \PY{n+nb+bp}{self}\PY{o}{.}\PY{n}{results}   
\end{Verbatim}
\end{tcolorbox}

    \begin{tcolorbox}[breakable, size=fbox, boxrule=1pt, pad at break*=1mm,colback=cellbackground, colframe=cellborder]
\prompt{In}{incolor}{17}{\boxspacing}
\begin{Verbatim}[commandchars=\\\{\}]
\PY{n}{Ramesh} \PY{o}{=} \PY{n}{student}\PY{p}{(}\PY{l+s+s2}{\PYZdq{}}\PY{l+s+s2}{Ramesh}\PY{l+s+s2}{\PYZdq{}}\PY{p}{,} \PY{p}{[}\PY{l+s+s2}{\PYZdq{}}\PY{l+s+s2}{Maths}\PY{l+s+s2}{\PYZdq{}}\PY{p}{,} \PY{l+s+s2}{\PYZdq{}}\PY{l+s+s2}{Physics}\PY{l+s+s2}{\PYZdq{}}\PY{p}{,} \PY{l+s+s2}{\PYZdq{}}\PY{l+s+s2}{Chemistry}\PY{l+s+s2}{\PYZdq{}}\PY{p}{]}\PY{p}{,} \PY{p}{[}\PY{l+m+mi}{50}\PY{p}{,} \PY{l+m+mi}{60}\PY{p}{,} \PY{l+m+mi}{70}\PY{p}{]}\PY{p}{)}
\PY{n}{Suresh} \PY{o}{=} \PY{n}{student}\PY{p}{(}\PY{l+s+s2}{\PYZdq{}}\PY{l+s+s2}{Suresh}\PY{l+s+s2}{\PYZdq{}}\PY{p}{,} \PY{p}{[}\PY{l+s+s2}{\PYZdq{}}\PY{l+s+s2}{Maths}\PY{l+s+s2}{\PYZdq{}}\PY{p}{,} \PY{l+s+s2}{\PYZdq{}}\PY{l+s+s2}{Physics}\PY{l+s+s2}{\PYZdq{}}\PY{p}{,} \PY{l+s+s2}{\PYZdq{}}\PY{l+s+s2}{Chemistry}\PY{l+s+s2}{\PYZdq{}}\PY{p}{]}\PY{p}{,} \PY{p}{[}\PY{l+m+mi}{50}\PY{p}{,} \PY{l+m+mi}{40}\PY{p}{,} \PY{l+m+mi}{30}\PY{p}{]}\PY{p}{)}
\PY{n}{Tony\PYZus{}Stark} \PY{o}{=} \PY{n}{student}\PY{p}{(}\PY{l+s+s2}{\PYZdq{}}\PY{l+s+s2}{Tony Stark}\PY{l+s+s2}{\PYZdq{}}\PY{p}{,} \PY{p}{[}\PY{l+s+s2}{\PYZdq{}}\PY{l+s+s2}{Maths}\PY{l+s+s2}{\PYZdq{}}\PY{p}{,} \PY{l+s+s2}{\PYZdq{}}\PY{l+s+s2}{Physics}\PY{l+s+s2}{\PYZdq{}}\PY{p}{,} \PY{l+s+s2}{\PYZdq{}}\PY{l+s+s2}{Chemistry}\PY{l+s+s2}{\PYZdq{}}\PY{p}{]}\PY{p}{,} \PY{p}{[}\PY{l+m+mi}{100}\PY{p}{,} \PY{l+m+mi}{100}\PY{p}{,} \PY{l+m+mi}{100}\PY{p}{]}\PY{p}{)}

\PY{k}{for} \PY{n}{sub}\PY{p}{,} \PY{n}{res} \PY{o+ow}{in} \PY{n+nb}{zip}\PY{p}{(}\PY{n}{Ramesh}\PY{o}{.}\PY{n}{subjects}\PY{p}{,} \PY{n}{Ramesh}\PY{o}{.}\PY{n}{results}\PY{o}{.}\PY{n}{values}\PY{p}{(}\PY{p}{)}\PY{p}{)}\PY{p}{:}
    \PY{n+nb}{print}\PY{p}{(}\PY{l+s+sa}{f}\PY{l+s+s1}{\PYZsq{}}\PY{l+s+s1}{The Result of }\PY{l+s+si}{\PYZob{}}\PY{n}{sub}\PY{l+s+si}{\PYZcb{}}\PY{l+s+s1}{ for }\PY{l+s+si}{\PYZob{}}\PY{n}{Ramesh}\PY{o}{.}\PY{n}{name}\PY{l+s+si}{\PYZcb{}}\PY{l+s+s1}{ is }\PY{l+s+si}{\PYZob{}}\PY{n}{res}\PY{l+s+si}{\PYZcb{}}\PY{l+s+s1}{\PYZsq{}}\PY{p}{)}
\PY{k}{for} \PY{n}{sub}\PY{p}{,} \PY{n}{res} \PY{o+ow}{in} \PY{n+nb}{zip}\PY{p}{(}\PY{n}{Suresh}\PY{o}{.}\PY{n}{subjects}\PY{p}{,} \PY{n}{Suresh}\PY{o}{.}\PY{n}{results}\PY{o}{.}\PY{n}{values}\PY{p}{(}\PY{p}{)}\PY{p}{)}\PY{p}{:}
    \PY{n+nb}{print}\PY{p}{(}\PY{l+s+sa}{f}\PY{l+s+s1}{\PYZsq{}}\PY{l+s+s1}{The Result of }\PY{l+s+si}{\PYZob{}}\PY{n}{sub}\PY{l+s+si}{\PYZcb{}}\PY{l+s+s1}{ for }\PY{l+s+si}{\PYZob{}}\PY{n}{Suresh}\PY{o}{.}\PY{n}{name}\PY{l+s+si}{\PYZcb{}}\PY{l+s+s1}{ is }\PY{l+s+si}{\PYZob{}}\PY{n}{res}\PY{l+s+si}{\PYZcb{}}\PY{l+s+s1}{\PYZsq{}}\PY{p}{)}
\PY{k}{for} \PY{n}{sub}\PY{p}{,} \PY{n}{res} \PY{o+ow}{in} \PY{n+nb}{zip}\PY{p}{(}\PY{n}{Tony\PYZus{}Stark}\PY{o}{.}\PY{n}{subjects}\PY{p}{,} \PY{n}{Tony\PYZus{}Stark}\PY{o}{.}\PY{n}{results}\PY{o}{.}\PY{n}{values}\PY{p}{(}\PY{p}{)}\PY{p}{)}\PY{p}{:}
    \PY{n+nb}{print}\PY{p}{(}\PY{l+s+sa}{f}\PY{l+s+s1}{\PYZsq{}}\PY{l+s+s1}{The Result of }\PY{l+s+si}{\PYZob{}}\PY{n}{sub}\PY{l+s+si}{\PYZcb{}}\PY{l+s+s1}{ for }\PY{l+s+si}{\PYZob{}}\PY{n}{Tony\PYZus{}Stark}\PY{o}{.}\PY{n}{name}\PY{l+s+si}{\PYZcb{}}\PY{l+s+s1}{ is }\PY{l+s+si}{\PYZob{}}\PY{n}{res}\PY{l+s+si}{\PYZcb{}}\PY{l+s+s1}{\PYZsq{}}\PY{p}{)}
\end{Verbatim}
\end{tcolorbox}

    \begin{Verbatim}[commandchars=\\\{\}]
The Result of Maths for Ramesh is Pass
The Result of Physics for Ramesh is Pass
The Result of Chemistry for Ramesh is Pass
The Result of Maths for Suresh is Pass
The Result of Physics for Suresh is Fail
The Result of Chemistry for Suresh is Fail
The Result of Maths for Tony Stark is Pass
The Result of Physics for Tony Stark is Pass
The Result of Chemistry for Tony Stark is Pass
    \end{Verbatim}


\section{Conclusion}
Studied Object oriented programming in Python and classes and objects are used to display
Students Grade as Pass/Fail.

\clearpage

\section{FAQ}
\begin{enumerate}
	\item \textbf{What is difference between procedural proxgramming and Object oriented programming.}\\

	      \textbf{Procedural Programming}:
	      \begin{itemize}
		      \item Procedural programming is a programming paradigm, based on the concept of the procedure call.
		      \item  A procedure, also known as a subroutine, is a routine that contains a series of programming
		      \item  statements to perform a specific task.
		      \item  A procedure is called by another part of the program using a procedure call.
		      \item  The main advantage of procedural programming is that it breaks a large, complex task into smaller and simpler sub-tasks.
		      \item  The disadvantage of procedural programming is that it does not model the real world very well.
		      \item  In procedural programming, the data and the functions that operate on the data are not associated with any particular class or object.
		      \item  In procedural programming, the data is passed to the functions as parameters.
		      \item  It is different from Object Oriented Programming (OOP) in the sense that in OOP, the data and the functions that operate on the data are associated with a class and an object.
	      \end{itemize}

	      \textbf{Object Oriented Programming}:
	      \begin{itemize}
		      \item Object-oriented programming is a programming paradigm based on the concept of "objects", which can contain data, in the form of fields, often known as attributes; and code, in the form of procedures, often known as methods.
		      \item A feature of objects is that an object's procedures can access and often modify the data fields of the object with which they are associated (objects have a notion of "this" or "self").
		      \item  In OOP, computer programs are designed by making them out of objects that interact with one another.
		      \item  An object can be defined as a data field that has unique attributes and behavior.
		      \item  Objects are basically the things you think about first in designing a program and they are also the units of code that are eventually derived from the process.
		      \item  A Class is a blueprint for the object.
		      \item  The Main advantage of OOP is that it provides a clear modular structure for programs. It is good for defining abstract data types, and also allows programmers to create full reusable applications with less code and shorter development times. It also makes software easier to maintain and modify.
		      \item The disadvantage of OOP is that it makes programs slower at runtime because of the overhead involved in the creation of the classes and objects.
		      \item In OOP, the data is not passed to the functions, but the functions are associated with the data.
	      \end{itemize}

	\item \textbf{What is superclass?}\\

	      \textbf{Superclass}: The Superclass is the class from which a subclass inherits. The subclass inherits all the attributes and methods of the superclass.

	      \textbf{Example}

	      \begin{Verbatim}[commandchars=\\\{\}]
class Parent:
	def __init__(self):
		self.value = "Inside Parent"

	def get_value(self):
		return self.value

class Child(Parent):

	def __init__(self):
		super().__init__()
		self.value = "Inside Child"

	def get_value(self):
		return "Inside Child"

	def get_parent_value(self):
		return super().get_value()
	      \end{Verbatim}

	      \begin{Verbatim}[commandchars=\\\{\}]
		      Inside Child
		      Inside Parent
	      \end{Verbatim}
	\item \textbf{Explain instance variables and Class variables?}\\

	      \textbf{Instance Variables}: Instance variables are variables whose value is assigned inside a constructor or method with self. Instance variables are not shared between objects. Each object contains its own copy of the instance variable.

	      \textbf{Example}

	      \begin{Verbatim}[commandchars=\\\{\}]
class Student:
	def __init__(self, name):
		self.name = name

	def get_name(self):
		return self.name
	      \end{Verbatim}

	      \begin{Verbatim}[commandchars=\\\{\}]
>>> student1 = Student("Ramesh")
>>> student2 = Student("Suresh")
>>> student1.get_name()
'Ramesh'
>>> student2.get_name()
'Suresh'
	      \end{Verbatim}

	      \textbf{Class Variables}: Class variables are variables whose value is assigned in the class. Class variables are shared between all objects. For example, the company name of all the employees is the same. So it is assigned in the class. It saves memory.

	      \textbf{Example}

	      \begin{Verbatim}[commandchars=\\\{\}]
class Student:
company = "MegaCorp"

	def __init__(self, name):
		self.name = name

	def get_name(self):
		return self.name

	def get_company(self):
		return self.company
	      \end{Verbatim}

	      \begin{Verbatim}[commandchars=\\\{\}]
>>> student1 = Student("Ramesh")
>>> student2 = Student("Suresh")
>>> student1.get_name()
'Ramesh'
>>> student2.get_name()
'Suresh'
>>> student1.get_company()
'MegaCorp'
>>> student2.get_company()
'MegaCorp'
	      \end{Verbatim}

	\item \textbf{What is the purpose of the init method?}\\

	      \textbf{init}: The init method is similar to constructors in C++ and Java. Constructors are used to initialize the object's state. The task of constructors is to initialize(assign values) to the data members of the class when an object of class is created. Like methods, a constructor also contains collection of statements (i.e. instructions) that are executed at time of Object creation. It is run as soon as an object of a class is instantiated. The method is useful to do any initialization you want to do with your object.

	      \textbf{Example}

	      \begin{verbatim}
class Student:
	def __init__(self, name):
		self.name = name

	def get_name(self):
		return self.name
	      \end{verbatim}

	      \begin{verbatim}
>>> student1 = Student("Ramesh")
>>> student2 = Student("Suresh")
>>> student1.get_name()
'Ramesh'
>>> student2.get_name()
'Suresh'
	      \end{verbatim}

	\item \textbf{What is 'self' in Python?}\\

	      \textbf{self}: The self parameter is a reference to the current instance of the class, and is used to access variables that belongs to the class. It does not have to be named self , you can call it whatever you like, but it has to be the first parameter of any function in the class.



\end{enumerate}

\end{document}