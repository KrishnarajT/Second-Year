% This is a Basic Assignment Paper but with like Code and stuff allowed in it, there is also url, hyperlinks from contents included. 

\documentclass[11pt]{article}

% Preamble

\usepackage[margin=1in]{geometry}
\usepackage{amsfonts, amsmath, amssymb}
\usepackage{fancyhdr, float, graphicx}
\usepackage[utf8]{inputenc} % Required for inputting international characters
\usepackage[T1]{fontenc} % Output font encoding for international characters
\usepackage{fouriernc} % Use the New Century Schoolbook font
\usepackage[nottoc, notlot, notlof]{tocbibind}
\usepackage{listings}
\usepackage{xcolor}
\usepackage{blindtext}
\usepackage{hyperref}
\hypersetup{
    colorlinks=true,
    linkcolor=black,
    filecolor=magenta,      
    urlcolor=cyan,
    pdfpagemode=FullScreen,
    }

\definecolor{codegreen}{rgb}{0,0.6,0}
\definecolor{codegray}{rgb}{0.5,0.5,0.5}
\definecolor{codepurple}{rgb}{0.58,0,0.82}
\definecolor{backcolour}{rgb}{0.95,0.95,0.92}

\lstdefinestyle{mystyle}{
    backgroundcolor=\color{backcolour},   
    commentstyle=\color{codegreen},
    keywordstyle=\color{magenta},
    numberstyle=\tiny\color{codegray},
    stringstyle=\color{codepurple},
    basicstyle=\ttfamily\footnotesize,
    breakatwhitespace=false,         
    breaklines=true,                 
    captionpos=b,                    
    keepspaces=true,                 
    numbers=left,                    
    numbersep=5pt,                  
    showspaces=false,                
    showstringspaces=false,
    showtabs=false,                  
    tabsize=2
}

\lstset{style=mystyle}

% Header and Footer
\pagestyle{fancy}
\fancyhead{}
\fancyfoot{}
\fancyhead[L]{\textit{\Large{Database Management Systems Assignment 8}}}
%\fancyhead[R]{\textit{something}}
\fancyfoot[C]{\thepage}
\renewcommand{\footrulewidth}{1pt}



% Other Doc Editing
% \parindent 0ex
%\renewcommand{\baselinestretch}{1.5}

\begin{document}

\begin{titlepage}
    \centering

    %---------------------------NAMES-------------------------------

    \huge\textsc{
        MIT World Peace University
    }\\

    \vspace{0.75\baselineskip} % space after Uni Name

    \LARGE{
        Database Management Systems\\
        Second Year B. Tech, Semester 4
    }

    \vfill % space after Sub Name

    %--------------------------TITLE-------------------------------

    \rule{\textwidth}{1.6pt}\vspace*{-\baselineskip}\vspace*{2pt}
    \rule{\textwidth}{0.6pt}
    \vspace{0.75\baselineskip} % Whitespace above the title



    \huge{\textsc{
            Cursors using PL/SQL
        }} \\



    \vspace{0.5\baselineskip} % Whitespace below the title
    \rule{\textwidth}{0.6pt}\vspace*{-\baselineskip}\vspace*{2.8pt}
    \rule{\textwidth}{1.6pt}

    \vspace{1\baselineskip} % Whitespace after the title block

    %--------------------------SUBTITLE --------------------------	

    \LARGE\textsc{
        Assignment No. 8
    } % Subtitle or further description
    \vfill

    %--------------------------AUTHOR-------------------------------

    Prepared By
    \vspace{0.5\baselineskip} % Whitespace before the editors

    \Large{
        Krishnaraj Thadesar \\
        Cyber Security and Forensics\\
        Batch A1, PA 20
    }


    \vspace{0.5\baselineskip} % Whitespace below the editor list
    \today

\end{titlepage}


\tableofcontents
\thispagestyle{empty}
\clearpage

\setcounter{page}{1}

\section{Aim}
Write PL/SQL Cursor for the given problem statements

\section{Objectives}
\begin{enumerate}
    \item To study and use Cursor using MySQL PL/SQL block
\end{enumerate}


\section{Problem Statement}
Create tables and solve given queries.

\section{Theory}

Explain following points

\subsection{What is Cursor?}

A cursor is a database object that allows a program to retrieve data from a database one row at a time. It acts like a pointer that points to a specific row in a result set returned by a SQL query. Cursors are useful in situations where we need to manipulate data row by row, such as in a loop. They allow us to navigate through the rows of a result set and manipulate them as needed.

Example:
\begin{lstlisting}[language=sql]
    DECLARE
        CURSOR c1 IS
            SELECT * FROM employees;
    BEGIN
        FOR employee IN c1 LOOP
            DBMS_OUTPUT.PUT_LINE(employee.first_name || ' ' || employee.last_name);
        END LOOP;
    END;
\end{lstlisting}

\subsection{Why do we need the Cursors?}

Cursors are necessary in situations where we need to retrieve data from a database one row at a time and process it. For example, if we want to perform a complex calculation on a large dataset, it may be more efficient to retrieve the data row by row rather than all at once. Cursors enable us to retrieve data one row at a time and process it as needed, which can be particularly useful in situations where we need to perform complex processing on data.

\begin{enumerate}
    \item Cursors are used to retrieve data from a database one row at a time.
    \item  Cursors are used to process data row by row.
    \item  Cursors are used to perform complex calculations on large datasets.
\end{enumerate}

\subsection{Different types of Cursors}

There are two types of cursors in PL/SQL:

\begin{enumerate}
    \item Implicit Cursor: This is a cursor that is automatically created by the Oracle database when a SQL statement is executed. It is used for simple, one-time queries that return a small number of rows.
    \item Explicit Cursor: This is a cursor that is explicitly declared and defined in a PL/SQL program. It is used for more complex queries that require multiple rows to be returned and processed.
\end{enumerate}

\subsection{Drawbacks of Implicit Cursors}

Implicit cursors have a few drawbacks, including:

\begin{enumerate}
    \item They do not provide much control over the result set returned by the query.
    \item They do not allow us to manipulate data row by row.
    \item They can cause performance issues if used in a loop or if the query returns a large number of rows.
\end{enumerate}

\subsection{PL/SQL variables in a Cursor}

PL/SQL variables can be used in a cursor to pass values into the query. For example, we can declare a variable in the cursor declaration and use it in the WHERE clause of the query. This can be particularly useful when we need to retrieve a subset of data from a larger dataset.


\subsection{Opening a Cursor}

Before we can use a cursor, we need to open it using the OPEN statement. This statement prepares the cursor for fetching data from the result set. Once the cursor is open, we can begin fetching data from it.

\begin{lstlisting}[language=sql]
    DECLARE
        CURSOR c1 IS
            SELECT * FROM employees;
    BEGIN
        OPEN c1;
        ...
    END;
\end{lstlisting}

\subsection{Fetching from a Cursor}

To retrieve data from a cursor, we use the FETCH statement. This statement retrieves the next row from the result set and makes it available for processing. We can fetch data from the cursor one row at a time, or we can retrieve all of the rows at once.

\begin{lstlisting}[language=sql]
    DECLARE
        CURSOR c1 IS
            SELECT * FROM employees;
        emp_row employees%ROWTYPE;
    BEGIN
        OPEN c1;
        FETCH c1 INTO emp_row;
        ...
    END;
\end{lstlisting}

\subsection{Closing a Cursor}

Once we are done processing the result set, we need to close the cursor using the CLOSE statement. This statement releases the resources used by the cursor and frees up memory. It is important to close cursors when we are done using them to avoid wasting system resources.

\begin{lstlisting}[language=sql]
    DECLARE
        CURSOR c1 IS
            SELECT * FROM employees;
        emp_row employees%ROWTYPE;
    BEGIN
        OPEN c1;
        FETCH c1 INTO emp_row;
        CLOSE c1;
    END;
\end{lstlisting}

\subsection{Cursor attributes}
Cursor attributes are properties of a cursor that provide information about its status. Some examples of cursor attributes include \%FOUND, which indicates whether the last fetch retrieved a row, and \%NOTFOUND, which indicates whether the last fetch did not retrieve a row.

\begin{lstlisting}[language=sql]
  DECLARE
    CURSOR employee_cursor IS
      SELECT * FROM employees;
    emp_row employees%ROWTYPE;
  BEGIN
    OPEN employee_cursor;
    LOOP
      FETCH employee_cursor INTO emp_row;
      EXIT WHEN employee_cursor%NOTFOUND;
      -- process data here
    END LOOP;
    CLOSE employee_cursor;
  END;
\end{lstlisting}

\section{Platform}
\textbf{Operating System}: Arch Linux x86-64 \\
\textbf{IDEs or Text Editors Used}: Draw.io for Drawing the ER diagram. \\


\section{Input}
Given Database from the Problem Statement for the Assignment for our batch. (A1 PA 20)

\section{Queries}
\lstinputlisting[language=SQL]{../../Programs/Assignment_8_queries.sql}

\section{Outputs in Execution}
\lstinputlisting[language=SQL]{../../Programs/Assignment_8.md}


\section{Conclusion}
Thus, we have learned creating and using Cursor in SQL. We have also learned about the different types of Cursors and their advantages and disadvantages.

\clearpage

\section{FAQ}
\begin{enumerate}


    \item \textbf{Enlist Advantages of Cursors ?}\\
          Advantages of Cursors:
          \begin{enumerate}
              \item Flexibility: Cursors offer a flexible way to process database records one at a time, allowing for complex processing logic to be applied to each record individually.
              \item Control: Cursors give developers more control over the data being processed, allowing for precise manipulation of records and fields.
              \item Sequential processing: Cursors allow records to be processed in a sequential order, which can be important in cases where records need to be processed in a specific order.
              \item Record-level operations: Cursors allow for record-level operations such as inserting, updating, and deleting individual records in a result set.
          \end{enumerate}

    \item \textbf{Enlist Disadvantages of Cursors?}\\

          Disadvantages of Cursors:
          \begin{enumerate}
              \item Overhead: Cursors can add overhead to the database server, as they require resources to be allocated to hold the result set and manage the cursor.
              \item Performance: Cursors can have a negative impact on database performance, particularly if they are used to process large result sets.
              \item Complexity: Cursors can make code more complex and difficult to understand, particularly if they are nested or used in conjunction with other database operations.
          \end{enumerate}
    \item \textbf{What are the Applications of Cursors.}\\

          Applications of Cursors:
          \begin{enumerate}
              \item Report generation: Cursors can be used to generate reports based on complex queries that require record-level processing.
              \item Data validation: Cursors can be used to validate data against complex business rules that cannot be easily expressed in a single query.
              \item Data migration: Cursors can be used to migrate data between databases, particularly when data needs to be transformed or manipulated during the migration process.
              \item Batch processing: Cursors can be used for batch processing operations, such as updating a large number of records based on a specific criteria.
          \end{enumerate}

    \item \textbf{Why do we need the Cursors?}\\

          Cursors are needed in situations where processing of individual records in a result set is required. They provide a way to traverse and manipulate data one record at a time, allowing for complex processing logic to be applied to each record individually. Cursors are particularly useful in scenarios where complex business rules need to be applied to individual records, and in cases where record-level operations such as inserting, updating, and deleting are required. However, cursors can also have a negative impact on database performance and should be used judiciously.
\end{enumerate}



\end{document}