% This is a Basic Assignment Paper but with like Code and stuff allowed in it, there is also url, hyperlinks from contents included. 

\documentclass[11pt]{article}

% Preamble

\usepackage[margin=1in]{geometry}
\usepackage{amsfonts, amsmath, amssymb}
\usepackage{fancyhdr, float, graphicx}
\usepackage[utf8]{inputenc} % Required for inputting international characters
\usepackage[T1]{fontenc} % Output font encoding for international characters
\usepackage{fouriernc} % Use the New Century Schoolbook font
\usepackage[nottoc, notlot, notlof]{tocbibind}
\usepackage{listings}
\usepackage{xcolor}
\usepackage{blindtext}
\usepackage{hyperref}
\hypersetup{
    colorlinks=true,
    linkcolor=black,
    filecolor=magenta,      
    urlcolor=cyan,
    pdfpagemode=FullScreen,
    }

\definecolor{codegreen}{rgb}{0,0.6,0}
\definecolor{codegray}{rgb}{0.5,0.5,0.5}
\definecolor{codepurple}{rgb}{0.58,0,0.82}
\definecolor{backcolour}{rgb}{0.95,0.95,0.92}

\lstdefinestyle{mystyle}{
    backgroundcolor=\color{backcolour},   
    commentstyle=\color{codegreen},
    keywordstyle=\color{magenta},
    numberstyle=\tiny\color{codegray},
    stringstyle=\color{codepurple},
    basicstyle=\ttfamily\footnotesize,
    breakatwhitespace=false,         
    breaklines=true,                 
    captionpos=b,                    
    keepspaces=true,                 
    numbers=left,                    
    numbersep=5pt,                  
    showspaces=false,                
    showstringspaces=false,
    showtabs=false,                  
    tabsize=2
}

\lstset{style=mystyle}

% Header and Footer
\pagestyle{fancy}
\fancyhead{}
\fancyfoot{}
\fancyhead[L]{\textit{\Large{Database Management Systems Assignment 2}}}
%\fancyhead[R]{\textit{something}}
\fancyfoot[C]{\thepage}
\renewcommand{\footrulewidth}{1pt}



% Other Doc Editing
% \parindent 0ex
%\renewcommand{\baselinestretch}{1.5}

\begin{document}

\begin{titlepage}
	\centering

	%---------------------------NAMES-------------------------------

	\huge\textsc{
		MIT World Peace University
	}\\

	\vspace{0.75\baselineskip} % space after Uni Name

	\LARGE{
		Database Management Systems\\
		Second Year B. Tech, Semester 4
	}

	\vfill % space after Sub Name

	%--------------------------TITLE-------------------------------

	\rule{\textwidth}{1.6pt}\vspace*{-\baselineskip}\vspace*{2pt}
	\rule{\textwidth}{0.6pt}
	\vspace{0.75\baselineskip} % Whitespace above the title



	\huge{\textsc{
			Learning SQL DCL and DDL Commands\\
			\textit{Data Definition Language and Data Control Language}
		}} \\



	\vspace{0.5\baselineskip} % Whitespace below the title
	\rule{\textwidth}{0.6pt}\vspace*{-\baselineskip}\vspace*{2.8pt}
	\rule{\textwidth}{1.6pt}

	\vspace{1\baselineskip} % Whitespace after the title block

	%--------------------------SUBTITLE --------------------------	

	\LARGE\textsc{
		Assignment No. 2
	} % Subtitle or further description
	\vfill

	%--------------------------AUTHOR-------------------------------

	Prepared By
	\vspace{0.5\baselineskip} % Whitespace before the editors

	\Large{
		Krishnaraj Thadesar \\
		Cyber Security and Forensics\\
		Batch A1, PA 20
	}


	\vspace{0.5\baselineskip} % Whitespace below the editor list
	\today

\end{titlepage}


\tableofcontents
\thispagestyle{empty}
\clearpage

\setcounter{page}{1}

\section{Aim}
Design and Develop SQL DDL statements for different system.

\section{Objectives}
To study DDL, DCL commands.

\section{Problem Statement}

\section{Theory}
\subsection{SQL Data Definition Language (DDL)}

\subsubsection{What is Data Definition Language?}
\begin{quote}
	\textit{Data Definition Language (DDL) is a computer language used to define the database schema. It includes commands to create, modify and drop database objects in the database. It is used to define the database structure or schema. It is also used to define the access permissions on the data, or the views that are presented to different users.}
\end{quote}
\subsubsection{DDL Commands}

The following are the Commands that are used in DDL:

\begin{enumerate}
	\item CREATE - Creates a new database or a new table in a database.
	\item ALTER - Modifies a database or a table.
	\item DROP - Deletes a database or a table.
	\item TRUNCATE - Deletes all the records from a table, including all spaces allocated for the records are removed.
	\item COMMENT - Adds comments to the data dictionary.
	\item RENAME - Renames an object.
\end{enumerate}

\subsubsection{DDL Command Syntax and Examples}

\begin{enumerate}
	\item CREATE TABLE - Creates a new database table.
	      \begin{verbatim}
		CREATE TABLE table_name constraints
		(
			Column_name datatype(size) constraints default '',
			Column_name datatype(size),
			constraint(column_name)
		);
	\end{verbatim}
	\item ALTER TABLE - Changes in columns and stuff.
	      \begin{verbatim}
		ALTER TABLE table_name
		ADD column_name datatype;
	\end{verbatim}
	\item DROP TABLE - Deletes a table from the database.
	      \begin{verbatim}
		DROP TABLE table_name;
	\end{verbatim}
	\item RENAME TABLE - Renames a table.
	      \begin{verbatim}
		RENAME TABLE old_name TO new_name;
	\end{verbatim}
	\item TRUNCATE TABLE - Deletes all the records from a table.
	      \begin{verbatim}
		TRUNCATE TABLE table_name;
	\end{verbatim}
	\item COMMENT ON - Adds comments to the data dictionary.
	      \begin{verbatim}
		COMMENT ON TABLE table_name IS 'comment';
	\end{verbatim}
\end{enumerate}

\subsection{SQL Data Control Language (DCL)}

\subsubsection{What is Data Control Language?}
\begin{quote}
	\textit{Data Control Language (DCL) is a computer language used to define the access permissions on the data, or the views that are presented to different users. It includes commands to grant and deny privileges on database objects to users.}
\end{quote}
\subsubsection{DCL Commands}

The following are the Commands that are used in DCL:

\begin{enumerate}
	\item GRANT - Gives the specified privileges to the specified user.
	\item REVOKE - Takes back the specified privileges from the specified user.
\end{enumerate}

\subsection{DCL Command Syntax and Examples}

\begin{enumerate}
	\item GRANT - Gives the specified privileges to the specified user.
	      \begin{verbatim}
		GRANT privileges ON object_name TO user_name;
	\end{verbatim}
	\item REVOKE - Takes back the specified privileges from the specified user.
	      \begin{verbatim}
		REVOKE privileges ON object_name FROM user_name;
	\end{verbatim}
\end{enumerate}


\section{Platform}
\textbf{Operating System}: Arch Linux x86-64 \\
\textbf{IDEs or Text Editors Used}: Draw.io for Drawing the ER diagram. \\

% \section{Pseudo Code or Algorithm}

\section{Input}
Given Database from the Problem Statement for the Assignment for our batch. (A1 PA 20)
\section{Output}
\lstinputlisting[language=SQL]{../../Programs/Assignment_2.md}

\section{Conclusion}
Thus, we have learned DDL and DCL commands thoroughly.
\clearpage

\section{FAQ}
\begin{enumerate}
	\item \textit{How to drop a column from a table?}
	      \begin{verbatim}
		ALTER TABLE table_name
		DROP COLUMN column_name;
	\end{verbatim}

	\item \textit{How to add a primary key in an already existing table?}
	      \begin{verbatim}
		ALTER TABLE table_name
		ADD PRIMARY KEY (column_name);
	\end{verbatim}
	\item \textit{How to create a new user in MySQL?}
	      \begin{verbatim}
		CREATE USER 'username'@'localhost' IDENTIFIED BY 'password';
	\end{verbatim}
\end{enumerate}



\end{document}