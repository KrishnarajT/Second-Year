% This is a Basic Assignment Paper but with like Code and stuff allowed in it, there is also url, hyperlinks from contents included. 

\documentclass[11pt]{article}

% Preamble

\usepackage[margin=1in]{geometry}
\usepackage{amsfonts, amsmath, amssymb}
\usepackage{fancyhdr, float, graphicx}
\usepackage[utf8]{inputenc} % Required for inputting international characters
\usepackage[T1]{fontenc} % Output font encoding for international characters
\usepackage{fouriernc} % Use the New Century Schoolbook font
\usepackage[nottoc, notlot, notlof]{tocbibind}
\usepackage{listings}
\usepackage{xcolor}
\usepackage{blindtext}
\usepackage{hyperref}
\hypersetup{
    colorlinks=true,
    linkcolor=black,
    filecolor=magenta,      
    urlcolor=cyan,
    pdfpagemode=FullScreen,
    }

\definecolor{codegreen}{rgb}{0,0.6,0}
\definecolor{codegray}{rgb}{0.5,0.5,0.5}
\definecolor{codepurple}{rgb}{0.58,0,0.82}
\definecolor{backcolour}{rgb}{0.95,0.95,0.92}

\lstdefinestyle{mystyle}{
    backgroundcolor=\color{backcolour},   
    commentstyle=\color{codegreen},
    keywordstyle=\color{magenta},
    numberstyle=\tiny\color{codegray},
    stringstyle=\color{codepurple},
    basicstyle=\ttfamily\footnotesize,
    breakatwhitespace=false,         
    breaklines=true,                 
    captionpos=b,                    
    keepspaces=true,                 
    numbers=left,                    
    numbersep=5pt,                  
    showspaces=false,                
    showstringspaces=false,
    showtabs=false,                  
    tabsize=2
}

\lstset{style=mystyle}

% Header and Footer
\pagestyle{fancy}
\fancyhead{}
\fancyfoot{}
\fancyhead[L]{\textit{\Large{Database Management Systems Assignment 10}}}
%\fancyhead[R]{\textit{something}}
\fancyfoot[C]{\thepage}
\renewcommand{\footrulewidth}{1pt}



% Other Doc Editing
% \parindent 0ex
%\renewcommand{\baselinestretch}{1.5}

\begin{document}

\begin{titlepage}
    \centering

    %---------------------------NAMES-------------------------------

    \huge\textsc{
        MIT World Peace University
    }\\

    \vspace{0.75\baselineskip} % space after Uni Name

    \LARGE{
        Database Management Systems\\
        Second Year B. Tech, Semester 4
    }

    \vfill % space after Sub Name

    %--------------------------TITLE-------------------------------

    \rule{\textwidth}{1.6pt}\vspace*{-\baselineskip}\vspace*{2pt}
    \rule{\textwidth}{0.6pt}
    \vspace{0.75\baselineskip} % Whitespace above the title



    \huge{\textsc{
            Basics of JSON
        }} \\



    \vspace{0.5\baselineskip} % Whitespace below the title
    \rule{\textwidth}{0.6pt}\vspace*{-\baselineskip}\vspace*{2.8pt}
    \rule{\textwidth}{1.6pt}

    \vspace{1\baselineskip} % Whitespace after the title block

    %--------------------------SUBTITLE --------------------------	

    \LARGE\textsc{
        Assignment No. 9
    } % Subtitle or further description
    \vfill

    %--------------------------AUTHOR-------------------------------

    Prepared By
    \vspace{0.5\baselineskip} % Whitespace before the editors

    \Large{
        Krishnaraj Thadesar \\
        Cyber Security and Forensics\\
        Batch A1, PA 20
    }


    \vspace{0.5\baselineskip} % Whitespace below the editor list
    \today

\end{titlepage}


\tableofcontents
\thispagestyle{empty}
\clearpage

\setcounter{page}{1}

\section{Aim}
Create a json document and write the json query to displayt the data.

\section{Objectives}
\begin{enumerate}
    \item To understand the key structure elements of a json file: object names and
          values.
    \item To study the core data types that json files can store including: boolean,
          numeric and string; hierarchical json structures including: objects, arrays and
          data elements.
    \item To use MySQL JSON data type to store JSON documents in the database
          and perform CRUD operations.\end{enumerate}

\section{Problem Statement}
Create tables and solve given queries.

\section{Theory}

\subsection{Introduction to JSON}

JSON, short for JavaScript Object Notation, is a lightweight data interchange format. It is a text format that is easy to read and write and is often used for transmitting data between a server and a web application. JSON has become popular because of its simplicity and flexibility in handling data structures, making it a popular choice for data exchange between systems.

JSON is based on two basic structures: key-value pairs and arrays. Key-value pairs consist of a key and a value, separated by a colon, and are enclosed in curly braces. Arrays are lists of values and are enclosed in square brackets. JSON is human-readable, which means that it can be easily understood by people, and it is also machine-readable, which means that computers can easily parse and generate JSON data.

\subsection{Reading and Writing Files in Python}

Python provides built-in support for reading and writing files. To open a file in Python, you can use the open() function, which takes two arguments: the file name and the mode in which you want to open the file. The mode can be "r" for reading, "w" for writing, "a" for appending, and "x" for exclusive creation. Once you have opened a file, you can read or write data to it using the file object's methods.

To read data from a file, you can use the read() method, which reads the entire contents of the file as a string. You can also use the readline() method to read a single line at a time or the readlines() method to read all the lines into a list. To write data to a file, you can use the write() method, which writes a string to the file, or the writelines() method, which writes a list of strings to the file.

Syntax for reading a file in Python:

\begin{lstlisting}[language=python]
with open("file.txt", "r") as f:
    data = f.read()
\end{lstlisting}

Syntax for writing to a file in Python:

\begin{lstlisting}[language=python]
with open("file.txt", "w") as f:
    f.write("Hello, world!")
\end{lstlisting}

\subsection{Writing to a JSON File}

Python provides a built-in module called json for working with JSON data. To write JSON data to a file, you can use the json.dump() function, which takes two arguments: the data you want to write and the file object you want to write it to. The json.dump() function automatically converts the data to JSON format and writes it to the file.

Syntax for writing JSON data to a file in Python:
\begin{lstlisting}[language=python]
import json

data = {"name": "John", "age": 30, "city": "New York"}

with open("data.json", "w") as f:
json.dump(data, f)
\end{lstlisting}

\subsection{Introduction to MySQL and JSON data type}

MySQL is a popular open-source relational database management system that provides a way to store and manage structured data. MySQL supports a JSON data type, which allows you to store JSON data in a column of a table. The JSON data type is a flexible and efficient way to store and manipulate data that has a variable structure.

When you store JSON data in a MySQL database, you can use SQL to query and manipulate the data just like you would with any other data type. MySQL provides a set of functions and operators for working with JSON data, such as JSONEXTRACT() for extracting data from a JSON object, JSONARRAY() for creating a JSON array, and JSONOBJECT() for creating a JSON object.

Syntax for creating a table with a JSON column in MySQL:

\begin{lstlisting}[language=sql]
CREATE TABLE table_name (
    column1 datatype,
    column2 datatype,
    json_column JSON,
    ...
);    

\end{lstlisting}
\section{Platform}
\textbf{Operating System}: Arch Linux x86-64 \\
\textbf{IDEs or Text Editors Used}: Visual Studio Code \\
\textbf{Interpreters Used}: Python 3.11\\

% \section{Pseudo Code or Algorithm}

\section{Input}
\begin{enumerate}
    \item Read CSV file and convert it into json file with python
    \item Read text file and convert it into json file with python
    \item Database with JSON Data field
\end{enumerate}

\section{Outputs}
\lstinputlisting[language=python, caption=Python Code]{../../Programs/assignment_10.py}
\lstinputlisting[language=python, caption=CSV Input]{../../Programs/data.csv}
\lstinputlisting[language=python, caption=JSON Output]{../../Programs/data.json}

\section{Conclusion}
Thus, we have learned about the data types, libraries and methods to create
JSON documents for storing and retrieving the data.
\clearpage

\section{FAQ}
\begin{enumerate}
    \item What is JSON?\\

          JSON stands for JavaScript Object Notation, which is a lightweight data interchange format that is easy for humans to read and write, and easy for machines to parse and generate. It is often used for transmitting data between a server and a web application.
    \item Mention what is the rule for JSON syntax rules? Give an example of JSON object?\\

          JSON syntax follows a set of rules that determine how data is represented in a JSON object. Some of the key rules for JSON syntax are:
          \begin{enumerate}
              \item JSON data is represented as key-value pairs, enclosed in curly braces { }.
              \item Each key in a JSON object must be a string enclosed in double quotes, followed by a colon, and then its value.
              \item Multiple key-value pairs in a JSON object are separated by commas.
          \end{enumerate}
          Here's an example of a JSON object that represents information about a person:
          \begin{lstlisting}[language=python]
{
"name": "John Doe",
"age": 30,
"address": {
"street": "123 Main St",
"city": "Anytown",
"state": "CA",
"zip": "12345"
},
"phone": [
{
  "type": "home",
  "number": "555-1234"
},
{
"type": "work",
"number": "555-5678"
}
]
}
\end{lstlisting}
          In this example, the JSON object represents a person named John Doe, including their name, age, address, and phone numbers.
    \item Mention what are the data types supported by JSON?\\
          JSON supports a limited set of data types, including:
          \begin{enumerate}
              \item string: a sequence of characters enclosed in double quotes
              \item number: an integer or floating-point value
              \item boolean: either true or false
              \item null: a special value representing "no value"
              \item array: an ordered list of values, enclosed in square brackets [ ]
              \item object: an unordered collection of key-value pairs, enclosed in curly braces { }
          \end{enumerate}

    \item List out the uses of JSON?\\

          JSON is a widely used format for data interchange between systems, and it has many applications, including:
          \begin{enumerate}
              \item Web APIs: JSON is often used as the data format for APIs that provide access to web services and data.
              \item Configuration files: JSON can be used to store configuration data for applications and systems.
              \item Data storage: JSON can be used to store data in databases or file systems.
              \item Data exchange: JSON can be used to exchange data between different programming languages and platforms.
              \item Front-end development: JSON can be used to represent data in web applications and JavaScript programs.
          \end{enumerate}
\end{enumerate}
\end{document}