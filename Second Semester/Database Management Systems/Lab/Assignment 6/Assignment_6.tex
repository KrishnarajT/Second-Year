% This is a Basic Assignment Paper but with like Code and stuff allowed in it, there is also url, hyperlinks from contents included. 

\documentclass[11pt]{article}

% Preamble

\usepackage[margin=1in]{geometry}
\usepackage{amsfonts, amsmath, amssymb}
\usepackage{fancyhdr, float, graphicx}
\usepackage[utf8]{inputenc} % Required for inputting international characters
\usepackage[T1]{fontenc} % Output font encoding for international characters
\usepackage{fouriernc} % Use the New Century Schoolbook font
\usepackage[nottoc, notlot, notlof]{tocbibind}
\usepackage{listings}
\usepackage{xcolor}
\usepackage{blindtext}
\usepackage{hyperref}
\hypersetup{
    colorlinks=true,
    linkcolor=black,
    filecolor=magenta,      
    urlcolor=cyan,
    pdfpagemode=FullScreen,
    }

\definecolor{codegreen}{rgb}{0,0.6,0}
\definecolor{codegray}{rgb}{0.5,0.5,0.5}
\definecolor{codepurple}{rgb}{0.58,0,0.82}
\definecolor{backcolour}{rgb}{0.95,0.95,0.92}

\lstdefinestyle{mystyle}{
    backgroundcolor=\color{backcolour},   
    commentstyle=\color{codegreen},
    keywordstyle=\color{magenta},
    numberstyle=\tiny\color{codegray},
    stringstyle=\color{codepurple},
    basicstyle=\ttfamily\footnotesize,
    breakatwhitespace=false,         
    breaklines=true,                 
    captionpos=b,                    
    keepspaces=true,                 
    numbers=left,                    
    numbersep=5pt,                  
    showspaces=false,                
    showstringspaces=false,
    showtabs=false,                  
    tabsize=2
}

\lstset{style=mystyle}

% Header and Footer
\pagestyle{fancy}
\fancyhead{}
\fancyfoot{}
\fancyhead[L]{\textit{\Large{Database Management Systems Assignment 6}}}
%\fancyhead[R]{\textit{something}}
\fancyfoot[C]{\thepage}
\renewcommand{\footrulewidth}{1pt}



% Other Doc Editing
% \parindent 0ex
%\renewcommand{\baselinestretch}{1.5}

\begin{document}

\begin{titlepage}
    \centering

    %---------------------------NAMES-------------------------------

    \huge\textsc{
        MIT World Peace University
    }\\

    \vspace{0.75\baselineskip} % space after Uni Name

    \LARGE{
        Database Management Systems\\
        Second Year B. Tech, Semester 4
    }

    \vfill % space after Sub Name

    %--------------------------TITLE-------------------------------

    \rule{\textwidth}{1.6pt}\vspace*{-\baselineskip}\vspace*{2pt}
    \rule{\textwidth}{0.6pt}
    \vspace{0.75\baselineskip} % Whitespace above the title



    \huge{\textsc{
            Stored Procedures and Functions in PL/SQL
        }} \\



    \vspace{0.5\baselineskip} % Whitespace below the title
    \rule{\textwidth}{0.6pt}\vspace*{-\baselineskip}\vspace*{2.8pt}
    \rule{\textwidth}{1.6pt}

    \vspace{1\baselineskip} % Whitespace after the title block

    %--------------------------SUBTITLE --------------------------	

    \LARGE\textsc{
        Assignment No. 6
    } % Subtitle or further description
    \vfill

    %--------------------------AUTHOR-------------------------------

    Prepared By
    \vspace{0.5\baselineskip} % Whitespace before the editors

    \Large{
        Krishnaraj Thadesar \\
        Cyber Security and Forensics\\
        Batch A1, PA 20
    }


    \vspace{0.5\baselineskip} % Whitespace below the editor list
    \today

\end{titlepage}


\tableofcontents
\thispagestyle{empty}
\clearpage

\setcounter{page}{1}

\section{Aim}
Write PLSQL Procedures and Function for given problem statements

\section{Objectives}
\begin{enumerate}
    \item To study PLSQL procedures and functions
\end{enumerate}


\section{Problem Statement}
Create tables and solve given queries using Subqueries

\section{Theory}
\subsection{PL/SQL}

PL/SQL is Oracle's procedural extension to industry-standard SQL. PL/SQL naturally, efficiently, and safely extends SQL for developers. Its primary strength is in providing a server-side, stored procedural language that is easy-to-use, seamless with SQL, robust, portable, and secure.

\subsection{Stored Procedures}

A stored procedure is a subroutine available to applications that access a relational database system. Stored procedures (sometimes called a proc, sproc, StoPro, or SP) are actually stored in the database data dictionary.

\subsection{Functions}

A function is a subroutine available to applications that access a relational database management system (RDBMS). Such applications can include multiple programming languages, APIs, and communication protocols. Functions are also called procedures, modules, or subroutines. Functions are stored in and callable from the database.

\subsection{Difference between Stored Procedures and Functions}

The following are the key differences between a stored procedure and a function.

\begin{enumerate}
    \item A function must return a value but in Stored Procedure it is optional.
    \item A function can have only input parameters for it whereas a stored procedure can have input/output parameters .
    \item Functions can be called from Procedure whereas Procedures cannot be called from a Function.
    \item Exception can be handled by try-catch block in a Procedure whereas try-catch block cannot be used in a Function.
    \item We can go for Transaction Management in Procedure whereas we can't go in Function.
    \item We can use a procedure in a select statement but we can't use Function in a select statement.
    \item We can't use a function in DML (insert, update, delete) statement. But we can use a procedure in DML statement.
\end{enumerate}


\section{Platform}

\textbf{Operating System}: Arch Linux x86-64 \\
\textbf{IDEs or Text Editors Used}: Draw.io for Drawing the ER diagram. \\

\section{Input}
Given Database from the Problem Statement for the Assignment for our batch. (A1 PA 20)
\section{Creation and Insertion of Values in the Tables}

\section{Queries}
\lstinputlisting[language=SQL]{../../Programs/Assignment_6_queries.sql}

\section{Outputs}
\lstinputlisting[language=SQL]{../../Programs/Assignment_6.md}

\section{Conclusion}
Thus, we have learned PLSQL Database Programming.

\clearpage

\section{FAQ}
\begin{enumerate}
    \item \textbf{What is PLSQL? What are Applications of PLSQL?}\\
    
          PL/SQL (Procedural Language/Structured Query Language) is a procedural extension of SQL that is used to write and execute program units such as stored procedures, functions, and triggers in Oracle Database. It offers a wide range of features such as exception handling, variable declaration, loops, conditional statements, and more, which make it a powerful tool for developing complex database applications. The applications of PL/SQL include building database applications, automating database administration tasks, creating reports, and more.

    \item \textbf{What is deterministic in stored funcions mean?}\\
    
          In PL/SQL, a stored function is deterministic if it always returns the same result for the same set of input parameters. This means that if the input parameters for a deterministic function remain the same, the function will always return the same output. This property is important because it enables developers to write functions that can be used in a wider range of contexts and can be optimized by the Oracle database engine.

    \item \textbf{Explain Various Input Parameter in PLSQL}\\
    
          Various input parameters in PL/SQL include:

          \begin{enumerate}
            \item IN: This parameter is used to pass values into a stored procedure or function. The values of the IN parameter are read-only within the program unit and cannot be modified.
            
            \item OUT: This parameter is used to return values from a stored procedure or function. The values of the OUT parameter are write-only within the program unit and must be assigned a value before the program unit completes.
            
            \item IN OUT: This parameter is used to pass values into a stored procedure or function and return values back to the calling program. The values of the IN OUT parameter can be read and modified within the program unit.
            
            \item DEFAULT: This parameter is used to provide a default value for a parameter. If a value is not specified for the parameter when the program unit is called, the default value will be used instead.
        \end{enumerate}
    \end{enumerate}



\end{document}